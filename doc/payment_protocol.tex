\documentclass{article}
\usepackage{amsmath}
\usepackage{algorithm}
\usepackage{algorithmicx}
\usepackage[noend]{algpseudocode}
\usepackage{parskip}
\usepackage{varwidth}
\usepackage{placeins}
\usepackage{xcolor}

\newcommand{\flo}[1]{ {\color{blue} [[FR:#1]]}}
\newcommand{\thi}[1]{ {\color{red} [[TD:#1]]}}

\algrenewcommand{\algorithmiccomment}[1]{{\color{gray}$\triangleright$ #1}}

\begin{document}

\section{Overview}
Outline of where it fits into the entire incentive system
\section{Bolt Micropayments}

\section{Nanopayment Extension}

\section{Primitives}

\section{Security Model}

\subsection{Economic Security}

\subsection{Anonymity}

\section{API}

\section{Variable Index}\label{variable index}

In following with the Bolt style, variables in the section
\ref{algorithms} are separated into two levels. The first level of
variables are used for actual cryptographic and accounting
operations. These are bundled into groups of higher level variable
names meant to represent abstraction concepts such as payment channels
and party states. Only these high level variables are saved outside
the context of the algorithms.

\textbf{Shared}

$nT = (\delta_C, \delta_R, n, hc^0)$ --- ``Nanopayment Channel Token'' ---
This token stores the static, public information that defines a
nanopayment channel including the payment values on both legs, the max
number of payments, and the hashchain head. This can be passed around
freely by all parties.

\textbf{Client}

$ncsk_C = (nwpk_C, nwsk_C, HC)$ --- ``Client Nanopayment Secrets'' --- The
public/private key pair allows the client to setup and close a
nanopayment channel while the precomputed hash chain stores the
ability to make incremental nanopayments

$nS_C = (k, hc^k)$ --- ``Client Nanopayment State'' --- The mutable state
of the nanopayment is simply the count of payments made so far and the
latest sent hash pre-image

$nrt_C$ --- ``Client Nanopayment Refund'' --- The refund token allows the
client to make a claim to the ledger on escrowed money. This refund is
signed by the intermediary and conditioned on revealing the latest
hash pre-image that the client claims to have sent.

$nrc_C$ --- ``Client Channel Closure Message'' --- Final message that is
posted to the ledger by the client to claim all funds of the
micropayment channel including any completed nanopayments.

\textbf{Intermediary}

$nS_I = \{nT: channel\_state\}$ --- ``Intermediary Nanopayment State''
The intermediary's nanopayment state is a map of all past and present
nanopayment channels and the corresponding channel state. Possible
states are:
\begin{itemize}
\item $\bot$ --- failed attempt at setting up a nanopayment channel
\item $ready$ --- channel has been set up by the sender (client)
\item $established$ --- channel has been established with a recipient
  (relay)
\item $closed||hc^k$ --- channel has been closed and no further payments
  are allowed
\end{itemize}

\textbf{Relay}

$ncsk_R (nwpk_C, nwsk_C, \bot)$ --- ``Relay Nanopayment Secrets'' --- The
public/private key pair allows the relay to setup and close a
nanopayment channel. Since relays cannot make payments in this setup,
the last field is left blank.

$nS_R = (k, hc^k)$ --- ``Relay Nanopayment State'' --- The mutable state
of the nanopayment is simply the count of payments made so far and the
latest received hash pre-image

$nrt_R$ --- ``Relay Nanopayment Refund'' --- The refund token allows the
relay to make a claim to the ledger on escrowed money. This refund is
signed by the intermediary and conditioned on revealing the latest
hash pre-image that the relay has received.

$nrc_C$ --- ``Client Channel Closure Message'' --- Final message that is
posted to the ledger by the relay to claim all funds of the
micropayment channel including any completed nanopayments.


\flo{How does the transaction happens on the ledger?}  \thi{Basically,
  the client/relay would send the refund token to the ledger and that
  transaction would be posted. There would be some delay period in
  which the intermediary might want to refute the claim by posting a
  revocation token and it should be obvious to the ledger/network that
  one of them is right. These steps should probably be formalized in
  nRefund, nRefute, and nResolve algorithms}
\section{Algorithms}\label{algorithms}

Conventions:
\begin{itemize}
\item All variable names in this section, except for possibly helper
  functions, are globally unique.
\item Variable subscripts denote a party or role ((I)ntermediary,
  (C)lient, (R)elay, (E)nd user).
\item New nanopayment variables are prefixed with the character
  (n). All other variables reference a value from the original Bolt
  scheme, although the name might be altered somewhat.
\item Payment values ($\epsilon, \delta$) are signed integers with
  respect to the end user. For example, $\delta_C$ is negative and
  $\delta_R$ is positive in the typical case where a client is paying
  a relay.
\end{itemize}

\begin{algorithm}
  \begin{algorithmic}[1]
    \caption{Collection of functions called by other algorithms in this section}
    \Function{CommitWallet}{$pp, pk_{payee}, w, \epsilon$}
    \State{parse $w$ as $(B, wpk, wsk, r, \sigma^w)$}
    \State{$(wsk', wpk') \gets $KeyGen$(pp)$}
    \State{$r' \gets $Random$()$}
    \State{$wCom' \gets $Commit$(wpk', B + \epsilon, r')$}
    \State{$\pi \gets PK\{(wpk', B, r', \sigma^w)$: \par}
    \State{\hskip\algorithmicindent{} $wCom' = $Commit$(wpk', B + \epsilon, r')\ \wedge$}
    \State{\hskip\algorithmicindent{} Verify$(pk_{payee}, (wpk, B), \sigma^w)\ \wedge$}
    \State{\hskip\algorithmicindent{} $B + \epsilon \geq 0$}
    \State{\Return{$(wsk', wpk', wCom', \pi)\}$}}
    \EndFunction{}

    \flo{(7) is redundant with (5)?}  \thi{Not quite; line (5) is where
      the commitment is actually created. Line (7) is part of a
      zero-knowledge proof $(\pi)$ that shows that the commitment was
      properly constructed since the other party ($I$ in this case)
      doesn't know wpk', B, or r'}

    \Function{MakeHashChain}{$hc^0, n$}
    \State{$HC \gets \{n : hc^0\}$}
    \For{$i = \{n-1, \ldots, 0\}$}
      \State{$HC[i] = $Hash$(HC[i+1])$}
    \EndFor{}
    \State{Return{$HC$}}
    \EndFunction{}

    \Function{VerifyHashChain}{$hc^0, k, hc^k$}
    \State{$hc^i \gets hc^k$}
    \For{$i \gets \{k-1, \ldots , 0\}$} \Comment{verify hash chain}
      \State{$hc^i \gets Hash(hc^i)$}
    \EndFor{}
    \If{$hc^i = hc^k$}
      \State{\Return{1}}
    \EndIf{}
    \State{\Return{0}}
    \EndFunction{}

  \end{algorithmic}
\end{algorithm}

\begin{algorithm}
  \caption{Nanopayment Channel Setup --- Protocol between a relay and
    intermediary to create a new nanopayment channel from an existing
    micropayment wallet. This can be run prior to circuit setup.}
  \begin{algorithmic}[1]
    \Procedure{Client}{$pp, pk_I, w_C, \delta_C, n$}
      \State{parse $w_C$ as $(B_C, wpk_C, wsk_C, r_C, \sigma^w_C)$}
      \If{$B_{C} + (\delta_C * n) < 0$}
        \flo{is $\delta_C$ supposed to be negative as you said above?}
        \thi{Yep, just fixed it. Thanks!}
        \State{Abort$()$ \Comment{consider opening a new micropayment channel}}
      \EndIf{}
      \State{$(nwpk_C, nwsk_C, nwCom_C, n\pi_C) \gets$ CommitWallet$(pp, pk_I, w_C, \delta_C * n)$}
      \State{$\delta_R \gets -(\delta_C  + tax)$} \Comment{the tax represents a net profit for the intermediary}
      \State{$HC \gets $MakeHashChain$($Random$(), n)$}
      \State{$nT \gets (\delta_C, \delta_R, n, HC[0])$}
      \State{Intermediary.Send$(wpk_C, nwpk_C, nwCom_C, n\pi_C, nT)$}
    \EndProcedure{}

    \Procedure{Intermediary}{$pp, S_I, nS_I$}
      \State{$(wpk_C, nwpk_C, nwCom_C, n\pi_C, nT) \gets $Client.Receive$()$}
      \State{parse $nT$ as $(\delta_C, \delta_R, n, hc^0)$}
      \If{$wpk_C \in S_I \vee \neg $Verify$(n\pi_C)$}
        \State{Abort$()$ \Comment{invalid wallet}}
      \EndIf{}
      \If{$-\delta_C \ne price \vee \delta_R + \delta_C + tax \ne 0$}
        \State{Abort$()$ \Comment{incorrect payment prices}}
      \EndIf{}
      \State{$S_I \gets S_I \cup \{wpk_C : \bot, nwpk_C: \bot\}$}
      \State{$nS_I \gets nS_I \cup \{nT : \bot\}$}
      \State{Client.Send$(verified)$}
    \EndProcedure{}

    \Procedure{Client}{}
      \State{$ver \gets $Intermediary.Receive$()$}
      \State{$nrt_C \gets $Intermediary.Blindsig$(ver, refund || nT || nwpk_C || B + (\delta_C * n))$}
      \State{$nS_C \gets (0, HC[0])$}
      \State{$ncsk_C \gets (nwpk_C, nwsk_C, HC)$}
      \State{$\sigma^{rev(w)}_C \gets $Sign$(wsk_C, revoke||wpk_C)$}
      \State{Intermediary.Send$(\sigma^{recv(w)}_C$)}
    \EndProcedure{}

    \Procedure{Intermediary}{}
      \State{$\sigma^{recv(w)}_C \gets $Client.Receive$()$}
      \If{$\neg $Verify$(wpk, revoke||wpk_C, \sigma^{recv(w)}_C) = 1$}
        \State{Abort$()$ \Comment{invalid revocation token}}
      \EndIf{}
      \State{$S_I[wpk_C] \gets \sigma^{recv(w)}_C$}
      \State{$nS_I[nT] \gets ready$}
      \State{Client.Send$(established)$}
    \EndProcedure{}

  \end{algorithmic}
\end{algorithm}

\begin{algorithm}
  \caption{Nanopayment Channel Establish --- Protocol between a
    client, intermediary, and relay to establish the nanopayment
    channel between the client and relay. This should be run at the
    start of circuit setup.}
  \begin{algorithmic}[1]

    \Procedure{Client}{$nT$}
      \State{Relay.Send$(nT)$}
    \EndProcedure{}

    \Procedure{Relay}{$pp, pk_I, B_{I:B}, w_R$}
      \State{$nT \gets $Client.Receive$()$}
      \State{parse $w_R$ as $(B_R, wpk_R, wsk_R, r_R, \sigma^w_R)$}
      \State{parse $nT$ as $(\delta_C, \delta_R, n, hc^0)$}
      \If{$B_{I:B} - (\delta_B * n) < 0$}
        \State{Abort$()$ \Comment{consider opening a new micropayment channel}}
      \EndIf{}
      \State{$(nwpk_R, nwsk_R, nwCom_R, n\pi_R) \gets $CommitWallet$(pp, pk_I, w_R, \delta_R * n)$}
      \State{Intermediary.Send$(wpk_R, nwpk_R, nwCom_R, n\pi_R, nT)$}
    \EndProcedure{}

    \Procedure{Intermediary}{$pp, S_I, nS_I$}
      \State{$(wpk_R, nwpk_R, nwCom_R, n\pi_R, nT) \gets $Relay.Receive$()$}
      \State{parse $nT$ as $(\delta_C, \delta_R, n, hc^0)$}
      \If{$wpk_R \in S_I \vee \neg $Verify$(n\pi_R)$}
        \State{Abort (invalid wallet)}
      \EndIf{}
      \If{$nS_I[nT] \ne ready$}
        \State{Abort (unregistered nanopayment channel)}
      \EndIf{}
      \State{$S_I \gets S_I \cup \{nwpk_R, \bot\}$}
      \State{$nS_I[nT] \gets established$}
      \State{Relay.Send$(verified)$}
    \EndProcedure{}

    \Procedure{Relay}{}
      \State{$ver \gets $Intermediary.Receive$()$}
      \State{$nrt_R \gets $Intermediary.Blindsig$(ver, refund || nT || nwpk_R || B_R + (\delta_R * n))$}
      \State{$ncsk_R \gets (nwpk_R, nwsk_R, \bot)$} \Comment{3rd null value to match client format}
      \State{$nS_R \gets (0, hc^0)$}
    \EndProcedure{}
  \end{algorithmic}
\end{algorithm}

\begin{algorithm}
  \caption{Nanopayment Channel Pay --- Protocol between the client and
    relay to forward a single nanopayment. This should be run
    periodically throughout the lifetime of the circuit.}
  \begin{algorithmic}[1]

    \Procedure{Client}{$nT, ncsk_C, nS_C$}
      \State{parse $nT$ as $(\delta_C, \delta_R, n, hc^0)$}
      \State{parse $ncsk_C$ as $(nwpk_C, nwsk_C, HC)$}
      \State{parse $nS_C$ as $(k, hc^k)$}

      \If{$k >= n$}
        \State{Abort$()$ \Comment{out of nanopayments to send, setup a new channel}}
        \flo{nClose here?}
        \thi{Maybe... I'm not sure whether it makes more sense to specify it locally here or push it off to the ``higher level'' control loop of whoever is calling this algorithm. We'd also have to send a signal to the relay to close his channel first and then go through setup/establish steps again. Also, to save cpu cycles, this would be where we look ``rolling into the next nanopayment channel'' efficiency shortcuts that Dan was talking about}
      \EndIf{}

      \State{$nS_C \gets (k+1, HC[k+1])$}
      \State{Relay.Send$(HC[k+1])$}
    \EndProcedure{}

    \Procedure{Relay}{$nT, nS_R$}
      \State{$hc^{k+1} \gets $Client.Receive$()$}
      \State{parse $nS_R$ as $(k, hs^k)$}
      \If{$k+1 >= n \vee Hash(hc^{k+1}) \ne hc^k$}
        \State{Abort$()$ \Comment{invalid nanopayment}}
      \EndIf{}
      \State{$nS_R \gets (hs^{k+1}, k+1)$}
    \EndProcedure{}
  \end{algorithmic}
\end{algorithm}

\begin{algorithm}
  \caption{Nanopayment Channel Close --- Protocol between an end user
    (client or relay) and an intermediary to close out the nanopayment
    channel and receive a micropayment wallet. This should be done any
    time after the circuit closure and the relay must close first}
  \begin{algorithmic}[1]
    \State{$\forall E \in \{Client, Relay\}$}
    \Procedure{EndUser}{$pp, pk_I, w_E, nT, ncsk_E, nS_E$}
      \State{parse $w_E$ as $(B_E, wpk_E, wsk_E, r, \sigma^w_E)$}
      \State{parse $nT$ as $(\delta_C, \delta_R, n, hc^0)$}
      \State{parse $ncsk_E$ as $(nwpk_E, nwsk_E, \_)$}
      \State{parse $nS_E$ as $(k, hc^k)$}
      \State{$\epsilon_E \gets \delta_C * k$ if (EndUser = Client) else $ \delta_R * k$}
      \State{$(wpk'_E, wsk'_E, wCom'_E, \pi'_E) \gets $CommitWallet$(pp, pk_I, wpk_B, \sigma^w_E, B_E, \epsilon_E$)}
      \State{Intermediary.Send$(wpk_E, wCom'_E, \pi_E, nT, \epsilon_E, k, hc^k)$}
    \EndProcedure{}

    \Procedure{Intermediary}{$pp, S_I, nS_I$}
      \State{$(wpk_E, wCom'_E, \pi_E, nT, \epsilon_E, k, hc^k) \gets $EndUser.Receive$()$}
      \State{parse $nT$ as $(\delta_C, \delta_R, n, hc^0)$}
      \If{$\epsilon_E < 0 \wedge closed \not\in nS_I[nT]$}
        \State{Abort$()$ \Comment{client attempting to close before relay}}
      \EndIf{}
      \If{Verify$(\pi_E) \vee nS_I[nT] \ne established$}
        \State{Abort$()$ \Comment{invalid wallet or channel}}
      \EndIf{}
      \If{$k > n \vee \neg $VerifyHashChain$(hc^0, k, hc^k)$}
        \State{Abort$()$ \Comment{invalid payment hash chain}}
      \EndIf{}
      \State{$nS_I[nT] \gets closed||hc^k$}
      \State{EndUser.Send$(verified)$}
    \EndProcedure{}

    \Procedure{EndUser}{}
      \State{$ver \gets $Intermediary.Receive$()$}
      \State{parse $ncsk_E$ as $(nwpk_E, nwsk_E, \bot)$}
      \State{$rt'_E \gets $Intermediary.Blindsig$(ver, refund || wpk'_E || B_E + (\delta_E * k))$}
      \State{$\sigma^{rev(nrt)}_E \gets $Sign$(nwsk_E, revoke||nwpk_E)$}
      \State{Intermediary.Send$(nwpk_E, \sigma^{rev(nrt)})$}
    \EndProcedure{}

    \Procedure{Intermediary}{}
      \State{$(nwpk_E, \sigma^{rev(nrt)}_E) \gets $EndUser.Receive$()$}
      \If{$nwpk_E n\in S_I \vee \neg $Verify$(nwpk_E, \sigma^{rev(nrt)})$}
        \State{Abort$()$ \Comment{unregistered nanopayment channel or revocation token}}
      \EndIf{}
      \State{$S_I[nwpk_E] \gets \sigma^{rev(nrt)}$}
      \State{EndUser.Send$(verified)$}
    \EndProcedure{}

    \Procedure{EndUser}{}
      \State{$ver \gets $Intermediary.Receive$()$}
      \State{$w'_E \gets $Intermediary.Blindsig$(ver, wpk_E'||B_E + \epsilon_E)$}
    \EndProcedure{}

  \end{algorithmic}
\end{algorithm}

\begin{algorithm}
  \caption{Nanopayment Refund --- Algorithm by an end user to close a micropayment
    channel and claim ledger funds. This is a modified version of
    Bolt's Refund algorithm to also allows for granular claims on
    open nanopayment channels}
  \begin{algorithmic}[1]
    \State{$\forall E \in \{Client, Relay\}$}
    \Function{EndUser}{$pp, csk_E, w_E, nT, ncsk_E, nS_E, nrt_E$}
    \State{parse $csk_E$ as $(\_, sk_E, \_, \_, \_, \_)$}
    \State{parse $w_E$ as $(B_E, \_, \_, \_, \_)$}
    \State{parse $nT$ as $(\delta_C, \delta_B, \_, \_)$}
    \State{parse $ncsk_E$ as $(nwpk_E, \_, \_)$}
    \State{parse $nS_E$ as $(k, hc^k)$}
    \State{$\delta_E \gets \delta_C$ if (EndUser = Client) else $ \delta_R$}
    \State{$m_E \gets (refund || nT || nwpk_E || B_E + \delta_E * n, nrt_E, hc^k_E, k_E)$}
    \State{$nrc_E \gets (m_E, Sign(sk_E, m_E))$}
    \State{\Return{$nrc_E$}}
    \EndFunction{}
\end{algorithmic}
\end{algorithm}

\begin{algorithm}
  \caption{Nanopayment Refute --- Algorithm by an intermediary to respond to an end user's refund claim by posting its own channel closure message to the ledger}
  \begin{algorithmic}[1]
    \State{$\forall E \in \{Client, Relay\}$}
    \Function{Intermediary}{$pp, T_E, S_I, nS_I, nrc_E$}
    \State{parse $nrc_E$ as $(m_E, \sigma^m_E)$}
    \State{parse $m_E$ as $(refund || nT || nwpk_E || B^{full}_E, nrt_E, k_E, hc^k_E)$}
    \State{\Comment{$B^{full}_E$ is the  balance of the nanopayment channel were fully saturated}}
    % however we get nrc_E... need to decide
    \State{parse $T_E$ as $(pk_E, \_)$}
    \If{$\neg$Verify$(pk_E, m_E, \sigma^m_E)$}
      \State{Abort$()$ \Comment{bad signature, should not be accepted by the ledger}}
    \EndIf{}
    \If{$\neg$Verify$(pk_I, (refund || nT || nwpk_E || B_E + \delta_E * n), nrt_E)$}
      \State{Abort$()$ \Comment{never approved this refund token, won't be accepted}}
    \EndIf{}
    \If{$S_I[nwpk_E] \ne \bot$}
      \State{\Comment{$E$ is posting an old token, $I$ should refute to claim penalty}}
      \State{$\sigma^{rev(nrt)}_E \gets S_I[nwpk_E]$}
      \State{$nrc_I \gets ((revoked, \sigma^{rev(nrt)}_E), Sign((revoked, \sigma^{rev(nrt)})))$}
    \EndIf{}
    \State{\Comment{Otherwise accept the closure and post latest preimage}}
    \State{$hc^k \gets nS_I[nT]$}
    \State{$nrc_I \gets ((accepted, k_I, hc^k_I), Sign(accepted, k_I, hc^k_E))$}
    \State{\Return{$nrc_I$}}
    \EndFunction{}
  \end{algorithmic}
\end{algorithm}

\begin{algorithm}
  \caption{Nanopayment Resolve --- Algorithm run by the ledger (and everyone verifying the ledger) to resolve all channel closure messages and allocate the appropriate final balances}
  \begin{algorithmic}[1]
    \State{\Comment{Returns the tuple $(B^{final}_E, B^{final}_I)$}}
    \Function{Ledger}{$pp, T_E, T_I, nrc_E, nrc_I$}
    \State{$B^{total} = B_E^{init} + B_I^{init}$}
    \State{parse $nrc_E$ as $(m_E, \sigma^m_E)$}
    \State{parse $nrc_I$ as $(m_I, \sigma^m_I)$}
    \State{parse $m_E$ as $(refund || nT || nwpk_E || B^{full}_E, nrt_E, k_E, hc^k_E)$}
    \State{\Comment{$B^{full}_E$ is the  balance of the nanopayment channel were fully saturated}}
    \State{parse $nT$ as $(\delta_C, \delta_R, n, hc^0)$}
    \State{$\delta_E \gets \delta_C$ if (EndUser = Client) else $ \delta_R$}

    \If{$nrc_E = \bot$}
      \State{\Comment{$E$ failed to respond to intermediary closure request in time}}
      \State{\Return{$(0, B_{total})$}}
    \EndIf{}
    \If{$\neg $Verify$(pk_E, m_E, \sigma^m_E) \vee \neg $Verify$(pk_I, m_I, \sigma^m_I)$}
      \State{\Return{$\bot$} \Comment{messages could not be authenticated}}
    \EndIf{}
    \If{$\neg $Verify$(pk_I, refund || nT || nwpk_E || B^{full}_E, nrt_E)$}
      \State{\Return{$(0, B_{total})$} \Comment{$E$ is attempting to use invalid token}}
    \EndIf{}
    \If{$revoked \in m_I$}
      \State{parse $m_I$ as $(revoked, \sigma^{rev(nrt)}_E)$}
      \If{Verify$(nwpk_E, \sigma^{rev(nrt)}_E)$}
        \State{\Return{$(0, B_{total})$} \Comment{$E$ is trying to use old channel}}
        \Else{}
        \State{\Return{$(B_{total}, 0)$} \Comment{$I$ submitted an invalid revocation token}}
      \EndIf{}
    \EndIf{}
    \State{\Comment{micropayments have been settled, now resolve nanopayments}}
    \State{parse $m_I$ as $(accepted, k_I, hc^0_I)$}
    \If{$k_I \leq k_E \leq n \wedge $VerifyHashChain$(hc^0, k_E, hc^k_E) )$}
      \State{\Comment{$E$ has the highest hash preimage}}
      \State{\Return{$(B^{full}_E - \delta_E * (n-k_E), B_{total} - B^{full}_E + \delta_E * (n-k_E))$}}
    \EndIf{}
    \If{$k_E \leq k_I \leq n \wedge $VerifyHashChain$(hc^0, k_I, hc^k_I) )$}
      \State{\Comment{$I$ has the highest hash preimage}}
      \State{\Return{$(B^{full}_E - \delta_E * (n-k_I), B_{total} - B^{full}_E + \delta_E * (n-k_I))$}}
    \EndIf{}

    \State{\Return{$\bot$} \Comment{something is wrong, final state is inconclusive}}

    \EndFunction{}
  \end{algorithmic}
\end{algorithm}


\section{Implementation notes}

\subsection{Tor overview}

I am going to explain here a high-level overview of the Tor implementation logic. The idea is to get both the intuition about how the payment protocol is going to be incorporated in the Tor codebase, where it is going to be tricky and what part does not really need to be fully implemented (e.g. maybe we can assume no cheater $\rightarrow$ no need to code all logic about detect \& punish users).

Tor can be abstracted by a event-driven network process that responds to network events to tight together instance of the Tor program through what we call ''circuits". Every Tor instance share the same codebase but respond differently to network events,  following some pieces of the Tor protocol\footnote{https://gitweb.torproject.org/torspec.git/tree/tor-spec.txt} depending of their configuration. The payment protocol must be seen as an extension of the current Tor protocol, hence a good knowledge overview of the Tor protocol is somewhat necessary.

Tor is mostly single-threaded and pulls work from a queue of events constructed by Libevent. Tor handles CPU-intensive tasks in worker threads (note: checks how Tor hanldes multiple CPU and if we can use other CPUs for channel opening/closing).

The Tor control protocol\footnote{https://gitweb.torproject.org/torspec.git/tree/control-spec.txt} is used by other programs (such as a browser) to communicate with the Tor process. Enabling/disabling payments should work through an extension of this protocol.

A lot of implementation documentation/details/logic are provided in the torguts repository: https://gitweb.torproject.org/user/nickm/torguts.git/. 

Most of our implementation is going to use/extend *\_edge.c files from src/or/ since every payment logic happens at "edges" of Tor circuits. A circuit edge t is defined w.r.t to the cell being forwarded through the circuit. E.g. a 2-time onion encrypted cell is going to be recognized by the middle relay, hence the edge of the circuit for that particular cell is the middle relay. This is also called "leaky-pipe design" in the original Tor paper.

\subsection{Payment specifications}
	\label{sec:payment_spec}
	
 Overall idea: Algorithms above are going to be written is standalone functions in some new file in "src/or/". When some cells related to a payment arrive on circuit, circuit\_receive\_relay\_cell() from src/or/relay.c is going call the entry point of payment management (e.g. circuit\_manage\_payment()) and manage itself circuit states, payment states, etc.

In normal use of circuit, when the payment is enabled, each client-relay handle a payment flow control in a similar way that bandwidth flow control is currently designed in Tor (using windows). The idea is that the client uses the nanopayment channel with the relay for maintaining this window above 0. If the window reaches 0, the relay stop prioritize the traffic from this circuit.
 \subsubsection{Cell specification}
 \label{sec:cell_spec}
 
 \subsubsection{Circuit specification}
 
 	\paragraph*{Intermediary circuit}
 	
 	\paragraph*{Payment-flow control}
 	
 \subsubsection{Intermediary specification}
 	\#Note: explains how a particular relay chooses intermediaries
\subsection{Payment scheduler specification}
	\#Note: 
\end{document}
