\documentclass{article}
\usepackage{parskip}

\title{Overview of Payment Modules and API}
\date{}

\begin{document}
\section{Front(Tor)-facing API}

This section describes the various functions required to interact with the
payment module. The rest of the Tor codebase should interact with the payment
scheme exclusively through this API.

In return, the payment modules will need to interact with the rest of the Tor
codebase sending cells to some given destination.

\subsection{Client Payment Controller}

\textbf{\emph{Init Payments}}

Inputs: Cryptocurrency keys and any existing channel info (possibly from torcc files), info to
communicate with the ledger

Trigger: Start of the Tor application.

\textbf{\emph{Establish Nanopayment Channels}}

Inputs: Circuit info (needs to include some way to reach all 3 relays
along the path)

Trigger: Some time just after circuit creation

\textbf{\emph{Make Payments}}

Inputs: Circuit info

Trigger: Called once every time the circuit has processed X number of cells

\textbf{\emph{Close Nanopayment Channels}}

Inputs: Circuit info

Trigger: Circuit close

\textbf{\emph{Cashout Channel}}

Inputs: Channel ID

Trigger: Caused by some deliberate user command.

\textbf{\emph{Handle Incoming Cell}}

Inputs: Circuit ID, cell

Trigger: Cell is received from the network

\subsection{Relay Payment Controller}

\textbf{\emph{Init Payments}}

Inputs: Cryptocurrency keys and any existing channel info (possibly from torcc files), info to
communicate with the ledger, payment processed callback function

Trigger: Start of the Tor application.

\textbf{\emph{Cashout Channel}}

Inputs: Channel ID

Trigger: Caused by some deliberate user command.

\textbf{\emph{Handle Incoming Cell}}

Inputs: Circuit info, cell

Trigger: Cell is received from the network

\textbf{\emph{**Payment Processed Callback**}};

This is a callback function that is provided to the module. It is invoked every
time the payment module processes a successful payment by a client. The callback
should handle any network-related effects related to the priority bandwidth.

\subsection{Intermediary Payment Controller}

\textbf{\emph{Init Payments}}

Inputs: Cryptocurrency keys and any existing channel info (possibly from torcc files), info to
communicate with the ledger

Trigger: Start of the Tor application.

\textbf{\emph{Cashout Channel}}

Inputs: Channel ID

Trigger: Caused by some deliberate user command.

\textbf{\emph{Handle Incoming Cell}}

Inputs: Circuit info, cell

Trigger: Cell is received from the network

\subsection{Ledger}

\textbf{\emph{Init Ledger}}

Inputs: Policy info (fees, taxes, Tor authority address, etc)

Trigger: Start of system creation

\textbf{\emph{Handle Incoming Cell}}

Inputs: Circuit info, cell

Trigger: Cell is received from the network

\section{Back End}

This section provides a brief description of other background modules needed to
support the main payment controllers.

\subsection{Micropayment  Algorithms}

These are direct implementation of the micropayment algorithms described in the
protocol documentation. There should be one function for each step in the
multi-party protocols (i.e. init1(), init2(), init3()...).

\subsection{Nanopayment Algorithms}

These are direct implementation of the nanopayment algorithms described in the
protocol documentation. There should be one function for each step in the
multi-party protocols (i.e. init1(), init2(), init3()...).

\subsection{Cryptography}

A cryptography module will serve as a libary providing cryptographic operations such as
signatures, blind signatures, and zero-knowledge proofs. By isolating the
crytographic operations here, the idea is that the fake crytographic simulations
can later be replaced with calls to real crypto libraries with minimal code
change.

\subsection{Tokens}

A tokens  module will serve as a libary which defines structs for all of the various
tokens used by the payment scheme. This is useful in it of itself as a reference
index for the absurd number of tokens flying around. Furthermore, it also
provides authenticated ``pack/unpack'' functionality to convert between
semantically meaningful c structs and byte strings that can be readily sent
across the network.

\subsection{Utils}

This final module simply provides some miscallenous common functions

\end{document}