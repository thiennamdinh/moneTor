\documentclass[USenglish,oneside,twocolumn]{article}
\makeatletter

% Don't use the obsolete fixltx2e package on newer systems (2015 or later).
\@namedef{ver@fixltx2e.sty}{2006/03/24}
\@namedef{opt@fixltx2e.sty}{}

% Don't use the multicol package (it raises a warning in twocolumn format).
\@namedef{ver@multicol.sty}{2016/04/07}
\@namedef{opt@multicol.sty}{}

% Footnote hyperlinks don't work due to a conflict between hyperref and footmisc.
% Turning them off prevents the warnings and broken links.
\PassOptionsToPackage{hyperfootnotes=false}{hyperref}
\usepackage[utf8]{inputenc}%(only for the pdftex engine)
%\RequirePackage[no-math]{fontspec}%(only for the luatex or the xetex engine)


\usepackage[big]{dgruyter_NEW}


% Don't use non-existant fonts for labels in footnotes.
\def\@makefnmark{\hbox{\@textsuperscript{\normalfont\@thefnmark}}}
\makeatother
\usepackage[utf8]{inputenc}
\usepackage{hyperref}
\usepackage{subcaption}
\newtheorem{theorem}{Theorem}
%\usepackage{algorithmicx}
\usepackage{caption}
\usepackage{algorithm}
\usepackage[noend]{algpseudocode}
\algrenewcommand{\algorithmiccomment}[1]{{\color{gray}$\triangleright$ #1}}
\makeatletter
\def\plist@algorithm{\space}
\makeatother
\usepackage[
n,
advantage,
operators,
sets,
adversary,
landau,
probability,
notions,
logic,
ff,
mm,
primitives,
events,
complexity,
asymptotics,
keys
]{cryptocode}


\hyphenation{moneTor}
\hyphenation{CIRCWINDOWSTART}
\hyphenation{STREAMWINDOWSTART}
\makeatletter
\renewcommand{\ALG@beginalgorithmic}{\small}
\makeatother

\newcommand{\flo}[1]{ {\color{red} FR: #1}}
\newcommand{\td}[1]{ {\color{blue} TD: #1}}
\newcommand{\op}[1]{ {\color{olive} OP: #1}}

\cclogo{\includegraphics{by-nc-nd.pdf}}
\begin{document}
 

\title{\huge Scaling Up Anonymous Communication with Efficient Nanopayment Channels} % TODO: replace with your title

\runningtitle{Scaling Up Anonymous Communication with Efficient Nanopayment Channels}

  %\subtitle{...}
\begin{abstract}
  { Tor, the most widely used and well-studied traffic anonymization network in
    the world, suffers from orthogonal limitations in network diversity and
    performance. We propose to mitigate both problems simultaneously through the
    introduction of a premium bandwidth market between clients and relays. To
    this end, we present moneTor: a relay incentivization scheme featuring
    anonymous, efficient, and economically robust nanopayments. Our approach
    leverages the latest advances in cryptocurrency research toward a design
    that is directly integrated into the existing Tor architecture. This work
    describes a full-stack strategy that covers economic policy, novel payment
    algorithms, and networking implementation. Through live empirical data
    collection and analysis of our emulated prototype, we argue that moneTor,
    based on short-lived channels on the top of a long-lived channel, is a
    feasible and flexible monetary incentive scheme for Tor, offering upwards of
    100\% improvements in differentiated bandwidth for paying users at
    near-optimal throughput and latency overhead. }
\end{abstract}

  \keywords{Tor, cryptocurrency, payment channels}
%  \classification[PACS]{}
 % \communicated{...}
 % \dedication{...}

  \journalname{Proceedings on Privacy Enhancing Technologies}
\DOI{Editor to enter DOI}
  \startpage{1}
  \received{..}
  \revised{..}
  \accepted{..}

  \journalyear{..}
  \journalvolume{..}
  \journalissue{..}
  
  
 \maketitle


\section{Introduction}
\label{sec:introduction}
Anonymous traffic routing through Tor remains one of the most popular
low-latency methods for censorship evasion and privacy protection
~\cite{dingledine2004tor}. In this setup, clients protect their TCP/IP metadata
and content by routing their traffic through an onion encrypted path with three
randomly selected volunteer relay nodes, referred to as a circuit. The network
presently consists of $\approx 6,400$ relays contributing over 230 Gbit/s of
bandwidth globally~\cite{portal2018tormetrics}. While Tor has proven to be a
highly effective option for privacy-seeking users, it suffers from two
orthogonal issues that are relevant to this work. The first is the broad family
of collusion attacks. These threats are relevant in scenarios where an attacker,
who controls multiple nodes or network vantage points, is probabilistically
placed in key roles along a single circuit, opening a much easier path to client
deanonymization~\cite{wright2004predecessor,murdoch2005low}. The second problem
is performance. Although the overlay protocol itself generates an inherent
overhead in network resources, Tor suffers from additional traffic congestion
that leads to suboptimal network performance~\cite{portal2018tormetrics,
  alsabah2016performance}.

One approach to mitigate these problems is to address them separately from a
networking standpoint. Indeed, a significant portion of recent research in Tor
proposes modifications to the core protocol itself, such as reengineering the
TCP+TLS part of the stack~\cite{reardon2009improving} or designing a better
kernel-aware scheduler~\cite{jansen2014never}. A second approach observes that
the anonymity and performance of the network are both proportional to the number
of nodes and users. From this perspective, the problem becomes a largely
economic question: how can we incentivize more relay participation? There is a
long line of research that explores various strategies for incentivization
spanning over the past decade. Progress in this field faces a multitude of
challenges. Consider the approach most relevant to our work: monetary
incentives. Aside from the analytically intractable set of legal and
sociopolitical obstacles, monetary payments in this environment must overcome a
trifecta of challenging constraints:

\begin{itemize}
\item \emph{Anonymity}: The paramount mission of Tor is user privacy. This
  cannot be compromised or reduced by transparent money transaction trails.
\item \emph{Payment Security}: Anonymity prohibits the formation of large
  credit and trust systems. As such, financial transactions cannot transpire
  without strong guarantees of cryptographic security.
\item \emph{Efficiency}: Tor services millions of concurrent users, all of whom
  maintain largely short-term relationships. A robust payment system must handle
  extremely lightweight and scalable payments so as to accommodate the dynamic
  and often short-lived activities of these clients.
\end{itemize}

\textbf{Present Landscape} While a number of prior works satisfy some subset of
these constraints, no single proposal has thus far been sufficiently convincing
so as to warrant further development. We speculate that the lack of a
breakthrough in this niche area is not a matter of insufficient ingenuity but
rather one of timing. In recent years, there has been an explosion of academic
research within the domain of cryptocurrencies. As of this writing, Satoshi
Nakamoto's Bitcoin whitepaper has already garnered over 3,200 references in
citing publications~\cite{nakamoto2008bitcoin}.  \footnote{The recency of this
  development is highlighted by the fact that about half of this work is dated
  after 2017}  This body of work has sparked the development of a multitude of
new techniques that will prove indispensable to our own area of research. Our
objective then, is to leverage the full extent of these innovations into a
practical next-generation incentivization strategy for Tor.

\label{sec:Contributions}
\textbf{Contributions} The moneTor scheme is a novel full-stack framework for
Tor incentives. We make the following contributions in this paper:

\begin{enumerate}
\item Discuss economic considerations for the Tor network to favor network
  diversity and to help supporting Tor as a public good
  %market control such that the Tor project can decide which notion of diversity
  %to promote
\item Introduce new highly-efficient nanopayment protocols which may be of some
  independent interest for other high-frequency payment applications
\item Detail an integration strategy to add payment
  into the existing Tor architecture and networking layer while being compatible
  with Tor's anonymity properties and efficiency
\item Collect privacy-preserving client usage data to justify design parameters
  and argue for efficacy
\item Implement a prototype extending the Tor protocol
  in the core C software, featuring more than 15k lines of code
\item Conduct network-scale simulations to analyze the performance impact of the
  embedded payment and incentivization scheme
\item Identify an efficient method of throttling to give priority users an
  advantage
\end{enumerate}

The moneTor design leverages fully-specified algorithms for some components and
well-studied existing research for all others. Consequently, we claim that the
entire technical stack is accounted for and that moneTor can be feasibly
developed today.

\paragraph*{Roadmap.} In Section~\ref{sec:background}, we draw on technical
preliminaries from two distinct fields: applied Tor research and cryptographic
payment channels. Section~\ref{sec:related_work} presents the related work. Our
contributions begin formally with Section~\ref{sec:economic}, where we first
describe a high-level economic model for the flow of money through the
network. In Section~\ref{sec:payment}, we outline the technical construction of
our nanopayment scheme at the payment protocol level. Section~\ref{sec:network}
expands the technical construction of moneTor to cover modifications at the
network level. In section~\ref{sec:analysis} we justify our design decisions
with real-world data collection from live Tor users. We continue in
Section~\ref{sec:experimentations} with a validation of our technical design,
carried out through experiments performed on a native proof-of-concept
implementation. Finally, we discuss limitations and future work in
Section~\ref{sec:limitations_futurework}, reference our source code in
Section~\ref{sec:code}, and present concluding remarks in
Section~\ref{sec:conclusion}.

%%% Local Variables:
%%% mode: latex
%%% TeX-master: "../main"
%%% End:


\section{Background}
\label{sec:background}
\subsection{Tor}

\subsubsection{Tor Architecture}
The Tor network is composed of multiple different components, and each of which
run the same code base. Volunteers can run relays and enable specific roles or
tasks such as \textit{network consensus directories}, \textit{HSDirs} and
\textit{Exit policies}. \textit{Directory authorities} and \textit{Bandwidth
  authorities} are the most important components of the network, and are only
operated by trustworthy core contributors. There are currently 9 directory
authorities which periodically reach agreement over the state of the network,
called the \textit{consensus document}. This consensus document holds the
identification information related to all available relays inside the network,
as well as the result of the authority vote (e.g., a set of \textit{flags}
associated to each relay). The bandwidth authorities constantly measure every
relay and provide to the directory authorities a measurement value for each of
them, which plays a critical role in the path selection algorithm. This paper
extends this architecture by adding two new roles needed for our monetary relay
incentivization: Intermediary and Ledger.
\subsubsection{Traffic Analysis}
Tor's threat model assumes a local adversary who can observe some fraction of
the network and can operate or compromise a number of onion routers. Tor also
assumes a local adversary who can manipulate user's streams. The adversary can,
insert, modify, delete or delay data to create observable
perturbations. Typically, by observing both ends of an anonymous stream, an
attacker can infer the participating parties using statistical correlation. The
adversary's precision can also be improved by traffic flow perturbations. This
attack is called \textit{end-to-end correlation}. Tor does not try to implement
countermeasures to this attack but strives to minimize its impact. Recently,
Rochet and Pereira~\cite{popets-dropping} showed that a silent near to perfect
and instantaneous active traffic confirmation attack exists in Tor, leveraging
an essential property of distributed system: forward compatibility. Those
results show the need to strive to reduce their impact. The approach presented
here is to induce more diversity to the Tor network, which would make traffic
analysis against a large fraction of Tor users more costly. The central goal of
this paper is to give a tool to the Tor project to shape the Tor network
diversity through monetary incentivization.

\subsubsection{Circuit handling on Tor clients}

\subsubsection{Evaluating Tor's performance}
Shadow~\cite{shadow-ndss12} is a discrete event networking simulator that allows
real, unmodified applications to run within a virtual network. It was originally
developed to conduct more accurate and large scale Tor experimentations in a
private and controlled environment. Shadow's primary advantage is the ability to
run native networking application code that interfaces with the simulator via an
external application-specific plugin. \td{first three sentences seem somewhat
  redundant, maybe shorten to two?}  Within the simulation environment itself,
Shadow faithfully mimics the real Tor network conditions including bandwidth and
latency. As a result, experiments conducted with this tool tends to be more
accurate than ones conducted over alternatives such as private universities
networks or PlanetLab~\cite{Chun:2003:POT:956993.956995}.

In this paper, we use Shadow to evaluate our payment layer extension of the Tor
protocol in order to measure its networking impact and its feasibility.
\subsubsection{Tor's scheduling}


\subsection{Cryptocurrencies}

The modern generation of decentralized digital currencies traces its roots to
Nakamoto's Bitcoin protocol~\cite{nakamoto2008bitcoin}. This family of
cryptocurrencies are characterized by use of a public distributed ledger, often
a blockchain, and a notion identity and ownership based in public key
cryptography. A wide range of base-layer payment protocols have emerged, most
notably for us the fully programmable smart contract platform
Ethereum~\cite{wood2014ethereum} and anonymity-focused schemes such as
Zerocash~\cite{sasson2014zerocash} and Cryptonote~\cite{van2013cryptonote}. In
this second class of anonymous currencies, the general security model requires
that user is able to upload verfiable and irreversible proof of payment to the
ledger without leaking sender identity, recipient identity, or payment
value. Cryptonote achieves a weaker probablistic guarantee of this privacy using
a combination of lightweight ring signatures and stealth addresses. Zerocash
achieves anonymity by leveraging more flexible but expensive non-interactive
zero-knowledge proofs.

As more niches have emerged for types of cryptocurrencies, one active area of
research is ledger interoperability. Back et al. published the first primitive
proposal for sidechains, which specifies how bitcoins can be migrated from the
main blockchain onto a secondary chain~\cite{back2014enabling}.  More recently,
Poon and Buterin explained how Ethereum can support multiple levels of
arbirarily configured \emph{child chains} that inherit many notions of security
from the main Ethereum he.

The core cryptocurrency component featured in moneTor are multi-party
bidirectional micropayment channels. Base layer cryptocurrency protocols suffer
from severe scalability limits as they typical are capped on the order of tens
of transactions per second~\cite{team2017blockchain}. As base layer scaling
solutions inevitably face fundamental limitations, the most actively pursued
path thusfar if off-chain payment channel networks~\cite{poon2016bitcoin}. In
this setup, a single ledger transaction is used to escrow funds by two parties
$A$ and $B$. $A$ and $B$ can then proceed to make bidirectional micropayments to
each other \emph{without ledger interaction} through the exchange of signed IOU
tokens. By themselves, channels are useful for reducing ledger transactions
between parties with reoccurring interactions. More importantly, however,
micropayment channels can be extended such that $A$ pays $B$ through some
\emph{intermediary} party $I$ to which they both have active micropayment
channels. Pragmatically, $A$ and $B$ might occupy the roles of a customer and
merchant who are registered with a well-known financial service $I$, enabling
remarkable scaling. Informally multi-party channels are secure if the following
properties can be guaranteed:

\begin{enumerate}
\item At every step of the protocol, all parties possess proof of execution of
  the last finalized payment state
\item Given two proofs of payment states, the network can unambigiously identify
  the more recent state.
\item When $A$ agrees to pay $B$ through $I$, the payment is atomic. That is,
  there is never a situation in which $I$ pays $B$ but is unable to extract the
  agreed-upon payment from $A$.
\end{enumerate}

If all of these properties can be cryptographically ensured, then it becomes a
matter of network policy to ensure game-theoretic security of payments. The
Bitcoin Lightning Network and other similar protocols utilize simple hash
commitments and transaction delays to construct a secure scheme.

Finally, the micropayment channel concept has been extended to support anonymity
design goals. Z-Channel specifies an implementation designed specifically for
Zerocash which only supports two-party channels~\cite{zhang2017z}. Green and
Miers designed Bolt, which does support multi-party bidirectional
channels~\cite{green2017bolt}. In this setup, the anonymity set is defined with
respect to the users connected to the intermediary. In other words, given a set
of end users $E_{all} = \{E_1, E_2, ... E_n\}$ who all have active channels with
$I$, $E_a$ should be able to send secure payment to $E_b$ where $I$ cannot
identify $E_a$ or $E_b$ from $E_{all}$ nor can $I$ determine the payment
value. Of course, $I$ should still be able to verify (using zero-knowledge
proofs) that the payment is valid and that its internal channel states have been
updated accordingly. Several nuances arise concerning end user privacy in the
micropayment channel setup phase and in the event that $I$ maliciously aborts,
but we do not consider it necessary to discuss caveats here.

%
%\section{Economic Design}
%\label{sec:economic}
%
We first summarize the economic layer of the incentivization scheme. At this
level, we assume the existence of an ideally secure and efficient payment layer
and proceed to outline the high-level policy design.
%The challenge here is to
%engineer the flow of money in a way that is both economically stable yet still
%adheres to the core mission of the Tor Project.

\subsection{Economic Model}
The moneTor scheme allows for relays to offer a \emph{premium} bandwidth product
to Tor users in exchange for monetary payments. Under this framework,
financially willing users send cash directly to each relay along their circuits
in exchange for higher internet bandwidth and faster download speeds relative to
unpaid users. The moneTor tokens themselves can be viewed as wrappers for some
external financial asset that is converted at an exchange service. These assets
may be any form of programmatic money which satisfies the standard properties of
\textit{scarcity}, \textit{fungibility}, \textit{divisibility},
\textit{durability}, and
\textit{transferability}~\cite[p.3]{crump2011phenomenon} A schematic of the cash
flow cycle is illustrated in Figure~\ref{fig:economic}.
\begin{figure}[h] \centering
  \includegraphics[trim={0.5cm, 0.5cm, 0.5cm, 0.5cm}, clip, scale=0.7]{images/economic_diagram.png}
  \caption[Cash Flow]{Cash Flow --- movement of moneTor tokens through the
    network. The value $\tau$ denotes the fraction of money that is collected
    for taxation purposes.}
  \label{fig:economic}
\end{figure}
We introduce a novel concept within the field in the form of a
taxation element. Intuitively, the shape of the network that will
emerge in a purely profit-seeking environment may not perfectly
correspond with the central goals of the Tor Project. The Tor Project
may for instance desire to compensate certain types of relays more
than others in order to improve the overall quality of the network. To
this end, we accommodate for a stream of taxed income that is
anonymously diverted into a shared pool of funds. These funds are
selectively redistributed to relays via a transparent policy.  In
essence, taxation provides a tunable control mechanism for The Tor
Project to shape the topology of the network towards some notion of
optimal diversity and performance.

A key economic question to address is the issue of price determination. While it
would be tempting to enlist some market-based mechanisms to set premium
bandwidth prices, any price differentiation between clients or relays inevitably
leaks more information. This leakage becomes more severe with higher granularity
payment options as adversaries begin to use price to link payment channels and
circuits. We therefore impose the constraint that all users should pay a single
uniform price for premium bandwidth at any time $t$. This price may be set
through a centralized calculation by the authorities or a more dynamical
consensus vote reached by the network.

\subsection{Incentivized Conformity} In a decentralized network, there is no
practical way to enforce standard behavior at each local node. We must therefore
consider whether all nodes are rationally incentivized to obey the stipulated
policies. For instance, we cannot guarantee that relays will actually confer
premium bandwidth to paying users. Even though the relay has no particular
reason to deviate, the client should periodically monitor her bandwidth and only
make payments when they appear to be making a difference. The relationship
between the client and relay can then be modeled as a game theoretic tit-for-tat
dynamic.

At the opposite end of the spectrum, relays might overly prioritize premium
circuits while rejecting all traffic from unpaid users. They might also attempt
to game the tax redistribution process to gain larger share of the proceeds. The
bandwidth measurement authorities must anticipate such modes of deviation from
the standard behavior to ensure that the risk for a relay to get blacklisted
from the network is greater than the incremental gains it might attain from
cheating. These attacks, while manageable, suggest that it would be prudent to
limit the complexity of our economic policies until we can better study
behavioral deviation dynamics in the live network.

%%% Local Variables:
%%% mode: latex
%%% TeX-master: "../main"
%%% End:


\section{Payment Design}
\label{sec:payment}
\subsection{Ledger}

In our payment design, we follow the Bitcoin paradigm in which all users maintain full and exclusive control of their monetary wealth through use of public key cryptography~\cite{nakamoto2008bitcoin}.
However, unlike Bitcoin, it is unnecessary for moneTor to rely on an inefficient decentralized consensus mechanism.
Since Tor already relies on a centralized set of authorities, we simply introduce a new ledger authority role to maintain the global payment state on a public tamper-evident database~\cite{crosby2009efficient}.
To maximize availability, this ledger may also be distributed across several authorities using, for example, the architecture proposed by RSCoin~\cite{danezis2015centrally}.
In principle, the ledger architecture is an orthogonal engineering problem.
We leave this challenge to the implementor and instead focus on the more critical design of off-ledger payment channels.

\subsection{Payment Protocols Overview}
\label{sec:payment_overview}

In this section, we specify formal protocols that comprise the moneTor payment infrastructure.
As first discussed Section~\ref{sec:background}, our chosen model is an implementation of the tripartite anonymous payment channel paradigm.
Compared to purely centralized schemes, payment channels are a critical method for scaling, allowing for a theoretically unbounded number of off-ledger transactions between any two parties.
However, in a na\"{i}ve two-party implementation, the total channel management complexity is on the order of $O(n \times m)$ where $n$ is the number of Tor clients and $m$ is the number of relays.
Our solution is to introduce the Intermediary Relay, a Tor node whose only role is to provide \emph{atomic} payment channel services between clients and relays.
By acting as trustless payment hubs maintaining persistent channels to many users, their services reduce the channel complexity to $0(n+m)$.
Notably, we can adjust the target number of intermediaries to balance the performance of the payment infrastructure and the size of the anonymity set for connected premium users.

\medskip \noindent\textbf{Bolt.} The basis of our scheme is an extension of Bolt's anonymous micropayment channel protocol, which is itself a privacy-focused adaption of the ``Lightning Network''~\cite{poon2016bitcoin}.
Due to its importance as a starting point for our work, we first provide a brief outline of the prerequisite micropayment channel procedures defined in Bolt.
All protocols are either two or three party interactions between a subset of the following roles: $C$ (client), $R$ (relay), $E$ (end-user: either a client or relay), $I$ (intermediary), and $L$ (ledger).
For our use case, the reader can assume that all communications are anonymously routed through Tor circuits~\cite{green2017bolt}.

\begin{itemize}
\item \textbf{KeyGen}: Generates a cryptographic keypair
\item \textbf{Init-E}: $E$ initializes half of a micropayment channel by
  escrowing funds on $L$
\item \textbf{Init-I}: $I$ initializes half of a micropayment channel by
  escrowing funds on $L$
\item \textbf{Establish}: $E$ and $I$ interact to establish a new micropayment
  channel from their respective halves
\item \textbf{Pay}: $C$ interacts with $I$ and $R$ to send a single micropayment to $R$
\item \textbf{Refund}: $E$ closes a channel on $L$ and makes a claim on
  the escrowed funds.
\item \textbf{Refute}: $I$ closes a channel on $L$ and makes a claim on
  the escrowed funds.
\item \textbf{Resolve}: $L$ determines the final balance of funds awarded to
  each party.
\end{itemize}

While anonymous micropayment channels present a tremendous advance for many applications, the relatively heavy cryptography (37-100 ms) and communication (7 messages) is a prohibitive expense, especially if we want to do it for every message transmission\footnote{In addition to concerns regarding global network overhead, it is also desirable to keep the barrier of entry low for smaller relay operators.}.
To overcome this barrier, we present a new payment layer design enabling far more efficient \emph{nanopayments} that will satisfy our constraints.

\medskip \noindent\textbf{moneTor.} The moneTor construction makes use of the existing anonymous micropayment structure to build \emph{locally transparent nanopayments}.
In this model, clients and relays extend single micropayment operations into \emph{nanopayment channels} through an intermediary (Figure~\ref{fig:parties}).
These highly lightweight channels allow the client to send $n$ unidirectional nanopayments to the relay through the established Tor circuit.
Each payment represents a fixed value $\delta$, established at the start of the channel.
By \emph{locally transparent}, we mean that each nanopayment in the same channel is trivially linked to each other.
However, the nanopayment channels themselves are unlinkable to other nanopayment channels and micropayment operations.
It is by design that this channel anonymity guarantee fits with Tor's existing circuit framework and security model, which similarly stipulates that messages within circuits are linkable internally but not externally to other circuits~\footnote{In contrast, as discussed in Section~\ref{sec:related_work}, some prior works implemented circuits which were needlessly unlinkable from the relay's viewpoint.}.
In essence, the overarching motivation of our work is to relax the costly anonymity guarantees provided by Bolt toward the design of a new set of protocols specifically adapted for Tor.
Finally, note that the establishments and closures of nanopayment channels do not require the clients or relay to interact with the ledger, resulting in a far more scalable design than any system which depends on resource-intensive centralized entities.

\begin{figure}[h] \centering
  \includegraphics[trim={0.5cm, 0.5cm, 0.5cm, 0.5cm}, clip,
    scale=0.6]{images/party_diagram.png}
  \caption[Payment Roles]{Payment Roles --- Dashed lines represent periodic transactions (rare), thin double lines indicate micropayment channels (used at the beginning and end of circuit lifetime), and thick double lines indicate a nanopayment channel (handling nanopayments during the lifetime of the circuit).
    The dashed outline around the intermediary represents a notion of payment anonymity for the end-users.
    Connections to the ledger and to the intermediary are protected by an internal Tor circuit.}
  \label{fig:parties}
\end{figure}

We now briefly describe our new set of protocols.
Any two parties $C$ and $R$ can construct a nanopayment channel once both have completed Bolt's $Establish$ with a common intermediary $I$.
We define the following set of protocols needed to manage nanopayments:

\begin{itemize}
\item \textbf{Nano-Setup:} $C$ and $I$ interact to prepare an incomplete half of a nanopayment channel on top of their existing micropayment channel.
\item \textbf{Nano-Establish:} $C$ sends her nanopayment channel information to $R$, who interacts with $I$ to complete the second half of the nanopayment channel on top of their own existing micropayment channel.
\item \textbf{Nano-Pay:} $C$ sends a single nanopayment to $R$.
This is repeatable for up to $n$ operations.
\item \textbf{Nano-Close-R:} $R$ closes his nanopayment channel with $I$.
\item \textbf{Nano-Close-C:} $C$ closes her nanopayment channel with $I$.
This step must happen after \emph{Nano-Close-R}.
\end{itemize}

We also specify the following modified channel conflict resolution procedures to ensure secure closure properties for the nanopayment scheme.
Note that a
malicious party can force a closure (i.e., abort), but will not be able to steal anything thanks to the dispute resolution on the ledger.
It is important to note that the following algorithms run only in case of misbehaviour.

\begin{itemize}
\item \textbf{Nano-Refund:} $E$ closes the channel on $L$.
\item \textbf{Nano-Refute:} $I$ closes the channel on $L$.
\item \textbf{Nano-Resolve:} $L$ makes a final determination on both outstanding micropayment
  and nanopayment balances.
\end{itemize}

The core of our nanopayment scheme is inspired by the classic \emph{Payword} two-party micropayment scheme in which payments are encoded by successively revealed preimages in a precomputed hash chain~\cite{rivest1996payword}.

The challenge in this construction is to securely integrate the hash chain concept into an existing three-party anonymous micropayment channel setup such that all parties maintain secure cryptographic ownership of their funds at all steps.
At the same time, we must ensure the scheme does not leak deanonymizing information outside of the nanopayment channel context, a nontrivial task that requires significant restructuring of the Bolt protocol fit our constraints.
Our final solution is a concrete scheme which incurs an overhead penalty of approximately two micropayment operations per nanopayment channel, one at the beginning and one at the end of the channel life cycle.

\subsection{Nanopayment Protocol Details}

\label{sec:nanopaymentdetails} In this section, we provide a summarized intuition for the basic steps in the payment protocol.
A more formal treatment of the steps are provided in Appendix~\ref{sec:algorithms} with algorithmic details.
Security considerations are detailed next in Section~\ref{subsec:paysecurity} and formally described in Appendix~\ref{sec:proof}.

\medskip \noindent\textbf{Nano-Setup} At the start of this protocol, $C$ has access to a micropayment wallet $w$ obtained from Bolt's \textbf{Establish} that enables her to operate her micropayment channel with the Intermediary $I$ as well as a refund token $rt$ that entitles her to claim her current funds on the ledger $L$ should $I$ misbehave or go offline.
To construct a nanopayment channel, $C$ first generates an array of values $hc$ of length $n$ where $hc_i = H(hc_{i+1})$ and $hc_n$ is a random number.
The root of the hash chain $hc_0$ is used to create a globally unique nanopayment token $nT$ that encodes the public parameters of the channel including the length $n$ and the per-payment value $\delta$.
$C$ sends $I$ a commitment to a fresh nanopayment channel parametrized by $nT$ along with a zero-knowledge proof of the following statements:

\begin{enumerate}
\item The nanopayment wallet $nw$ is well-formed from $w$.
\item $C$ has ownership of a micropayment channel containing at least $n \times \delta$ funds.
\end{enumerate}

$I$ verifies these messages and supplies $C$ with a new signed refund token $nrt$ that entitles $C$ to cash out the full balance of the micropayment channel using \textbf{Nano-Refund} if needed.
$C$, now protected against misbehavior by $I$, agrees to send a revocation token $\sigma_w$, which revokes her right to use $w$ to create other nanopayment channels from this micropayment wallet or to use $rt$ to cash out the micropayment channel before the \textbf{Nano-Close} is run.
$I$ is now protected against double spending by $C$ and can safely inform $C$ that the nanopayment channel has been set up successfully.

\medskip \noindent\textbf{Nano-Establish:} At this point, $C$ sends $R$ the same $nT$ token used to setup the channel with $I$.
$R$ uses the token to initiate her end of the nanopayment channel with $I$ by executing essentially the same procedure that $C$ used in \emph{Nano-Setup}.
The nanopayment channel is now fully established and ready to be used.
A key observation is that both ends of the channel ($C$-$I$ and $R$-$I$) are rooted at the same hash chain root $hc_0$.

\medskip \noindent\textbf{Nano-Pay:} To make the $i^{th}$ payment, $C$ simply sends the next hash preimage $hc_i$ to $R$.
Knowledge of this preimage $hc_i$ is sufficient for $R$ to prove possession of a nanopayment.
At any given time, $R$ can broadcast the tuple ($nrt$, $hc_i$) to $L$ to prove ownership of the correct balance of funds.
Notice that this action simultaneously reveals $hc_i$ to $I$, who can then claim an equivalent value of funds from $C$.
As a result, the scheme satisfies a correct-by-construction property of \emph{atomicity} whereby both legs of the protocol are finalized at the same time.

\medskip \noindent\textbf{Nano-Close:} After some number of payments $k < n$ has transpired and $C$ wants to close the Tor circuit, both $C$ and $R$ will generally prefer to close their nanopayment channels through $I$.
In this process, the $R$-$I$ leg must be closed before the $C$-$I$ leg.
This is due to the unidirectional nature of nanopayment channels.
Since payments are flowing from $C$ to $R$, $I$ must first determine its debt to $R$ in order to know how much it can claim from $C$.

$R$ first sends to $I$ a commitment to a new micropayment wallet $w'$ and a zero-knowledge proof of the following statements:

\begin{enumerate}
\item $w'$ is well-formed from $w$ ($w$ was either created by Bolt's establish phase or by a previous moneTor Nano-Close).
\item The balance of $w'$ is equal to the sum of the balance from the previous wallet $w$ and $\delta \times k$.
\end{enumerate}

Once verified, $I$ issues a refund token $rt'$ on the new funds.
$R$ agrees to invalidate the nanopayment channel by issuing a revocation token $\sigma_{nw}$ to $I$.
$I$ and $R$ proceed to create a blind signature on $w'$ thus validating the wallet for future use.

Once $I$ has closed his nanopayment channel leg with $R$, $I$ and $C$ are free to complete the exact same close protocol.
All parties now revert to the original state preceding \emph{Nano-Setup} save for a securely updated balance.

\medskip \noindent\textbf{Nano-Refund, Nano-Refute, Nano-Resolve:} Honest parties will not typically close active nanopayment channels on the ledger, opting instead to run Bolt micropayment closure procedures when they wish to cash out.
However, in the event of malicious behavior or premature abortion, \emph{Nano-Refund} and \emph{Nano-Refute} enable $E$ and $I$ to withdraw funds on the ledger with the latest payment information at any time.
After a set amount of time allowing for the counterparty to reciprocate, the ledger runs \emph{Nano-Resolve} to make a final publicly verifiable determination on the final balance.
Correct execution of these procedures allow all honest parties to retain their funds in some cases and obtain the full balance of the malicious party's escrowed funds in others.

\subsection{Payment Security and Anonymity} \label{subsec:paysecurity} Our security model must account for both privacy and payment security.
The privacy threat model derives from the local active adversary paradigm ubiquitously found in Tor research~\cite{dingledine2004tor}.
Like all cells in Tor, Nanopayment messages are locally linkable by relays participating in the circuit.
However, since each circuit is only ever associated with one anonymous nanopayment channel at any given time, relays and intermediaries cannot link two nanopayment channels with the same user.
Furthermore, Tor circuits protect all communication in the tripartite nanopayment protocol.
Hence, relationship between client to intermediary, client to ledger, relay to intermediary, and relay to ledger are themselves anonymized.
We provide formal definitions and proofs for the following theorem in Appendix~\ref{sec:proof}.

\begin{theorem}

  The nanopayment channel scheme offers anonymity (\ref{def:anon1}, \ref{def:anon2}) and secure balance (\ref{def:balance}) under the assumptions that the commitment scheme is secure, the zero-knowledge system is simulation extractable and zero-knowledge, and the hash function used to create the hashchain and verify the preimage during Nano-Pay is modeled as a random oracle.

\end{theorem}

This theorem fully covers the security and anonymity characteristics of the protocol leading up to the closing of a micropayment channel.
However, two points must be made with consideration to potential deanonymization after the conclusion of the protocol.
First, the price revealed to the ledger at micropayment close enables a subtle passive attack by $I$ which our security model does not capture.
By examining the final number of payments made on each channel in conjunction with the globally fixed nanopayment cost, $I$ may potentially link all of $C$'s nanopayment channels.\footnote{This attack is best illustrated with a trivial example.
Suppose that $I$ facilitates a number of nanopayment channels with the following number of payments, each of which is known to represent one unit of money: $[58, 839, 356, 881, 23, 89, 561]$.
Now $C$ closes her micropayment channel and terminates with exactly $58 + 356 = 414$ units of money.
Once the micropayment channel is closed, $I$ must necessary gain knowledge of the final balance of funds and can easily link the first and third nanopayment channels as belonging to the same $C$.}
To mitigate this vulnerability, we stipulate that $C$ must make at least one micropayment, which has a monetary value hidden from $I$, before closing a micropayment channel.
We expect that this micropayment should contain a random value not greater than the channel escrow maximum value as stated in the Tor consensus and may be made to another account owned by $C$.

Secondly, in the event of a dispute resolution on the ledger, the two parties on either end of the micropayment channel ($C$-$I$ or $I$-$R$), must disclose their on-ledger identities.
However, given an adequately privacy-preserving choice of an on-ledger transaction protocol such as Zerocash, even this result would be meaningless for $R$~\cite{sasson2014zerocash}.
Thanks to the built-in ledger privacy, $R$ can only link the ledger interaction to its circuit, but not the client's real-world identity.

Having accounted for all privacy considerations both during and after the protocol, we informally state the following anonymity guarantees relative to unmodified Tor:

\begin{enumerate}
\item Additional parties needed to operate the moneTor system (i.e., ledgers and intermediaries) cannot extract any more information about a given client than any middle relay.
\item Excluding side channels, circuits do not leak any more information than the single bit needed to differentiate premium and nonpremium users.
\end{enumerate}

Our threat model for payment security is similar to those found in prior works in blockchain micropayment channels~\cite{poon2016bitcoin}.
In such models, the user is protected from malicious intermediaries by the ability to prove misbehavior to a global ledger.
Our protocols also guarantee that the client would not risk more money than initially agreed-upon (Balance property, see Section~\ref{def:balance}).

\medskip \noindent\textbf{Side-channels on Micropayment events.}
The moneTor protocol prescribes unlinkable micropayment events by protocol design.
However, a na\"{\i}ve implementation of this scheme may be susceptible to side-channel vulnerabilities, most notably, timing attacks.
For instance, a channel-creation policy that immediately mints a new nanopayment channel after the closing of the previous channel would allow $I$ to trivially link all channels belong to $C$.
To mitigate this particular risk, we require a random delay $t_r$ between nanopayment channels.
Since our scheme specifies the availability of preemptive channels, such a delay should not perversely impact user experience.
User privacy with respect to side channels and other statistical techniques depends on the number of users concurrently building channels to the same intermediary.
If we require an anonymity set of size $N$, then the maximum number of Intermediaries allowed in the network is $\frac{\mathbb{E}[W]}{N}$ where $W$ is a discrete random variable holding the number of payment events between any $t_i$, the time at \textbf{Nano-Close} and $t_{i+1}$, the time of the next \textbf{Nano-Setup} or \textbf{Nano-Establish}.
Consequently, the number of active Intermediaries should be a parameter of the system that the Tor project determines based on the activity of premium users.
Furthermore, clients should maintain channels with an number of different intermediaries to further mitigate any other unexpected side-channel leaks.

\subsection{Economic considerations}
\label{sec:economic_considerations}

Given the delicate nature of this research area, a key motivation in the technical design of our scheme is to explicitly accommodate some of the more consequential economic considerations.
For instance, in case of deployment, the Tor project would have to decide how to assign value to the moneTor tokens, a choice which entails fundamental trade-offs between control, liability, and social perception.
We enumerate three broad categories and briefly comment on the technical implementations as follows:

\begin{enumerate}

\item The Tor Project reserves control of monetary supply for the purpose of enforcing a publicly declared policy, for instance, to peg the value of tokens against one or more fiat currencies it that it holds in reserve.
This option would require the Tor Project to reserve a privileged signing key for minting new moneTor tokens (e.g.
when a user makes fiat deposit).
\item The moneTor tokens are instantiated as a standard cryptocurrency whose value fluctuates as a function of market pressures and the chosen distribution schedule.
As an any cryptocurrency, a node that rejects the distribution mechanism (e.g.
mining in Bitcoin) would be violating the protocol.
\item The moneTor tokens act as a secure wrapper for an external cryptocurrency such as Bitcoin, Ethereum, or a future state-backed currency.
This option is made possible by ongoing work in the field of ledger interoperability protocols~\cite{back2014enabling,poon2017plasma}.

\end{enumerate}

In all three cases, note that the mechanism required implement the police takes place entirely on the ledger.
Consequently, the off-ledger moneTor payment layer, which is the main contribution of this work, is compatible with any of these three options. 

Whereas the Tor Project has several acceptable options for monetary policy, there should not be such freedom in the pricing mechanism for premium bandwidth since any price differentiation between circuits would inevitably leak information.
This leakage becomes more severe with higher granularity payment options.
We therefore impose the constraint that all users should pay a single uniform price for premium bandwidth at any time $t$.

It is of course central to make sure that moneTor serves the goal of advancing of human rights and freedom of the Tor Project, which may be quite different from those that would result from a purely profit-seeking environment.
To this end, moneTor includes an explicit taxation mechanism, which we describe technically in Section~\ref{sec:tax_integration}.
Under this scheme, any time a payment takes place within the moneTor layer, a tunable fraction of the revenue is diverted into a special account controlled by the Tor Project.
This is analogous to a secure and automatic sales tax on premium traffic payments.
The Tor Project can use this fund to shape the topology of the network towards some notion of desirable diversity and performance via a transparent policy.
The exact content of such policy is an active subject of research that is orthogonal to our paper.\footnote{E.g., Waterfilling~\cite{waterfilling-pets2017} argues for security by maximum diversity in endpoints of user paths, and TAPS~\cite{taps-ndss2017} argues for security by trust policies.}
The tax rate also dictates incentives for nodes seeking to game the system, perhaps by inserting dummy traffic to collect tokens or faking bandwidth measurements.
We further elaborate on such questions and trade-offs in Section~\ref{sec:discussion}.

\subsection{Tax Integration}
\label{sec:tax_integration}

The zero-knowledge setup allows an elegant way to anonymously handle the tax collection policy.
Thus far, we have treated the nanopayment value $\delta$ as symmetric for both the client and relay leg.
In practice, it requires only a trivial modification to specify separate values of $\delta_C$ and $\delta_R$ such that the following equality is satisfied.

\begin{equation}
  \delta_C - \delta_R = \mathit{tax} + \mathit{fee}
  \label{eq:payment}
\end{equation}

\medskip \noindent Here, $\mathit{tax}$ is the portion of every payment that is redirected to the Tor tax authority while $\mathit{fee}$ represents compensation for $I$'s services.
$I$ gradually accumulates these overhead charges in his balance over the course of running many nanopayment channels.
When it is time for $I$ to cash out the full micropayment channel, $L$ simply divides the funds between the $I$ and the tax authority.
Note that this does \emph{not} mean that $L$ can arbitrarily control money as this process is well-defined in setup of the network protocol.

\subsection{Integration in Tor circuits}

Up to this point, we have described payments that occur between a single client and a single relay.
In practice, it is typical for each client to maintain a handful of concurrently active circuits,\footnote{For instance, the popular Tor Browser user application typically does not share circuits with streams targeting a different destination address unless those streams come from the same SOCKS connection.}
each of which requires three streams of payments to the guard, middle, and exit relays.
These channels must be actively managed to optimize computational overhead as well as money flow.
Furthermore, connections between the client and the guard relay are transparent and persistent across the timescale of several months.
We optimize toward this setup by enabling transparent and direct payment channels between the client and guard, considerably reducing the time to establish or close the channel.
In contrast, middle or exit relays channels require the flexibility of our full tripartite scheme. 

%%% Local Variables:
%%% mode: latex
%%% TeX-master: "../popets_monetor"
%%% End:


\section{Network Design}
\label{sec:network}

%\subsection{Extending the Tor protocol}
%
%We extend the Tor routing protocol described in Tor's
%specifications~\cite{dingledine2018tor} and exploit Tor's leaky-pipe circuit
%topology\footnote{``Leaky pipe'' refers to the ability of the user to direct
%  traffic that ends at an intermediate hop along the circuit} to exchange
%payment information with each hop of the circuit. We introduce two new control
%cell types: one link-level cell and one relay-level cell. The link-level cell is
%used to exchange information related to the payment protocol between the Tor
%client and its guard relay while the relay-level cell is used for the middle
%relay and the exit relay. This subtype of relay cell is comprised of a payment
%header denoting the type of payment cell, followed by a payload of payment
%data. To an outside observer, payment cells are indistinguishable to normal
%relay cells. Figure~\ref{fig:relay_command_mt_structure} shows the internal structure of the cell.
%\begin{figure}[h]
%    \centering
%    \includegraphics[scale=0.38]{images/payment_cell_header.png}
%    \caption{Relay Payment Cell --- Cell definition specifying the structure of
%      a moneTor payment cell. Note: a block that appears blank and empty is in fact the continuity of the previous row}
%\label{fig:relay_command_mt_structure}
%\end{figure}
%
%All the bytes starting from StreamID (included) are onion encrypted. RelCMD is set to RELAY\_COMMAND\_MT, PCMD is the payment command which is different for each step of the payment protocol. If some message overflows the payload available length (495 bytes), we queue multiple cells of the same PCMD and buffer them on the receiver side to unpack the whole message.

\subsection{Pre-built Channels}
By default, Tor attempts to pre-build circuits in order to reduce latency once a
user wishes to create a data stream. Much like circuits, moneTor payment
channels are high in initial latency because of the many in-out messages in the protocol. We therefore exploit the same strategy currently used in circuit establishment by
allowing payment channels to be preemptively set up and established on clean
pre-built circuits. This dramatically reduces the effective time to first
payment. Unfortunately, excessive establishment of preemptive channels will
eventually afflict the network with unused overhead. Our implementation features
a rudimentary prediction strategy that attempts to balance this trade-off by
anticipating the number of needed channels using historical usage
information in a similar way Tor anticipates the need for a fresh circuit.

Moreover, potential linkability of micro-wallets due to timings 
depends on the number of users concurrently building channels to the 
same intermediary when a micro-wallet is renewed at time $t$ and 
used at the time of the next unlikable Nano-Setup. By design, 
the anonymity set of micro-wallets is split within the number of 
intermediaries. The intermediary flag is validated by the Tor 
authorities in our implementation, and rules to ensure a sufficient 
large anonymity set is a deployment problem far from intractable, 
and out-of-scope of this paper. Note that the intermediary would 
have to link all micro-wallets of a same Tor user to learn the full 
balance of that user, and this is an exponentially difficult problem 
with the size of the anonymity set. 

%However, our approach may not be sufficiently optimized and further
%work on this front is warranted.
%
%\subsection{Network Scalability}
%\label{subsub:scalability}
%%\td{TODO: describe scalability of intermediary system and any networking
%%  bottlenecks that might arise such as port limits, etc.}
%
%In our design, we are concerned with memory consumption, kernel socket
%consumption and CPU consumption. Our choice for a tripartite protocol
%effectively shifts the memory consumption of opened and idle micro
%channels to the intermediary nodes of the network.
% 
%% A more basic setup
%%whereby each Tor client maintains a micropayment channel with each
%%relay would incur an $O(n*m)$ cost with
%%respect to channel management complexity, with $n$ the number of Tor clients and $m$ the number of relays. By engineering an additional
%%intermediary layer, the complexity of moneTor channel connections is
%%reduced to $O(n+m)$. In our implementation, intermediary relays do not
%%participate in routing user streams and are tasked only with providing
%%payment channel services. Intermediaries devote the full extent of
%%their computational resources toward this task, allowing only a few
%%strong nodes to handle all of channel management needs of the network.
%
%Interactions between parties are realized within Tor circuits to allow
%multiplexing of circuits over the same TCP connection. This also protects the
%identity of the client and its chosen circuit against identification by the
%ledger or the intermediary.
%%\footnote{We assumed no side-channel exploitation in
%%  this work but do discuss timing attacks in
%%  Section~\ref{sec:limitations_futurework}.}. 
% As a result, the intermediary and
%the ledger must have a number of available sockets higher than the number of
%relays in the network in the worst case. Since this limit is in line with Tor's
%current assumption (i.e., any relay has available sockets to connect to every other relay), our design inherits the same socket consumption scalability
%of the greater Tor network.

\subsection{Prioritized Traffic}
\label{subsub:prioritized}

%The final component of our network-oriented design addresses the need to deliver
%prioritized bandwidth given an explicit signalling of premium or nonpremium
%traffic. Our objective is to provide a tunable range of prioritization while
%incurring as little cost as possible to average global performance. In our
%design, the chosen value is enforced network-wide by the directory authorities
%in order to avoid partitioning of the anonymity set. 
Traffic scheduling is
perhaps the most intuitive mechanism with which to implement
prioritization. However, we found that local scheduling decisions on each relay for priority do not work well anymore with the current state of the Tor network, which precludes previous works scheduling approaches based on DiffServ~\cite{dovrolis1999case} and EWMA~\cite{tang2010improved} to offer the expected priority. A more detailed
discussion of our findings can be found in Appendix~\ref{sec:scheduling}, and raises interest for future work to comprehensively understand when priority mechanisms, including the one we suggest here, fail to provide the expected priority. Yet, the intuition to understand why previous priority mechanism fail today lays in the evolution of the Tor network capacity since a few years: the capacity for guard bandwidth and middle bandwidth has raised to a magnitude where we now observe most of the congestion happening between the exit relay and the destination address. Apart from performance improvement thanks to the added bandwidth, this evolution affects Tor's performance because Tor's internal control-flow window sizes tend to be more accurate with less congestion inside the Tor network~\cite{archive-2009-mail,kiraly2008solving}. As a downside, it affects schedulers' impact since most of the relays in our experiments were able to flush all queues at once at each write event, which makes inefficient any form of queue priority locally at a particular relay.

However, Tor's overlay flow control mechanism provides an alternative route than local scheduling policies
to implement our desired priority functionality. Indeed, since local decisions inside the scheduler at a particular relay may fail to achieve priority, designing priority as a global function of the circuit may help. Recall that edge nodes regulate the
traffic flux in either direction using a set of flow control windows. Roughly
speaking, these windows determine space allotted to each circuit on a relay's
scheduling queue, which in turn is positively correlated with effective
bandwidth. We implement our prioritization scheme by statically readjusting the window maximum sizes once
according to the following formula (both Circ window and Stream window).
\begin{equation}
  window' = window(1+ \alpha(premium / pr\% - 1))
  \label{eq:flow}
\end{equation}
Here, a circuit is marked as prioritized by the bit
$premium \in \{0, 1\}$. The tunable priority benefit
$\alpha \in [0, 1]$ defines the proportion of the non-premium capacity that we wish
to transfer to premium clients. By accounting for
$pr\% \in [0,1]$, the fraction of premium to nonpremium clients, we
can keep the total flow capacity constant. Keeping the total flow capacity constant means that the memory consumption at relays induced by processing cells should stay constant as well.
%Our policy may be
%implemented in one of two ways. First, each node could track the
%$premium\%$ locally and dynamically adjust their own windows. This
%introduces a considerable amount of added complexity with unclear
%consequences on network performance. A more sound approach calls for
%the Tor authorities to track the global value for $premium\%$ and
%periodically broadcast static flow control windows to be used by the
%entire network. We adopt the latter approach in this iteration of
%moneTor.

\paragraph*{Interlude.} This concludes the design of the moneTor framework. In
the proceeding sections, we describe steps taken to iteratively select and
validate key parameters as well as the scheme as a whole. Such parameters
include: payment frequency, preemptive channel creation, and prioritization
amounts ($\alpha$). Throughout this process, the underlying objective is to
prove that we can confer qualitatively ``significant'' advantage to paid premium
users while incurring minimal overhead costs with respect to throughput, memory
usage, and latency within a realistic network environment.

%\footnote{Our
%  research in traffic prioritization is meant to demonstrate at least some crude
%  capacity for premium advantage in our models and to suggest potential avenues
%  for further study. A more definitive design for production-ready policies is
%  left for future networking-oriented research.}

%%% Local Variables:
%%% mode: latex
%%% TeX-master: "../main"
%%% End:


\section{Experimental Validation}
\label{sec:experimentations}
Understanding typical Tor usage and assessing if it can benefit from our priority scheme is a crucial requirement.
Appendix~\ref{sec:analysis} covers a real Tor measurement study illustrating the importance of token exchange within the first few seconds of the data stream.
Having established the empirical context for a channel payment scheme, we validated our technical design via experiments performed on a prototype software implementation within the native Tor codebase.
This section illustrates that, due to the pre-built payment channel setup and low payment verification cost, our scheme potentially supports the majority of observed short-lived and bursty Tor circuits in a near-fair-exchange setting.

\subsection{Prototype}

\begin{figure*}[t] \centering
  \begin{subfigure}[t]{0.32\textwidth} \centering
    \includegraphics[clip, width=1.0\textwidth]{images/overhead_downloadtime.pdf} \caption{Download Time Overhead - Web + Bulk}
    \label{fig:overhead_ttlastbyte}
  \end{subfigure}
  \begin{subfigure}[t]{0.32\textwidth} \centering
    \includegraphics[clip, width=1.0\textwidth]{images/overhead_throughput.pdf}
    \caption{Throughput - Compared to baseline}
    \label{fig:overhead_throughput}
  \end{subfigure}
  \begin{subfigure}[t]{0.32\textwidth} \centering
    \includegraphics[clip, width=1.0\textwidth]{images/overhead_memory.pdf}
    \caption{Simulation Memory}
    \label{fig:overhead_shadow}
  \end{subfigure}
  \caption{Global Overhead --- Comparison of overhead in pure multicore and singlecore network.
    Figure~\ref{fig:overhead_ttlastbyte} shows two sets of time CDF curves for each file size (2 MiB and 5 MiB), Figure~\ref{fig:overhead_throughput} shoes the 5 minute moving average the simulation 10 consensus file `2018-02-03-00-00-00-consensus'.}
  \label{fig:overhead}
\end{figure*}

The implementation of moneTor embeds a substantial portion of our research contributions.
The modifications, applied to Tor release version 0.3.2.10, cover approximately fifteen thousand added lines of code across Tor's core C software.
We emphasize that the implementation is engineered solely for our experiments.
Most notably, we simulated expensive cryptographic operations such as ZKPs and commitments using methods that account for Shadow's unique virtual time management~\cite{jansen2011shadow}.
We consider both scenarios in which the simulated Tor process is running on a multicore or a singlecore processor.
In the multicore case, the cryptographic operations are replaced by an ``idle'' command that allows the virtual node to complete other tasks in parallel.
In the singlecore case, cryptographic operations are simulated by looping through a series of dummy SHA256 Hash operations.
Using these methods, we conservatively tuned the delays to reflect real measurements from prior background work~\cite{green2017bolt}.\footnote{Extracted values are conservative in the sense that our zero-knowledge proofs require proving only a subset of the statements required in each corresponding Bolt zero-knowledge proof.}
Note that our prototype does not implement anything that does not help us to answer our research goals, such as coin/wallet management, an extension of the Tor control protocol to manage the asset, and various options, Intermediary information recovery in case of a crash, etc.
The prototype serves the following purposes in our study:

\begin{enumerate}
\item Our implementation necessarily handles nuances missing from our protocol designs; we show that there are no unexpected and prohibitive practical conflicts with the existing Tor design.
\item Our platform allows us to study the feasibility of premium circuit prioritization from a networking perspective.
\item Our platform allows us to obtain a rough factor-of-two approximation for all bandwidth, computation, and memory requirements of a real deployment, both globally and at individual nodes.
\end{enumerate}

The first design purpose is qualitative, and we briefly note that we did not discover any insurmountable logical flaws in the design.
To analyze the networking dynamics and resource consumption, we studied our implementations through the following experiments.

\subsection{Methodology}
\label{subsec:methodology}

We used the Tor shadow simulator tool~\cite{jansen2011shadow, tracey2018high} to conduct experiments at two different network scales obtained from a consensus document published in early February 2018.
The first scale features 100 relays, 1000 clients, 10 intermediaries, and ran for a total of 90 minutes, which is sufficient to gather information concerning the system overhead and protocol execution times.
The second scale features 250 relays, 2500 clients, 25 intermediaries, 80 minutes of total run time, and more accurately models the networking performance benefits conferred to premium clients.
In both cases, simulated traffic included 8\% \emph{bulk} clients who continuously download 5 MiB files and 92\% \emph{web} clients who periodically download 2 MiB files.\footnote{While 5 MiB bulk files are a common standard in Tor benchmarking~\cite{portal2018tormetrics}, 2 MiB web files reflect the approximate size of modern web pages~\cite{team2018httparchive}.}
The number and behavior of clients were chosen to satisfy (A) realistic congestion rates measured by a transfer timeout percentage around 4\%~\cite{portal2018tormetrics} and a historical bulk/web global traffic ratio of about 25\%/75\%~\cite{privcount-ccs2016, learning-ccs2018}.
We do not intend for the scale of our experiments the exact configuration of client nodes to precisely replicate real-world conditions.
Tor networking is itself a complex area of research, and we are content to adopt the simplest model that broadly highlights the networking dynamics of our incentivization scheme.

\subsection{Experiments}

\label{subsec:experiments} Our experiments fall into one of three groups each capturing a separate characteristic of the scheme.

\medskip \noindent \textbf{Global Overhead.}
First, we attempt to show the total cost of the moneTor scheme in terms of total network throughput.
To highlight worst-case performance, we configured a medium-scale experiment consisting of 100\% of premium clients, which we compared to a baseline trial with 0\% premium clients.
The purpose of this experiment to understand the worst-case overhead imposed by the payment scheme \emph{without} applying any network prioritization in either control-flow or EWMA.
Since our protocol can benefit from concurrently executed cryptographic operations, a key parameter to the simulation is the number of CPU cores available on each relay.
Unfortunately, this information is not publicly available.
As a result, we conducted two trials: one in which all nodes are running on multicore hardware and one in which all nodes are running on singlecore hardware.
Figure~\ref{fig:overhead} summarizes the results of both trials.

\begin{figure*}[t] \centering
  \begin{subfigure}[t]{0.32\textwidth} \centering
    \includegraphics[trim={0 0cm 0 0cm}, clip, width=1.0\textwidth]{images/payment_establish.pdf}
    \caption{Nano-Establish - Built after the circuit construction, but before the circuit usage!}
    \label{fig:payments_establish}
  \end{subfigure}
  \begin{subfigure}[t]{0.32\textwidth} \centering
    \includegraphics[trim={0 0cm 0 0cm}, clip, width=1.0\textwidth]{images/payment_pay.pdf}
    \caption{First Payment - Should match 1 half RTT if the preemptive Nano-Establish is perfect}
    \label{fig:ttfp}
  \end{subfigure}
  \begin{subfigure}[t]{0.32\textwidth} \centering
    \includegraphics[trim={0 0cm 0 0cm}, clip, width=1.0\textwidth]{images/payment_close.pdf}
    \caption{Nano-Close - Happens just before the circuit is destroyed}
    \label{fig:payments_close}
  \end{subfigure}
  \caption{Protocol Execution Time --- Time to finish each protocol step split across interactions with each of the three relays.
    The simulation includes 100 relays, 2 authorities, 1 ledger authority, 10 intermediaries and 1000 Tor clients scaled down from the public consensus file `2018-02-03-00-00-00-consensus'.}
  \label{fig:latencymeasurements}
\end{figure*}

Our findings indicate that even in the worst-case scenario, our system incurs statistically negligible overhead at these scales across the two measures of download time (e.g., less than 2\% increase on the mean web download for the singlecore experiment), throughput, and memory usage.
When examining the raw network messages, we found corroborating evidence that moneTor only contributes to a small fraction, less than $1\%$, of the total network traffic in our experiment, a result which holds across all of our trials.
By default, we configured a payment rate of one payment cell for every 1000 data cells exchanged in either direction.
If the network requires more fairness, it is also possible to increase the payment rate with negligible CPU cost as long as the network overhead introduced by the control cells remains under an acceptable fraction of the overall bandwidth.

\medskip \noindent \textbf{Payment Latency.}
Given results from our data collection, we surmise that payment latency is a crucial factor in servicing our front-loaded clients.
To this end, we measure the distribution of completion times for various steps in the protocol.
To highlight the effects of inherent latency in the Tor network, we show payments split across each relay role of guard, middle, and exit.
Recall that moneTor makes use of high-overhead, low-marginal cost payment channels (i.e., the channels take time to build, but the client needs them long after they are built).
In other words, the bulk of the cost in our scheme lies in the execution of the nanopayment channel \emph{establish} and \emph{close} protocols as shown in Figure~\ref{fig:payments_establish} and Figure~\ref{fig:payments_close}.
Notice that close operations take roughly twice as long to complete as the establish operations due to the need for the relay to close his half of the nanopayment channel before the client can complete hers.
Figure~\ref{fig:ttfp} illustrates the time-to-first-payment, our most revealing latency metric.
This measure includes the overhead in channel establishment when we do not have available preemptive channels.
In the best-case scenario, when all three payment channels have been correctly pre-built for the circuit, this measure is equivalent to a single trip toward each relay.
Comparing this Figure~\ref{fig:ttfp} to Figure~\ref{fig:payments_establish}, we observe the effectiveness of preemptive channel building.
The other observation sustaining the effectiveness of the pre-built strategy is the recorded time for the ``call'' versus ``send'' lines; if there is no discrepancy between them, it means that the pre-built was a success leading to fully established pre-built channels.

In all protocol phases, we observe that latencies for guard relays are negligible in comparison to the middle and exit relays, which is a result of our design decision to implement directly-paid guard channels.
Again, this is a Tor-specific optimization made possible by the fact that guards maintain a semi-persistent, transparent relationship with only a small subset of clients.

\medskip \noindent \textbf{Network Priority.}
\label{sec:priority_exp} Our final set of experiments studies the success of our scheme in delivering prioritized traffic for premium users.
To perform this analysis, we prepared sets of three small experiments with varying modifier priorities $\alpha \in \{0, 0.25, 0.5\}$ and $\beta \in \{1, 5, 10\}$, where $\alpha = 0, \beta = 1$ represents vanilla Tor when payments are off.
In these experiments, we assume 25\% of premium users.
From Figures \ref{fig:modifier_pr25_web}, \ref{fig:modifier_pr25_bulk}, and \ref{fig:modifier_pr25_all}, we observe that, first, variations in our network-wide tunable parameters do offer differentiation in download speed.
Yet, as we detail in Appendix~\ref{sec:scheduling}, offering bandwidth differentiation for the Tor network is more complex than previously assumed.
Indeed, local scheduling priority, which was historically effective for past Tor topologies, appears to be ineffective under current conditions where congestion concentrates at the exit interface.
Second, the differentiation in bandwidth for $\alpha = .25, \beta=5$ ``averages out'' to approximately mirror the baseline experiment, indicating little loss in overall network performance, and confirming our overhead experiment (recall 25\% of premium users).
Nevertheless, our result for $\alpha = 0.5, \beta=10$ indicates that the performance degrades faster for nonpremium users when we become too aggressive in procuring gains for premium users.
The data also highlights the complexity in selecting a set of parameters and techniques to offer efficient prioritization without degrading the overall throughput of the network.
The overarching takeaway is that the network prioritization mechanism appears to be an even more complex challenge than the design of the anonymous payment layer itself.

Note that our analysis of the scheme holds the total network capacity static.
However, the motivation for any Tor incentivization scheme is to attract new relays to grow the network, which would, in principle, improve anonymity and censorship resistance for all users.
The effect on performance is unclear.
Although attracting new relays would increase the throughput, a faster Tor would likely appear to more users as a well.
In the absence of a reliable economic model, it is unclear how incentives would affect the experience of the average user, and so we opted to forgo the modeling of added capacity.

\begin{figure*} \centering
  \begin{subfigure}[t]{0.32\textwidth} \centering
    \includegraphics[trim={0 0cm 0 0cm},
      clip,width=1.0\textwidth]{images/modifier_pr25_web_lowloss.pdf}
    \caption{Web Download Time}
    \label{fig:modifier_pr25_web}
  \end{subfigure}
  \begin{subfigure}[t]{0.32\textwidth} \centering
    \includegraphics[trim={0 0cm 0 0cm}, clip,
      width=1.0\textwidth]{images/modifier_pr25_bulk_lowloss.pdf}
    \caption{Bulk Download Time}
    \label{fig:modifier_pr25_bulk}
  \end{subfigure}
  \begin{subfigure}[t]{0.32\textwidth} \centering
    \includegraphics[trim={0 0cm 0 0cm}, clip,
      width=1.0\textwidth]{images/modifier_pr25_all_lowloss.pdf}
    \caption{Total throughput, all clients}
    \label{fig:modifier_pr25_all}
  \end{subfigure}
  \caption{Prioritization Benefit --- Performance differentiation between paid and unpaid users.
    We display results for 25\% premium users.
    Simulations feature 250 relays, 2 authorities, 1 ledger authority, 25 intermediaries and 2500 Tor clients scaled down from the public consensus file `2018-02-03-00-00-00-consensus'.}
  \label{fig:modifier}
\end{figure*}

%%% Local Variables:
%%% mode: latex
%%% TeX-master: "../popets_monetor"
%%% End:


\section{Related Work}
\label{sec:related_work}
\subsection{Tor Incentive Schemes}

\subsection{MoneTor Preliminaries}

%\section{Limitations and Future Work}
%\label{sec:limitations_futurework}
%In this discussion, we frame the social nuances of incentivization that accompany moneTor.
Today, the Tor network owes its success to voluntary contributors.
These relay operators incur hardware, bandwidth, and labor expenses to participate in the Tor network often for intrinsic reasons, among them: political, philosophical, and philanthropic.
In implementing a system like moneTor, the Tor Project must consider the consequences of financial incentives on these potentially fragile value systems.
Critics have called attention to the social consequences of incentives~\cite{jansenblogpost}.
For instance, the ``crowding out effect'', describes a psychological phenomenon whereby the introduction of extrinsic motivations displaces previously dependable intrinsic motivations~\cite{10.1257/jep.25.4.191}.
At the sociological level, empirical studies of prosocial behaviour~\cite{10.1257/aer.96.5.1652} have shown that explicit incentives can reduce participation.
This means that in our application, they risk degrading the average social quality of Tor nodes without necessarily growing the network.
None of this is to say that extrinsic incentives are always ineffective,~\cite{10.1257/aer.96.5.1652}, only that choosing the right solution is nontrivial.
We do not presume to offer an authoritative opinion on the best social incentive design.
Instead, we show that the versatility of our token-centric technical design is sufficient to support a wide range of potential strategies.

By design, we have left many variables to be decided by the Tor Project.
Chief among these are the premium bandwidth price, premium bandwidth advantage, tax rate, monetary policy, and redistribution policy.
Enforcement mechanisms include a combination of network monitoring, parameter broadcast, ledger constraints, and organizational policy.
Together, these options give the Tor Project the freedom to implement a large variety of incentivization strategies.
We present a few theoretical paradigms below.

\begin{enumerate}

\item \textbf{Benevolent Ruler}: The Tor Project is a nonprofit with altruistic organizational intentions.
  In this hypothetical approach, the authorities set the tax rate to 100\%, turning moneTor into a revenue stream for purely centralized efforts to improve the network.

\item \textbf{Crypto-Libertarian}: On the other end of the spectrum, a simple na\"{\i}ve strategy is to implement moneTor as unregulated market where relays sell bandwidth for money, perhaps trustlessly backed by a cryptocurrency like Bitcoin or Ethereum.

\item \textbf{In-Kind Rewards}: If financial rewards turn out to be prohibitive, moneTor can mimic several of the purely in-kind non-transferable and transferable schemes described in Section~\ref{sec:related_work} by limiting the number of on-ledger transactions to zero or one, respectively.
  For instance, at zero allowed transactions, relays would be unable to use their tokens for any purpose other than to buy premium traffic from other relays.

\item \textbf{Subsistence Relaying}: Here, although relays would receive real money, the expected value rewards would be limited to the break-even cost of running a relay.
In this paradigm, every relay operator would effectively act as an individual NGO to avoid the potential liabilities that come with financial profit.
In practice, moneTor could support this design by adjusting global parameters such that the income of any relay does not exceed the cost-to-bandwidth ratio in the cheapest geographic region.
The tax redistribution policy could cover any discrepancy from this income floor due to location or other factors.
Previously, studies have indicated that incentives can be useful when sparingly applied to ``concrete'' task~\cite{10.1257/jep.25.4.191, 10.1086/431263}.
Provided careful public communication of intent, the same results would be expected here.

\item \textbf{Decentralized Grant-making}: Finally, the Tor Project can implement moneTor as a form of \emph{altruistic money}, which payees can donate to an approved list of prosocial projects, reminiscent of participatory grant-making and participatory budgeting~\cite{dinh2020universal}.
Rather than cashing out tokens, the ledger would only allow relays to contribute to their choice of special programs that, for instance, increase the number of relays in region $X$, fund research on topic $Y$, or advocate for privacy-related issue $Z$.
The result is a fully fungible asset that is extrinsically worthless to profit-driven entities but intrinsically rewarding for parties invested in Tor's core mission.

\end{enumerate}

Much future work remains to implement some of these methods.
On the technical side, robust monitoring must be implemented to mitigate against inflated bandwidth for tax redistribution and junk traffic insertion for direct payments.
Economically, we require a better understanding of location-based incentives and the elasticity of supply and demand for various types of relay incentives and client premium bandwidth.

The purpose of these paradigms is to illustrate the versatility moneTor's technical infrastructure provides.
In practice, the Tor Project can implement nearly any combination of our listed approaches, and many others as well.
The mindset we wish to instill is this: an overly flexible design can always be constrained afterward.
Until there is a consensus for the optimal social approach to incentivization, the most useful technical base is one that can be adapted to many models of human behavior.

%%% Local Variables:
%%% mode: latex
%%% TeX-master: "../popets_monetor"
%%% End:


\section{Discussion}

% Discuss first the problem
% Discussing conservative deployment of moneTor with Tax collection
% Discuss impact on usability, size and distribution of Tor users if only available for pay 
% Discuss incentives to game the system
% Discuss Liability impact to run relays
% Discuss auditing, taxation and payment distribution?
% Careful about offering more money for exits
% Comparing Tor incentives with incentives for studenship
% Big difference between providing premium service -- and setting % the relay in the network in a good place

Tor is a distributed network based on volunteered contribution: 
physical machines are bought or rented by people all around the 
world and connected through the Tor routing protocol. Those 
people are taking time and spending part of their income 
contributing to Tor for various kind of reasons, among 
philosophical, political and philantropical intrinsic 
motivations. That is, relay operators do not expect to make money 
from their contributions. An important consideration to have is 
to maintain the strength of intrinsic motivations of the current 
community and to prevent monetary rewards to harm it. In this spirit,
several questions~\cite{jansenblogpost} were raised covering 
potential social issues from previous attempts to design incentives, 
and among them the ``crowding out'' effect~\cite{10.1257/jep.25.4.191}, showing how 
explicit incentives such as monetary rewards can lead to a decrease 
of intrinsic motivations. Social experimentation in the litterature 
covering prosocial behaviours~\cite{10.1257/aer.96.5.1652} has shown that rewarding the 
behaviour could backfire and decrease the amount of participants to the 
task instead of the expected increase, due to psychological effects on 
self-representation, on the image we want to send to others, or to 
mismatching expectation on what the reward should be. 

Rewards in the prosocial studies were usually linked to profit or to compensation. However, monetary 
incentives does not necessarly need to be about that. A 
conservative approach could be taken by not rewarding relays for relaying 
bandwidth, but by offering an
in-kind support with moneTor tokens -- that is, the expected value of the received tokens cannot exceed the cost of 
running relays, and is expected to support the estimated 
cost of the relay. In this reasoning, every relay operator could act as nonprofit and not 
suffer from potential liability of what running relays for profit could create. The received 
money would not compensate the 
cost of running the infrastructure, which is fueled by the operator's intrinsic 
motivations.  In practice, moneTor could support this design by ajusting the 
global price for direct payments such that the direct incomes of the 
cheapest place in the Internet to run relays does not exceed the cost of running 
a relay at this place. Then, the tax redistribution could be a function 
accounting for relay location and relay bandwidth that would try to reimburse 
the marginal cost of running the relays in the network. Many more questions are 
tied to tax redistribution, such as preventing operators to game the system 
(security), evaluating the right price for a given location (economic) or 
protecting the trust relationship~\cite{10.1257/aer.96.5.1611} between Tor authorities and 
relays (social). Could the tax redistribution be a community decision? 
Answering those questions would yield interesting further work, with the goal to incentive 
operators with intrinsic motivations to deploy 
more relays, and potentially at more diverse locations (usually more costly). It has been 
shown in incentives for Education~\cite{10.1257/jep.25.4.191, 10.1086/431263}, that when 
some support was offered for a concrete task (e.g., course enrollment), the social 
experiment was showing positive results. We expect that offering support for operators with 
intrinsic motivation to run relays should yield similar results than offering support to 
student with intrinsic motivation to complete studies.

Applying an in-kind support should not mix operators attracted by extrinsic 
values with operators holding intrinsic motivation, yet this remains to be 
verfied in concrete experimentation. However, it could be interesting to motivate the Intermediary 
role with extrinsic value. The Tor network would require in practice a 
limited amount of Intermediaries, but they must offer strong availability to 
maintain the micropayment channel states for all peers and enable 
robustness in the payment layer. Running Intermediary nodes requires the operator to have 
more CS abilities than operators running classical Tor relays. It is likely that observing, 
understanding and helping the Tor project with technical limitations of the implementation 
would be an important task for a limited amount of operators handling a new role within the 
network. However, an Intermediary does not require a trustworthy operator, but someone 
incentivized to provide good services.
Those intermediaries are by design not tasked 
to relay user streams, hence those operators are not to be mixed with the relay 
operator community.


\label{sec:discussion}

\section{Data Reproducibility}
\label{sec:code}
Our code developed during this research and needed to reproduce our graphical results can be found in various repositories at @Monetor Github account~\cite{monetor-github}.

\section{Conclusion}
\label{sec:conclusion}
The Tor network suffers from concerns in both performance and
diversity. In accordance with the general law that large networks are better than
small networks, we assert that the quality of Tor along both of these vectors
can be simultaneously improved by incentivizing more relay participation, while promoting some diversity notion. To this end, we
present moneTor, a comprehensive framework for incentivizing Tor relay
participation through true monetary payments. Our design broadly covers every
major level of implementation from general economic model to the protocol
payment layer and integration into the Tor networking architecture. Along the
way, we developed novel protocols for highly efficient and cryptographically
secure nanopayment channels as well as novel techniques in networking
integration.

A small venture into empirical data collection reinforced our intuition that
existing user behavior is compatible with a light-weight payment incentive
scheme. This led to our more involved round of experimentation in which we
tested our natively integrated moneTor prototype. The results of this
investigation were highly encouraging, indicating low latencies, negligible
throughput overhead, and upwards of 100\%-200\% benefits for paying users in the
simulated environment.

Legal, political, and sociological questions concerning the prudency of
introducing money into the Tor network are difficult to answer in a laboratory
setting. However, our work indicates that moneTor is feasible from a
technical and economical perspective.

We optimistically conclude that the built-in efficiency and scalability in particular makes our tripartite micro+nanopayment layers an attractive option for other usage than Tor bandwidth prioritization, such as metered video viewing, metered video gaming, and pay-per-page websites.
%moneTor an attractive option
%We optimistically conclude with an
%overarching statement that moneTor is the first true monetary incentive scheme for
%Tor which is ready for serious production-level development today.


%Acknowledgements:
% Edouard Cuvelier for comments
% Tor safety board

\bibliographystyle{IEEEtran}
\bibliography{IEEEabrv,sections/references.bib}

%\textbf{Sandia National Laboratories is a multimission laboratory managed and operated by National Technology \& Engineering Solutions of Sandia, LLC, a wholly owned subsidiary of Honeywell International Inc., for the U.S. Department of Energy’s National Nuclear Security Administration under contract DE-NA0003525.}
%
\appendix
\section{Empirical Analysis}
\label{sec:analysis}
\subsection{Methodology}
\subsection{Data Collection}
\label{subsec:datacollection}

We deployed a data collection system to look for more realistic information about lifetime and bandwidth consumption through time of Tor circuits. Our objective is to have a deeper understanding of typical Tor usage, and if such usage can benefit from our channel-based payment system. For example, those measurements could capture some notion about the type and magnitude of potential premium traffic. We define the type of traffic based on the port used to connect to the request service. Besides the classical ports 80 and 443 for web traffic, we aggregate data based on some other families, such as the WHOIS protocol~\cite{rfc3912} and RWHOIS~\cite{rfc2167} with port 43 and 4321. The complete list of families is constructed from the reduced exit policies~\cite{reducedexitpolicies} we run on our relays. It allows us to reason based on application specific traffic.
%We interested to know about the distribution lifetime of Tor circuits for each port we allow. We are also interested to picture how many cells those circuits handled through their lifetime with some level of granularity.

\subsubsection{Efforts to preserve users privacy}

\subsubsection{Observations}

\begin{figure*}
	\centering
	\begin{subfigure}[t]{0.32\textwidth}
		\centering
		\includegraphics[scale=0.28]{images/exitmeasurement.png}
		\label{fig:stats_a}
	\end{subfigure}
	\begin{subfigure}[t]{0.32\textwidth}
		\centering
		\includegraphics[scale=0.28]{images/totcellcountscdf.png}
		\label{fig:stats_b}
	\end{subfigure}
	\begin{subfigure}[t]{0.32\textwidth}
		\centering
		\includegraphics[scale=0.28]{images/stddevs.png}
		\label{fig:stats_c}
	\end{subfigure}
	\label{fig:measurements}
	\caption{Tor measurements}
\end{figure*}
\subsection{Ethical Considerations}


\section{Algorithms}
\label{sec:algorithms}
This appendix describes the algorithms we design to operate moneTor nanopayment
channels.

\subsection{Conventions}

We adopt the following conventions in our algorithms.

\begin{itemize}
\item All variable names in this section, except for possibly helper
  functions, are globally unique.
\item Variable subscripts denote a party or role ((I)ntermediary,
  (C)lient, (R)elay, (E)nd user).
\item New nanopayment variables are prefixed with the character
  (n). All other variables reference a value from the original Bolt
  scheme, although the name might be altered somewhat.
\item Payment values ($\epsilon, \delta$) are signed integers with
  respect to the end user. For example, $\delta_C$ is negative and
  $\delta_R$ is positive in the typical case where a client is paying
  a relay.
\end{itemize}

\subsection{Variable Index}

The follow itemizes variables used in the algorithm design. The first level of
variables are used for actual cryptographic and accounting operations. These are
bundled into groups of higher level variable names meant to represent
abstraction concepts such as payment channels and party states. Only these high
level variables are saved outside the context of the algorithms.

$nT = (\delta_C, \delta_R, n, hc^0)$ --- ``Nanopayment Channel Token'' ---
Stores static, public information that defines a nanopayment channel including
the payment values on both legs, the max number of payments, and the hashchain
head. This can be passed around freely by all parties.

$ncsk_C = (nwpk_C, nwsk_C, HC)$ --- ``Client Nanopayment Secrets'' --- Includes
a Public/private key pair which allows the client to setup and close a
nanopayment channel and a precomputed hash chain to make incremental
nanopayments

$nS_C = (k, hc^k)$ --- ``Client Nanopayment State'' --- Mutable state of the
nanopayment; includes the count of payments made so far and the latest sent hash
pre-image

$nrt_C$ --- ``Client Nanopayment Refund'' --- Allows the client to make a claim
to the ledger on escrowed money at any time. This refund is signed by the
intermediary and conditioned on revealing the latest hash pre-image that the
client claims to have sent.

$nrc_C$ --- ``Client Channel Closure Message'' --- Final message that is
posted to the ledger by the client to claim all funds of the
micropayment channel including any completed nanopayments.

$nS_I = \{nT: channel\_state\}$ --- ``Intermediary Nanopayment State'' Map of
all past and present nanopayment channels and the corresponding channel
state. Possible states are:

\begin{itemize}
\item $\bot$ --- failed attempt at setting up a nanopayment channel
\item $ready$ --- channel has been set up by $C$
\item $established$ --- channel has been established with $R$
\item $closed||hc^k$ --- channel has been closed and no further payments
  are allowed
\end{itemize}

$ncsk_R (nwpk_C, nwsk_C, \bot)$ --- ``Relay Nanopayment Secrets'' --- Includes a
public/private key pair allows the relay to setup and close a nanopayment
channel. Since relays cannot make payments in this setup, the last field is left
blank.

$nS_R = (k, hc^k)$ --- ``Relay Nanopayment State'' --- See $nS_C$

$nrt_R$ --- ``Relay Nanopayment Refund'' --- See $nrt_C$

$nrc_C$ --- ``Relay Channel Closure Message'' --- See $nrc_C$
message
\subsection{Algorithms}

Here we formally specify the algorithms.

\begin{algorithm}
  \begin{algorithmic}[1]
    \caption{Helper function for creating a new wallet}
    \Function{Wal}{$pp, pk_{payee}, w, \epsilon$}
    \State{parse $w$ as $(B, wpk, wsk, r, \sigma^w)$}
    \State{$(wsk', wpk') \gets $KeyGen$(pp)$}
    \State{$r' \gets $Random$()$}
    \State{$wCom' \gets $Commit$(wpk', B + \epsilon, r')$}
    \State{$\pi \gets PK\{(wpk', B, r', \sigma^w)$: \par}
    \State{\hskip\algorithmicindent{} $wCom' = $Commit$(wpk', B + \epsilon, r')\ \wedge$}
    \State{\hskip\algorithmicindent{} Verify$(pk_{payee}, (wpk, B), \sigma^w)\ \wedge$}
    \State{\hskip\algorithmicindent{} $B + \epsilon \geq 0$}
    \State{\Return{$(wsk', wpk', wCom', \pi)\}$}}
    \EndFunction{}

  \end{algorithmic}
\end{algorithm}

\begin{algorithm}
  \caption{Nanopayment Channel Setup --- Protocol between a relay and
    intermediary to create a new nanopayment channel from an existing
    micropayment wallet. Run prior to circuit setup.}
  \begin{algorithmic}[1]
    \Procedure{Client}{$pp, pk_I, w_C, \delta_C, n$}
      \State{parse $w_C$ as $(B_C, wpk_C, wsk_C, r_C, \sigma^w_C)$}
      \If{$B_{C} + (\delta_C * n) < 0$}
        \State{Abort$()$ \Comment{consider new micropayment channel}}
      \EndIf{}
      \State{$\epsilon_C \gets \delta_C * n$}
      \State{$(nwpk_C, nwsk_C, nwCom_C, n\pi_C) \gets$ Wal$(pp, pk_I, w_C, \epsilon_C)$}
      \State{$\delta_R \gets -(\delta_C  + tax)$} \Comment{the tax is a net profit for $I$}
      \State{$HC \gets $MakeHC$($Random$(), n)$}
      \State{$nT \gets (\delta_C, \delta_R, n, HC[0])$}
      \State{Intermediary.Send$(wpk_C, nwpk_C, nwCom_C, n\pi_C, nT)$}
    \EndProcedure{}

    \Procedure{Intermediary}{$pp, S_I, nS_I$}
      \State{$(wpk_C, nwpk_C, nwCom_C, n\pi_C, nT) \gets $Client.Receive$()$}
      \State{parse $nT$ as $(\delta_C, \delta_R, n, hc^0)$}
      \If{$wpk_C \in S_I \vee \neg $Verify$(n\pi_C)$}
        \State{Abort$()$ \Comment{invalid wallet}}
      \EndIf{}
      \If{$-\delta_C \ne price \vee \delta_R + \delta_C + tax \ne 0$}
        \State{Abort$()$ \Comment{incorrect payment prices}}
      \EndIf{}
      \State{$S_I \gets S_I \cup \{wpk_C : \bot, nwpk_C: \bot\}$}
      \State{$nS_I \gets nS_I \cup \{nT : \bot\}$}
      \State{Client.Send$(verified)$}
    \EndProcedure{}

    \Procedure{Client}{}
      \State{$ver \gets $Intermediary.Receive$()$}
      \State{$\epsilon^k_C = B_C + (\delta_C * k)$}
      \State{$nrt_C \gets $Intermediary.Blindsig$(ver, refund || nT || nwpk_C || \epsilon^k_E)$}
      \State{$nS_C \gets (0, HC[0])$}
      \State{$ncsk_C \gets (nwpk_C, nwsk_C, HC)$}
      \State{$\sigma^{rev(w)}_C \gets $Sign$(wsk_C, revoke||wpk_C)$}
      \State{Intermediary.Send$(\sigma^{recv(w)}_C$)}
    \EndProcedure{}

    \Procedure{Intermediary}{}
      \State{$\sigma^{recv(w)}_C \gets $Client.Receive$()$}
      \If{$\neg $Verify$(wpk, revoke||wpk_C, \sigma^{recv(w)}_C) = 1$}
        \State{Abort$()$ \Comment{invalid revocation token}}
      \EndIf{}
      \State{$S_I[wpk_C] \gets \sigma^{recv(w)}_C$}
      \State{$nS_I[nT] \gets ready$}
      \State{Client.Send$(established)$}
    \EndProcedure{}

  \end{algorithmic}
\end{algorithm}

\begin{algorithm}
  \caption{Nanopayment Channel Establish --- Protocol between a client,
    intermediary, and relay to establish the nanopayment channel between the
    client and relay. Run at the start of circuit setup.}
  \begin{algorithmic}[1]

    \Procedure{Client}{$nT$}
      \State{Relay.Send$(nT)$}
    \EndProcedure{}

    \Procedure{Relay}{$pp, pk_I, B_{I:B}, w_R$}
      \State{$nT \gets $Client.Receive$()$}
      \State{parse $w_R$ as $(B_R, wpk_R, wsk_R, r_R, \sigma^w_R)$}
      \State{parse $nT$ as $(\delta_C, \delta_R, n, hc^0)$}
      \If{$B_{I:B} - (\delta_B * n) < 0$}
        \State{Abort$()$ \Comment{consider new micropayment channel}}
      \EndIf{}
      \State{$\epsilon_R \gets \delta_R * n$}
      \State{$(nwpk_R, nwsk_R, nwCom_R, n\pi_R) \gets $Wal$(pp, pk_I, w_R, \epsilon_R)$}
      \State{Intermediary.Send$(wpk_R, nwpk_R, nwCom_R, n\pi_R, nT)$}
    \EndProcedure{}

    \Procedure{Intermediary}{$pp, S_I, nS_I$}
      \State{$(wpk_R, nwpk_R, nwCom_R, n\pi_R, nT) \gets $Relay.Receive$()$}
      \State{parse $nT$ as $(\delta_C, \delta_R, n, hc^0)$}
      \If{$wpk_R \in S_I \vee \neg $Verify$(n\pi_R)$}
        \State{Abort (invalid wallet)}
      \EndIf{}
      \If{$nS_I[nT] \ne ready$}
        \State{Abort (unregistered nanopayment channel)}
      \EndIf{}
      \State{$S_I \gets S_I \cup \{nwpk_R, \bot\}$}
      \State{$nS_I[nT] \gets established$}
      \State{Relay.Send$(verified)$}
    \EndProcedure{}

    \Procedure{Relay}{}
      \State{$ver \gets $Intermediary.Receive$()$}
      \State{$\epsilon^k_R = B_R + (\delta_R * k)$}
      \State{$nrt_R \gets $Intermediary.Blindsig$(ver, refund || nT || nwpk_R || \epsilon^k_R)$}
      \State{$ncsk_R \gets (nwpk_R, nwsk_R, \bot)$} \Comment{match client format}
      \State{$nS_R \gets (0, hc^0)$}
    \EndProcedure{}
  \end{algorithmic}
\end{algorithm}

\begin{algorithm}
  \caption{Nanopayment Channel Pay --- Protocol between the client and relay to
    forward a single nanopayment. Run periodically throughout the lifetime of
    the circuit.}
  \begin{algorithmic}[1]

    \Procedure{Client}{$nT, ncsk_C, nS_C$}
      \State{parse $nT$ as $(\delta_C, \delta_R, n, hc^0)$}
      \State{parse $ncsk_C$ as $(nwpk_C, nwsk_C, HC)$}
      \State{parse $nS_C$ as $(k, hc^k)$}

      \If{$k >= n$}
        \State{Abort$()$ \Comment{out of nanopayments, setup a new channel}}
      \EndIf{}

      \State{$nS_C \gets (k+1, HC[k+1])$}
      \State{Relay.Send$(HC[k+1])$}
    \EndProcedure{}

    \Procedure{Relay}{$nT, nS_R$}
      \State{$hc^{k+1} \gets $Client.Receive$()$}
      \State{parse $nS_R$ as $(k, hs^k)$}
      \If{$k+1 >= n \vee Hash(hc^{k+1}) \ne hc^k$}
        \State{Abort$()$ \Comment{invalid nanopayment}}
      \EndIf{}
      \State{$nS_R \gets (hs^{k+1}, k+1)$}
    \EndProcedure{}
  \end{algorithmic}
\end{algorithm}

\begin{algorithm}
  \caption{Nanopayment Channel Close --- Protocol between an end user (client or
    relay) and an intermediary to close out the nanopayment channel and receive
    a micropayment wallet. Run any time after the circuit closure. Also, the
    relay must close first}
  \begin{algorithmic}[1]
    \State{$\forall E \in \{Client, Relay\}$}
    \Procedure{EndUser}{$pp, pk_I, w_E, nT, ncsk_E, nS_E$}
      \State{parse $w_E$ as $(B_E, wpk_E, wsk_E, r, \sigma^w_E)$}
      \State{parse $nT$ as $(\delta_C, \delta_R, n, hc^0)$}
      \State{parse $ncsk_E$ as $(nwpk_E, nwsk_E, \_)$}
      \State{parse $nS_E$ as $(k, hc^k)$}
      \State{$\epsilon_E \gets \delta_C * k$ if (EndUser = Client) else $ \delta_R * k$}
      \State{$(wpk'_E, wsk'_E, wCom'_E, \pi'_E) \gets $Wal$(pp, pk_I, wpk_B, \sigma^w_E, B_E, \epsilon_E$)}
      \State{Intermediary.Send$(wpk_E, wCom'_E, \pi'_E, nT, \epsilon_E, k, hc^k)$}
    \EndProcedure{}

    \Procedure{Intermediary}{$pp, S_I, nS_I$}
      \State{$(wpk_E, wCom'_E, \pi_E, nT, \epsilon_E, k, hc^k) \gets $EndUser.Receive$()$}
      \State{parse $nT$ as $(\delta_C, \delta_R, n, hc^0)$}
      \If{$\epsilon_E < 0 \wedge closed \not\in nS_I[nT]$}
        \State{Abort$()$ \Comment{client attempting to close before relay}}
      \EndIf{}
      \If{Verify$(\pi_E) \vee nS_I[nT] \ne established$}
        \State{Abort$()$ \Comment{invalid wallet or channel}}
      \EndIf{}
      \If{$k > n \vee \neg $VerifyHC$(hc^0, k, hc^k)$}
        \State{Abort$()$ \Comment{invalid payment hash chain}}
      \EndIf{}
      \State{$nS_I[nT] \gets closed||hc^k$}
      \State{EndUser.Send$(verified)$}
    \EndProcedure{}

    \Procedure{EndUser}{}
      \State{$ver \gets $Intermediary.Receive$()$}
      \State{parse $ncsk_E$ as $(nwpk_E, nwsk_E, \bot)$}
      \State{$\epsilon^k_E = B_E + (\delta_E * k)$}
      \State{$rt'_E \gets $Intermediary.Blindsig$(ver, refund || wpk'_E || \epsilon^k_E)$}
      \State{$\sigma^{rev(nrt)}_E \gets $Sign$(nwsk_E, revoke||nwpk_E)$}
      \State{Intermediary.Send$(nwpk_E, \sigma^{rev(nrt)})$}
    \EndProcedure{}

    \Procedure{Intermediary}{}
      \State{$(nwpk_E, \sigma^{rev(nrt)}_E) \gets $EndUser.Receive$()$}
      \If{$nwpk_E n\in S_I \vee \neg $Verify$(nwpk_E, \sigma^{rev(nrt)})$}
        \State{Abort$()$ \Comment{unregistered channel or revocation token}}
      \EndIf{}
      \State{$S_I[nwpk_E] \gets \sigma^{rev(nrt)}$}
      \State{EndUser.Send$(verified)$}
    \EndProcedure{}

    \Procedure{EndUser}{}
      \State{$ver \gets $Intermediary.Receive$()$}
      \State{$w'_E \gets $Intermediary.Blindsig$(ver, wpk_E'||B_E + \epsilon_E)$}
    \EndProcedure{}

  \end{algorithmic}
\end{algorithm}

\begin{algorithm}
  \caption{Nanopayment Refund --- Algorithm by an end user to close a micropayment
    channel and claim ledger funds. This is a modified version of
    Bolt's Refund algorithm to also allows for granular claims on
    open nanopayment channels}
  \begin{algorithmic}[1]
    \State{$\forall E \in \{Client, Relay\}$}
    \Function{EndUser}{$pp, csk_E, w_E, nT, ncsk_E, nS_E, nrt_E$}
    \State{parse $csk_E$ as $(\_, sk_E, \_, \_, \_, \_)$}
    \State{parse $w_E$ as $(B_E, \_, \_, \_, \_)$}
    \State{parse $nT$ as $(\delta_C, \delta_B, \_, \_)$}
    \State{parse $ncsk_E$ as $(nwpk_E, \_, \_)$}
    \State{parse $nS_E$ as $(k, hc^k)$}
    \State{$\delta_E \gets \delta_C$ if (EndUser = Client) else $ \delta_R$}
    \State{$m_E \gets (refund || nT || nwpk_E || B_E + \delta_E * n, nrt_E, hc^k_E, k_E)$}
    \State{$nrc_E \gets (m_E, Sign(sk_E, m_E))$}
    \State{\Return{$nrc_E$}}
    \EndFunction{}
\end{algorithmic}
\end{algorithm}

\begin{algorithm}
  \caption{Nanopayment Refute --- Algorithm by an intermediary to respond to an end user's refund claim by posting its own channel closure message to the ledger}
  \begin{algorithmic}[1]
    \State{$\forall E \in \{Client, Relay\}$}
    \Function{Intermediary}{$pp, T_E, S_I, nS_I, nrc_E$}
    \State{parse $nrc_E$ as $(m_E, \sigma^m_E)$}
    \State{parse $m_E$ as $(refund || nT || nwpk_E || B^{full}_E, nrt_E, k_E, hc^k_E)$}
    \State{\Comment{$B^{full}_E \gets$ balance if nanopayment channel were saturated}}
    % however we get nrc_E... need to decide
    \State{parse $T_E$ as $(pk_E, \_)$}
    \If{$\neg$Verify$(pk_E, m_E, \sigma^m_E)$}
      \State{Abort$()$ \Comment{bad signature, well be rejeced by ledger}}
    \EndIf{}
    \If{$\neg$Verify$(pk_I, (refund || nT || nwpk_E || B^{full}_E), nrt_E)$}
      \State{Abort$()$ \Comment{unapproved refund token}}
    \EndIf{}
    \If{$S_I[nwpk_E] \ne \bot$}
      \State{\Comment{$E$ is posting an old token, $I$ should refute}}
      \State{$\sigma^{rev(nrt)}_E \gets S_I[nwpk_E]$}
      \State{$nrc_I \gets ((revoked, \sigma^{rev(nrt)}_E), Sign((revoked, \sigma^{rev(nrt)})))$}
    \EndIf{}
    \State{\Comment{Everything checks out; accept the closure}}
    \State{$hc^k \gets nS_I[nT]$}
    \State{$nrc_I \gets ((accepted, k_I, hc^k_I), Sign(accepted, k_I, hc^k_E))$}
    \State{\Return{$nrc_I$}}
    \EndFunction{}
  \end{algorithmic}
\end{algorithm}

\begin{algorithm}
  \caption{Nanopayment Resolve --- Algorithm run by the ledger (and everyone verifying the ledger) to resolve all channel closure messages and allocate the appropriate final balances}
  \begin{algorithmic}[1]
    \State{\Comment{Returns the tuple $(B^{final}_E, B^{final}_I)$}}
    \Function{Ledger}{$pp, T_E, T_I, nrc_E, nrc_I$}
    \State{$B^{total} = B_E^{init} + B_I^{init}$}
    \State{parse $nrc_E$ as $(m_E, \sigma^m_E)$}
    \State{parse $nrc_I$ as $(m_I, \sigma^m_I)$}
    \State{parse $m_E$ as $(refund || nT || nwpk_E || B^{full}_E, nrt_E, k_E, hc^k_E)$}
    \State{\Comment{$B^{full}_E \gets$ balance if nanopayment channel were saturated}}
    \State{parse $nT$ as $(\delta_C, \delta_R, n, hc^0)$}
    \State{$\delta_E \gets \delta_C$ if (EndUser = Client) else $ \delta_R$}

    \If{$nrc_E = \bot$}
      \State{\Comment{$E$ failed to respond closure request in time}}
      \State{\Return{$(0, B_{total})$}}
    \EndIf{}
    \If{$\neg $Verify$(pk_E, m_E, \sigma^m_E) \vee \neg $Verify$(pk_I, m_I, \sigma^m_I)$}
      \State{\Return{$\bot$} \Comment{messages could not be authenticated}}
    \EndIf{}
    \If{$\neg $Verify$(pk_I, refund || nT || nwpk_E || B^{full}_E, nrt_E)$}
      \State{\Return{$(0, B_{total})$} \Comment{$E$ is attempting to use invalid token}}
    \EndIf{}
    \If{$revoked \in m_I$}
      \State{parse $m_I$ as $(revoked, \sigma^{rev(nrt)}_E)$}
      \If{Verify$(nwpk_E, \sigma^{rev(nrt)}_E)$}
        \State{\Return{$(0, B_{total})$} \Comment{$E$ is trying to use old channel}}
        \Else{}
        \State{\Return{$(B_{total}, 0)$} \Comment{invalid revocation from $I$}}
      \EndIf{}
    \EndIf{}
    \State{\Comment{micropayments settled, now resolve nanopayments}}
    \State{parse $m_I$ as $(accepted, k_I, hc^0_I)$}
    \If{$k_I \leq k_E \leq n \wedge $VerifyHC$(hc^0, k_E, hc^k_E) )$}
      \State{\Comment{$E$ has the highest hash preimage}}
      \State{$B^{final}_E = B^{full}_E - \delta_E * (n-k_E)$}
      \State{$B^{final}_I = B_{total} - B^{full}_E + \delta_E * (n-k_E)$}
    \EndIf{}
    \If{$k_E \leq k_I \leq n \wedge $VerifyHC$(hc^0, k_I, hc^k_I) )$}
      \State{\Comment{$I$ has the highest hash preimage}}
      \State{$B^{final}_E = B^{full}_E - \delta_E * (n-k_I)$}
      \State{$B^{final}_I = B_{total} - B^{full}_E + \delta_E * (n-k_I)$}

    \EndIf{}
    \State{\Return{$(B^{final}_E, B^{final}_I)$}}

    \EndFunction{}
  \end{algorithmic}
\end{algorithm}


\section{Formal definitions and proofs}
\label{sec:proof}

%Full paper available at http://u6qr2zmt2bhiw5tm.onion/gender-jubilant,  using Tor Browser (total of 20 pages with formal verifications).

\subsection{Definitions}

\paragraph*{Anonymity and Security for nanopayment channel}
Our objective in this work is to provide an efficient, secure and
privacy-preserving payment system for Tor network bandwidth. Our nanopayment
channel is built on the top of an existing micropayment channel as designed by
Green and Miers~\cite{green2017bolt}. Intuitively, the Pay protocol of their
bidirectional channel is replaced by our set of Nano-Setup, Nano-Establish,
Nano-Pay and Nano-Close protocols which allows high-granularity payments of up
to $n$ iterations at the cost of roughly two Pay protocols.
%We require that the Tor client must be responsible for initiating and closing a nanopayment channel.
We require that the intermediary does not learn more than the number of
nanopayments realised between an unknown Tor client and a unknown
relay.~\footnote{Due to the fact that moneTor nanopayment channels are
  inherently transparently, we do not require that Nano-Setup and Nano-Establish
  protocols are unlinkable to the Nano-Close protocol from the perspective of
  the relay and the intermediary.} Moreover, we require that the nanopayment
protocol always produce a correct outcome for each valid execution of the
protocol.  Informally, the anonymity guarantees provided by the nanopayment
channel states that any relay (except the Guard relay) of a circuit learns no
information except that a valid nanopayment channel establishment, payment or
closure has occurred over an open micropayment channel with some intermediary. A
particular relay should not be able to link any two nanopayment channel for
separate circuits that it operates.

We reuse the payment anonymity and balance properties of Green and
Miers~\cite{bolt-eprint} for an Anonymous Payment Channel (APC scheme) but we
adapt them for our tripartite protocol. The scheme requires a privacy property
that holds against the Intermediary, a privacy property that holds against a
relay, and a balance property to define monetary security.  We prove that if
there exists an adversary able to break the anonymity property, then this
adversary is able to distinguish the Real experiment from the Ideal experiment
of an APC scheme with non-negligible advantage. Furthermore, we prove that the
only adversary able to break the balance property is an adversary able to break
preliminary security assumptions.

\subsubsection{Payment anonymity with Intermediary:}
\label{def:anon1}

Let $\adv$ be an adversary playing the role of Intermediary. We consider an experiment involving P customers (a.k.a. Tor client) and Q relays, each interacting with the Intermediary as follows. First, $\adv$ is given $pp$, then outputs $T_\mdv$. Next $\adv$ issues the following queries in any order:\\

\textbf{Initialize channel for $\cdv_i$ and $\rdv_j$.} When $\adv$ makes this query on input $B^{cust}$, $B^{inter}$, it obtains the commitment $T^i_\cdv$ generated as
$$(T^i_\cdv, csk^i_\cdv) \sample Init_\cdv(pp, B^{cust}, B^{inter})$$
where the customer might also be a relay. In this case, the Intermediary obtains the commitment $T^j_\rdv$ generated as $$(T^j_\rdv, csk^j_\rdv) \sample Init_\rdv(pp, B^{relay}, B^{inter})$$

\textbf{Establish channel with $\cdv_i$ and $\rdv_j$.} In this query, $\adv$ executes the Establish protocol with $\cdv_i$ (resp. $\rdv_j$) as
$$Establish(\{\cdv(pp, T_\mdv, csk^i_\cdv)\},\{\adv(state)\})$$
Where $state$ is the adversary's state. We denote the customer's output as $w_i$, where $w_i$ may be $\bot$.\\

\textbf{Nano-Setup from $\cdv_i$.} In this query, if $w_i \neq \bot$, then $\adv$ executes the Nano-Setup to escrow $\epsilon$ with $\cdv_i$ as:
$$\operatorname{Nano-Setup}(\{\cdv(pp, \epsilon, w^i_\cdv)\}, \{\adv(state)\})$$

Where $state$ is the adversary's state. We denote the customer's output as $w^i_\cdv$, the hashchain root $hc^0$, the customer's nanopayment secret $ncsk_\cdv$, the customer's state $nS_\cdv$ and the refound token $nrt_\cdv$, where any may be $\bot$.\\

\textbf{Nano-Establish from $\rdv_j$}. In this query, if $w^j_\rdv$ and $nT$ $\neq \bot$, then $\adv$ executes the Nano-Establish to register the nanopayment channel with the relay $\rdv_j$ as:
$$\operatorname{Nano-Establish}(\{\rdv(pp, w^j_\rdv, nT)\}, \{\adv(state)\})$$

Where $state$ is the adversary state. We denote the relay's output as $w^j_\rdv$, the refound token $nrt_\rdv$, the relay's nanopayment secret $ncsk_\rdv$ and the state of the relay's nanopayment channel $nS_\rdv$. \\

\textbf{Nano-Close from $\cdv_i$ and $\rdv_j$.} In this query, if $\epsilon^i_\cdv$, $nT$, $ncsk_\cdv$ and $nS_\cdv$ $\neq \bot$, then $\adv$ executes the Nano-Close to close the nanopayment channel and update the micropayment wallet with $\cdv_i$ (resp. $\rdv_j$).
$$\operatorname{Nano-Close}(\{\cdv(pp, \epsilon^i_\cdv, nT, ncsk_\cdv, nS_\cdv)\}, \{\adv(state)\}) \rightarrow w^i_\cdv$$
Where $state$ is the adversary's state. We denote the customer's and relay's output as $w^i_\cdv$ (resp. $w^j_\rdv$), where it may be $\bot$.\\

\textbf{Finalize with $\cdv_i$ (resp. $\rdv_j$).} When $\adv$ makes this query, it obtains $rc^i_\cdv$, computed as $rc_\cdv \sample Refund(pp, w^i_\cdv)$.

We say that $\adv$ is $legal$ if $\adv$ never asks to spend from a wallet where $w^i_\cdv$ or $w^j_\rdv$ is $\bot$ or undefined, and where $\adv$ never asks $C_i$ to spend unless the customer has sufficient balance to complete the spend.

Let $pp'$ be an auxiliary trapdoor not available to the participants of the real protocol. We require the existence of a simulator $\sdv^{X-Y(\cdot)}(pp, pp', \cdot)$ such that for all $T_\mdv$, no allowed adversary $\adv$ can distinguish the following two experiments with non-negligible advantage:\\
\textbf{Real experiment.} In this experiment, all responses are computed as described in our Algorithms.\\
\textbf{Ideal experiment.} In this experiment, the micropayment operations are handled using the procedure above. However, for the nanopayment procedures, $\adv$ does not interact with $\cdv_i$ and $\rdv_j$ but instead interacts with a simulator $\sdv^{X-Y(\cdot)}(pp,pp',\cdot)$.

\subsubsection{Payment anonymity with the relay.}
\label{def:anon2}

Let $\adv$ an adversary playing the role of Relay. We consider an experiment involving P customers (a.k.a. Tor clients), each interacting with the Relay as follows. First, $\adv$ establishes a micropayment channel with the Intermediary. Next, $\adv$ issues the following queries in any order:\\

\textbf{Nano-Establish from $\cdv_i$.} In this query, $nT$ may be $\bot$, then $\adv$ executes only the part of Nano-Establish which interacts with $\cdv_i$:
$$\operatorname{Nano-Establish}(\{\cdv(pp, nT)\}, \{\adv(state)\})$$

Where $state$ is the adversary state. We denote the customer's output $nT$, which may be $\bot$.


\textbf{Nano-Pay from $\cdv_i$.} In this query, $nT \neq \bot$ and $p_k$ may be $\bot$, then $\adv$ executes the Nano-Pay protocol for an amount $\delta$ with $\cdv_i$ as:
$$\operatorname{Nano-Pay}(\{\cdv(pp, \delta, p_k)\}, \{\adv(state)\})$$

Where $state$ is the adversary's state and $p_k$ is the preimage of the current hash stored in the adversary's state, or $\bot$.

We say that $\adv$ is $legal$ if $\adv$ never asks to spend more than $n*\delta$.

Let $pp'$ be an auxiliary trapdoor not available to the participants of the real protocol. We require the existence of a simulator $\sdv^{X-Y(\cdot)}(pp, pp', \cdot)$ such that for all $T_\mdv$, no allowed adversary $\adv$ can distinguish the following two experiments with non-negligible advantage:\\
\textbf{Real experiment.} In this experiment, all responses are computed as described in our Algorithms.\\
\textbf{Ideal experiment.} In this experiment, the micropayment operations and nanopayment operations with the Intermediary are handled using our algorithms. However, for the nanopayment procedures between the Tor client and the adversary relay, $\adv$ does not interact with $\cdv_i$ but instead interacts with $\sdv^{X-Y(\cdot)}(pp,pp',\cdot)$.

\subsubsection{Payment Security (Balance)}
\label{def:balance}

Let $\adv$ an adversary playing the role of Relay. We consider an experiment
involving $P$ honest Tor clients $\cdv_1,..., \cdv_P$ interacting with the
relay. We assume the micropayment channels have been properly setup and
established with the intermediary and that the intermediary continues to
interact honestly with the client and relay.

Given the micropayment channel setup and established, parties hold funds valued
at $B^{cust}$ and $B^{\adv}$. Let $bal_\adv \leftarrow 0$ be the amount of funds
the adversary may claim. Now $\adv$ may issue the following queries in any order:\\

\textbf{Nano-Establish from $\cdv_i$.} In this query, $nT$ may be $\bot$, then $\adv$ executes only the part of Nano-Establish which interacts with $\cdv_i$:
$$\operatorname{Nano-Establish}(\{\cdv(pp, nT)\}, \{\adv(state)\})$$

Where $state$ is the adversary state. The adversary obtains $nT$ and establishes the nanopayment channel with the Intermediary.

\textbf{Nano-Pay from $\cdv_i$.} The nanopayment channel has been correctly
established before. This query can executed up to $n$ times before
\textbf{Nano-close} is called. For each execution, $nT \neq \bot$ and $p_k$ may
be $\bot$.  $\adv$ executes the Nano-Pay protocol for an amount $\delta$ with
$\cdv_i$ as:

$$\operatorname{Nano-Pay}(\{\cdv(pp,\delta,p_k)\},\{\adv(state)\}) \rightarrow p_k$$

If $H(p_k)$ matches the hash stored in the adversary's state, then $bal_\adv = bal_\adv+\delta$ and $H(p_k)$ is stored in the internal state. If it does not match, we output $\bot$.

\textbf{Nano-Close with Intermediary.} In this query, $\epsilon_\adv \leftarrow k*\delta$ for k Nano-Pay executions. $nT, ncsk_\adv, nS_\adv \neq \bot$, then $\adv$ executes the Nano-Close protocol to close its leg of the nanopayment channel and claim funds to the Intermediary.

$$Nano\-close(\{\adv(pp, \epsilon^i_\adv, nT, ncsk_\adv, nS_\adv)\},$$
$$\{Intermediary(state)\}) \rightarrow w^i_\adv$$

We denote the adversary output $w^i_\adv$, where it may be $\bot$. The Tor client closes also its leg of the nanopayment channel with the intermediary to transfer $k*\delta$ and update its wallet accordingly.

At any point, all parties have the option to call Nano-Refund to initiate a partial or full refund of their escrowed fund and close the nanopayment channel.

We say that $\adv$ is $legal$ if it never agrees to execute the Nano-Pay protocol upon $nT = \bot$. We further restrict $\adv$ to establish one nanopayment channel per micropayment channel established with any Tor client. $\adv$ wins if after executions of queries, $bal_\adv > n*\delta$.

\subsection{Theorem}
The nanopayment channel scheme satisfies the properties of anonymity
(\ref{def:anon1}, \ref{def:anon2}) and security (\ref{def:balance}) under the
restriction that the adversary does not abort before Nano-Close finished, the
assumptions that the commitment scheme are secure, the zero-knowledge system is
simulation extractable and zero-knowledge, and the hash function used to create
the hashchain and verify the preimage during the Nano-Pay is a cryptographic
secure hash function.

\subsection{Proofs}

\subsubsection{Anonymity}

\sloppy We prove that the nanopayment channel scheme satisfies our anonymity properties using a simulator $\sdv^{X-Y(\cdot)}(pp,pp',\cdot)$ such that no allowed adversary $\adv$ can distinguish the Real experiment from the Ideal experiment with non-negligible advantage. The way this proof proceeds requires honest runs of the appropriate algorithms for the micropayment channel. When Nanopayment channel operations are called, the client side or relay side of the protocol is emulated by the simulator for the Ideal experiment. To prove that they are indistinguishable, we borrow Green and Miers's proof and extend it to our notion of payment anonymity to the intermediary, and to the relay. We start with the Real experiment and we create Games which modify elements of the protocol until we match the Ideal experiment conducted with the simulator $\sdv$.

Let be $\nu_1, \nu_2$ be negligible functions and let \textbf{Adv[Game i]} be $\adv$'s advantage in distinguishing the output of \textbf{Game i} from the Real Distribution. \\


\textbf{Game 0.} This is the Real experiment: Nano-Setup, Nano-Establish and Nano-Close between customers (Tor clients) and the Intermediary.

\textbf{Game 1.} This game is identical to \textbf{Game 0} except that we replace NIZK proofs generated by the customer at the Nano-Setup and Nano-Close with simulated proofs. If the proof system is zero-knowledge, then \textbf{Adv[Game 1] $\leq \nu_1 + \nu_2$}. With $\nu_1$ being the information given by the customers and $\nu_2$ the information given by the relay.

\textbf{Game 2.} This game is identical to \textbf{Game 1} except that we replace the commitments $nwCom_C$, $nwCom_R$, $wCom_C'$ and $wCom_R'$ by commitments on random messages. If the commitment scheme is computationally hiding, then \textbf{Adv[Game 2] $-$ Adv[Game 1]} $\leq \nu_1+\nu_2$.

\textbf{Game 3.} This game is identical to \textbf{Game 2} except that we
replace the root of the hashchain $HC[0]$ by a value generated from
Random(). Note that Random() was also used for the original value, therefore
\textbf{Adv[Game 3] $-$ Adv[Game 2]} $= 0$.

\textbf{Game 4.} This game is identical to \textbf{Game 3} except that we replace $wpk_C, nwpk_C, wpk_R, nwpk_R$ with random keys using the KeyGen algorithm described for anonymous micropayment channels. Since the distribution is identical to the distribution of original values, \textbf{Adv[Game 4] $-$ Adv[Game 2]} $= 0$

Since $\adv$ cannot distinguish the real experiment from the Ideal experiment obtained in \textbf{Game 4.} with non-negligible advantage, the interaction between customers and Intermediary is anonymous.

Now, we have to prove the indistinguishably between the Real experiment and the Ideal experiment for the payment anonymity property with the relay.  We proceed with the same logic:\\

\textbf{Game 0'.} This is the Real experiment: Nano-Establish and Nano-Pay between  Tor clients and relays.

\textbf{Game 1'.} This game is identical to \textbf{Game 0'} except that we replace the root of the hashchain $HC[0]$ by a value generated from Random() in the Nano-Establish interaction. Note that Random() was also used for the original value, therefore \textbf{Adv[Game 1'] $-$ Adv[Game 0']} $= 0$

\textbf{Game 2'.} This game is identical to \textbf{Game 1'} except that we replace the preimage $p_k$ sent to the relay by a value generated from Random(). In the random oracle model, both original value and simulated one provide from the same distribution, hence \textbf{Adv[Game 2'] $-$ Adv[Game 1']} $= 0$

Since \textbf{Game 2'} is identical in the Ideal experiment, the interaction between Tor clients and relays is anonymous.

By showing that the interaction with the Intermediary and the interaction with the relay through the nanopayment algorithms are anonymous, we conclude that our nanopayment channel is anonymous.

\subsubsection{Security (Balance)}

We prove that the Nanopayment channel satisfies the security definition if the micropayment channel is itself secure, if the hash function used is a cryptographic hash function (i.e. it behaves like a random function, it's easy to compute a hash of any given data, it's computationally hard to find a valid preimage of a given hash value, and it is unlikely to find two different pieces of data that hash to the same value), and if the signature scheme is EU-CMA secure (i.e. Existential Unforgeability under a Chosen Message Attack).

To win, $\adv$ must claim more money than the agreed upon price between a honest
Client and the adversary. The adversary must make this claim while running a
protocol that is indistinguishable from the honest protocol. We proceed by
showing that $\adv$ cannot diverge from the protocol without breaking classical
security assumptions or aborting.

\textbf{Game 0.} This is the Real experiment.

\textbf{Game 1.} This game is identical to Game 0 except that we replace $hc^0$ in $nT$ by a value chosen by $\adv$ from a hashchain created by $\adv$. From this hashchain, $\adv$ creates $nT'$. If the Intermediary is honest, the nanopayment cannot be established because $nT'$ is unknown to the Intermediary for this micropayment channel. If the Intermediary is dishonest, then it can accept $nT'$ but cannot prove, under the assumption that the signature scheme is EU-CMA secure, that $nT'$ was issued by the client. Hence, \textbf{Adv[Game 1]} $\leq \nu_1$.

\textbf{Game 2.} This game is identical to Game 1 except that $\adv$ tries in the Nano-Pay protocol to find herself a preimage to the stored hashchain, and claim more $\delta$ fund. Assuming the hash function is a cryptographic hash function, then the adversary cannot find a preimage unless the Tor client sends it to issue a payment. Hence, \textbf{Adv[Game 2]} $\leq \nu_2$.

Finally, the Nano-Close protocol borrows the micropayment Pay protocol to update the micropayment wallet according to the number of preimage the adversary received from the Tor client. The Pay protocol has been proved secure by Green and Miers, hence we observe that the adversary cannot win the game with a non-negligible probability.

\section{Scheduling insights}
\label{sec:scheduling}

\subsection{background: scheduling}

Tor handles multiple queues of cells for each circuit and writes cells in the outbound connection while favoring bursty over bulky traffic.
The main idea is to prioritize circuit handling interactive data streams, like chats or web browsing.
Tor uses a heuristic called EWMA~\cite{tang2010improved} (i.e., computes the exponentially weighted moving average for the number of cells sent on each circuit) to decide which circuit to prioritize.
Recently, the Kist~\cite{jansen2014never} scheduler improved the efficiency of EWMA by reducing the congestion on the kernel outbound queue and pushing back this delicate problem on to the Tor layer.

In an ideal network, we might expect that traffic movement is an exclusive function of the raw bandwidth capacity in each edge connection and the scheduling algorithm implemented at each node.
The Tor network employs EWMA to favor interactive web clients over continuous bulk clients.
In moneTor, we originally proposed to modify EWMA with a simple linear scaling factor that would favor paid circuits.

\subsection{Obstacles}

Implementation into the concrete Tor infrastructure has proven to be a considerably more complex problem.
Upon failing to achieve meaningful differentiation with low values of $\beta$, we adopted a more blunt policy which \emph{always} services premium circuits first and implemented it in a zero-overhead version of moneTor.\footnote{This ideal version of moneTor strips away all payment operations and instead passes a single signal through the network to distinguish premium circuits.}
The results are displayed in Figure~\ref{fig:scheduling_priority}.
Although we observed some moderate differentiation, the difference falls well short of the benefit needed to incentivize paid users as well as our expectations for such an inequitable scheduling policy.
The same observation holds even under heavy levels of induced congestion.

This result is a severe issue since all previous works are using strategies that make local scheduling decisions on each relay to serve premium bandwidth.
We observe that offering bandwidth priority based on local decisions would be ineffective.

\begin{figure} \centering
  \includegraphics[trim={0 3cm 0 3cm}, clip,
    width=0.32\textwidth]{images/scheduling_priority.pdf}
  \caption[Prioritized Scheduling]{Prioritized Scheduling --- CDF download times
    for superimposed web and bulk clients where premium status is enforced only
    via scheduling. Almost no priority is observed.}
  \label{fig:scheduling_priority}
\end{figure}

\subsection{Investigation and discussion}

Our negative results indicate that scheduling is not the most decisive determining factor in performance.
To verify this hypothesis, we studied the incoming queue from which the scheduler can select new active circuits.
Figure~\ref{fig:scheduling_far} illustrates the temporal load in the queue at a single exit relay over one minute.
The height of the curve represents the total number of cells waiting to be serviced at each continuous point in time, while the colors group quantities of cells that belong to the same circuit.
Figure~\ref{fig:scheduling_close} displays a subset of the same information within a smaller time interval.\footnote{While the graph has the visual appearance of a bar graph, this is just a function of the striking data pattern.
In actuality, the plot displays a stacked area graph.}

In the figures mentioned earlier, notice that the queue is only populated for a period of 10 $ms$ before it is completely flushed, implying the queue spends the vast majority of its time empty.
This 10 $ms$ window is a product of Tor's internal event handling framework and is consistent with data from Jansen et al~\cite{jansen2018kist}.
We found in an analysis of the line-by-line observations of the queue activity that while cells get flushed in the correct order, they appear in the queue at roughly equal proportions.
In effect, bandwidth in our simulation is not constrained by the ordering policy of the scheduler but rather by the rate at which they arrive from the network.
As a consequence, local decisions at a particular relay for scheduling falls short of offering the expected priority.
Note that Jansen \textit{et al.}~\cite{jansen2018kist} discovered the same problem, which motivated the deployment of KIST.
Our results show that despite the use of KIST, relays may still be able to flush all queues at once, dismissing the effect of the scheduler's choice.

\begin{figure} \centering
  \includegraphics[width=0.49\textwidth]{images/scheduling_far.png}
  \caption[Queue Temporal Profile (60 seconds)]{Queue Temporal Profile (60
    seconds) --- Size of the scheduling buffer over time at a single exit relay
    in terms of number of cells. Colors group cells belonging to the same
    circuit.}
  \label{fig:scheduling_far}
\end{figure}

\begin{figure} \centering
  \includegraphics[width=0.49\textwidth]{images/scheduling_close.png}
  \caption[Queue Temporal Profile (2 seconds)]{Queue Temporal Profile (2
    seconds) --- Size of the scheduling buffer over time at a single exit relay
    in terms of number of cells. Colors group cells belonging to the same
    circuit.}
  \label{fig:scheduling_close}
\end{figure}


We now proceed to investigate why local relay scheduling does not work as expected.
Why do we fail to reproduce the positive results from previous works such as BRAIDS and LIRA despite using the same methodology?
First, it may be the case that other network control mechanisms within the Tor codebase constrain the flow of cells, rendering the mostly idle scheduler to be ineffective.
This could reflect any combination of factors including point-to-point flow control, connection throttling, or some less documented threshold embedded in the code.
However, we do not believe that the differences in the code explain fully explain our results.
As shown by our positive results in Section~\ref{sec:priority_exp}, the performance increases when we increase the circuit windows, a result which contradicts previous works exploring the effect of window sizes~\cite{archive-2009-mail, kiraly2008solving, dingledine2009performance}.
Moreover, the Tor code implementing these aspects did not meaningfully change over the years.
This information is likely linked to our failure to achieve prioritization via the scheduler.

To explain those oddities, we found and verified the following reason: the constraining network bottleneck has moved from the Tor network itself to the exit relay interface with external servers on the web.
In this scenario, cell queuing within Tor is not nearly as important as the TCP/IP packet handling at each exit relay.
Both approaches to prioritize flows are complementary: when the congestion is inside the Tor network, applying local scheduling policies makes an efficient priority mechanism, as demonstrated by previous works.
Also, in such a situation, priority based on the flow-control (that is, a function of the global circuit) would not be efficient because all cells would spend the majority of the time waiting in relays' FIFO queues.
Conversely, if the congestion is outside the Tor network (between the exit relays and the destination), then local scheduling policies would fall short of making any prioritization as showed in Figure~\ref{fig:scheduling_priority}.
Priority based on the flow-control would achieve to prioritize flows in this case, as demonstrated in Section~\ref{subsec:experiments}.

The shift of congestion from the internal Tor network to the exit gateways explains why our scheduling results in Figure~\ref{fig:scheduling_priority} are different from BRAIDS and LIRA.
Indeed, BRAIDS run experiments with Tor version \texttt{0.2.0.35} and a network history from January 2010~\cite{braids-repository}.
This Tor version's commit is before a major change in the path selection mechanism that reduced Tor's internal congestion problem and greatly improved the performance.
Mike Perry's commit \texttt{0ff86042ac16} affecting bandwidth weights introduced this major change in September 2010.
The bandwidth-weights are a set of weights specified in the directory specifications, section 3.8.4~\cite{dirspec}, that aim at balancing the overall network usage.
Those weights are critical for the network performance, and also for anonymity~\cite{waterfilling-pets2017, wf_proposal}.
Importantly, the benefit of Mike Perry's bandwidth-weights is proportional to the inequalities between the overall bandwidth in-circuit positions, which has grown rapidly over the years, as shown in Figure~\ref{fig:bw_inequalities}.

\begin{figure}
  \includegraphics[scale=0.415]{images/bandwidth-flags-2011-01-01-2019-02-25.pdf}
  \caption{Evolution of bandwidth aggregated by relay flags.}
  \label{fig:bw_inequalities}
\end{figure}

LIRA's experiments ran on \texttt{tor-0.2.3.13-alpha} from March 2012, which benefits from Mike Perry's bandwidth-weights.
This timeline may explain why LIRA has less impressive priority advantage than the one exposed in BRAIDS.
Experiments in LIRA are probably less internally congested than the ones from BRAIDS, a change that is seemingly caused by the significant change in the path selection algorithm.
However, the different congestion status may also be due to other factors, such as a different client usage model and the local scheduling policies.
LIRA used a simulated environment scaled down from an April 2012 consensus where $\approx 42\%$ of relays had an \texttt{Exit} flag
Meanwhile, only $\approx 13\%$ of relays in our experiment had an \texttt{Exit} flag.
Figure~\ref{fig:bw_comp} plots the distribution of bandwidth allocated to each position and illustrates the different state of the Tor network, which in turn is reflected in the Shadow Simulations.
Consensus data in more recent years, which we use in moneTor's experiments, indicate network conditions where exit bandwidth is much scarcer~\cite{waterfilling-pets2017}.

\begin{figure}
	\centering \includegraphics[scale=0.415]{images/bw_analysis_comp.png}
  \caption{Bandwidth distribution in consensuses used for BRAIDS, LIRA and moneTor experiments - BRAIDS does not benefit from bandwidth-weights to refill Middle, LIRA benefits from bandwidth-weights to offer balance between all positions, moneTor benefits from bandwidth-weights to balance entry and middle, yet does not have enough exit nodes to achieve the full balance}
  \label{fig:bw_comp}
\end{figure}
 
In summary, as the network grows to offer greater internal bandwidth (Figure~\ref{fig:bw_inequalities}, Figure~\ref{fig:bw_comp}), LIRA and BRAIDS's schedulers would tend to become less efficient for providing priority, as demonstrated in our experiments and showed in Figure~\ref{fig:scheduling_close}.
We should emphasize that this observation describes a coarse-grained trend.
Since our experiments ran on a scaled-down topology with simplistic models for user behavior, the results do not necessarily describe the state affairs for all relays in the real Tor network.
What can be said is that networking as a whole is an immensely complex and unpredictable domain and that the attainment of a simulation environment conducive to effective scheduling is, at the very least, nontrivial.
Any serious deployment of an incentivization scheme would require further research into robust prioritization mechanisms.
We leave this task for future work.

%Full paper available at http://u6qr2zmt2bhiw5tm.onion/gender-jubilant, using Tor Browser (total of 21 pages with extensive analysis on why a scheduling-based priority does not work well).
%Full paper available at http://u6qr2zmt2bhiw5tm.onion/gender-jubilant, using Tor Browser (total of 21 pages with extensive analysis on why a scheduling-based priority does not work well).


\end{document}
