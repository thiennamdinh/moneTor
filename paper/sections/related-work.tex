There have been a number of standalone incentivization schemes proposed over the
the past decade. We classify these into three types of strategies:
non-transferable benefits, transferable benefits, and monetary payments. The
list presented her is comprehensive to the best of our knowledge.

\subsection{Non-Transferable Benefits}

These proposals aim to recruit relays by offering some personal-use priviledged
status. These priviledges, however, are crucially non-transferable and
unexchangable for general financial profit. \flo{Giving a second read through, I think we miss here to explain why a non-transferable and nonexchangeable system might be wanted: see Roger's blogpost https://blog.torproject.org/two-incentive-designs-tor.}

\textbf{Gold Star} As one of the earliest incentive proposals, Gold Star introduces the notion of
premium bandwidth. Premium, or ``gold star'' status, is awarded exclusively to
the fastest 7/8ths of relays. Like moneTor many of the successive proposals,
Gold Star leaks traffic information from traffic differentiation. However, in
the leakage is exacerabated by the fact the pool of possible premium nodes is
effecitvely fixed~\cite{dingledine2010building}.

\textbf{BRAIDS/LIRA} The BRAIDS scheme shifts ports the notion of premium
bandwidth from a single node to tickets. Small numbers of temporary premium
tickets are freely distributed by a central ``bank'' to any client upon request
or by relays that have accumulated spent tickets from clients. Crucially, the
relay-specific tickets are designed to eliminate the double spending threat by
original ticket obtainers~\cite{jansen2010recruiting}. LIRA is an idealogical
successor to BRAIDs that seeks to reduce the scalability limits imposed by the
centralized infrastructure. In LIRA, clients probabilistically ``win'' premium
tickets without any interaction with the bank. While LIRA solves many pragmatic
shortcoming of BRAIDS, neither of these designs were construct to handle strong
threats of client cheating nor do they confer economically robust incentives for
relays~\cite{jansen2013lira} \flo{economically robust incentives was not the goal of these families of designs: maybe we should not argue that this is a problem. We mostly have a different philosophy with MoneTor, and we can't say that ours is better (unless we solve technical problems that they don't)}

\subsection{Transferable Benefits}

The intrinsic drawback in the non-transferable approach is the limited economic
appeal. In the general case, incentives that translate into financial value will
have broader demand than those that do not. The next class of transferable
benefits categorizes those schemes which do not directly confer weath to relays
operators but nevertheless offers a product that might hold value in a secondary
resale market.

\textbf{TEARS} TEARS introduces a two-layer approach whereby \emph{shallots}
token are awarded by a central bank to relays who contribute to the
network. These shallots, which may be securely exchange, can then be redeemed
for BRAIDS-style \emph{priority pass} tokens. While fully exchangable shallots
represent an economic improvement to non-transferable priviledges, these tokens
are conceptually discrete and indivisible assets that are not as easily
exchanged as true currency. Furthermore, its reliance on a centralized bank and
bandwidth measurement authority place implicit limits on its scalability and
security ~\cite{jansen2010recruiting}.

\textbf{TorPath to TorCoin} Perhaps the most radical redesign suggestion,
TorCoin proposes to revamp the Tor architecture into one which can serve as the
basis for a proof-of-bandwidth based cryptocurrency. The TorPath component
specifies verifiable pseudorandom shuffling as a method for user circuit
selection. This modified protocol would in theory provide a weakly secure means
for relays to mint new TorCoin tokens based on their bandwidth contribution. In
addition to several unresolved security issues enumerated by the authors
themselves, it is not entirely clear whether the earned TorCoins themselves
would be able to hold real-world value ~\cite{ghosh2014torpath}.

\textbf{Proof-of-work as anonymous micropayment} In this simple design,
users provide partial proof-of-work tokens that can then be redeemed by relays
for profit in real cryptocurrency mining pools. The drawback of this scheme is
simple an issue of unrealistic magnitude; the paper itself estimates that a user
who mines with consumer CPU hardware for 24 hours will be able to make only a
few cents worth of payments while incuring a much higher cost in
electricity~\cite{biryukov2015proof}.

\subsection{Monetary Payments}

Proposals in the final class offer the most direct incentivization by outlining
strategies to pay relays with real, externally valuable money. In theory, this
is the most economically robust approach and the one that MoneTor adopts.

\textbf{PAR} PAR is the earliest incentive scheme and first approach that
implies a true notion of currency. A pre-Bitcoin design, PAR utilizes a hybrid
payment scheme that implements inefficient but anonymous Chaumian e-cash
protocols and efficient but transparent probabilistic micropayments. PAR
introduces the \emph{honest but curious} bank paradigm in which the bank cannot
deanonymize client but is in control of the reserve financial assets. As with
BRAIDS, PAR suffers from unscalable centralized design~\cite{androulaki2008payment}.

\textbf{ORPay, PlusPay, CoinPay} These three protocols, part of the Chaumian
e-cash tradition of payments schemes, are a series of incremental improvements
released across two papers~\cite{chen2009xpay}~\cite{carbunar2012tipping}. In
each of the protocols, the micropayment building block is derived from Payword
hash chains, as is ours. While these schemes offer practical advantages to many
aspects of PAR, it suffers from the same \emph{honest but curious} security
assumptions while its scalability remains capped by the need for all clients to
interact with the central bank for each payment chain.

\textbf{Orchid} Orchid is an alternative project to Tor altogether which we
include for thoroughness and some broad idealogical similarities
~\cite{salamon2018orchid}. Orchid is an Ethereum-based project which proposes to
construct a brand new decentralized and market-based anonymous routing
network. However, as its stated intent has a stronger focus on cenorship
resistance than strong anonymity, it unlikely to have the same security
guarantees as Tor. Indeed, the payment protocol enlisted for Orchid payments
makes no claims on anonymity itself~\cite{pass2015micropayments}.