There have been a number of standalone incentivization schemes proposed over the
the past decade. We classify these into three types of strategies:
non-transferable benefits, transferable benefits, and monetary
payments. \footnote{Generally speaking, incentives provided by monetary payments
  are more economically robust than transferable benefits, which in turn are
  more robust than non-transferable benefits. Although we examine related works
  through the prism of economic appeal, this is not a statement of superiority
  so much as an observation on diverging motivation. Indeed, a number of
  sociological and legal implications considerations must be weighed before
  introducing financial profitability into the Tor ecosystem. We consider these
  to be out of scope for the purposes of our research.}

\subsection{Non-Transferable Benefits}

These proposals aim to recruit relays by offering some priviledged status for
personal use only. The priviledges, crucially, are non-transferable in the sense
that they cannot be exchanged for general financial profit.

\textbf{Gold Star} As one of the earliest incentive proposals, Gold Star
introduces the notion of premium bandwidth. Premium, or \emph{gold star} status,
is awarded exclusively to the fastest 7/8ths of relays. Like all of the
proposals listed, Gold Star leaks user information through priority traffic
differentiation. However, the leakage in this scheme is exacerabated by the fact
that the pool of possible premium nodes is effecitvely
fixed~\cite{dingledine2010building}.

\textbf{BRAIDS/LIRA} The BRAIDS scheme introduces \emph{tickets} to represent
premium status. Small numbers of temporary premium tickets are freely
distributed by a central \emph{bank} to any client upon request or to relays
that have accumulated spent tickets from clients. Crucially, tickets can only be
spent at a single relay defined at the time of their minting to elegantly
prevent double spending~\cite{jansen2010recruiting}. LIRA is an idealogical successor
to BRAIDs that seeks to reduce scalability limits imposed by the centralized
infrastructure. In LIRA, clients probabilistically ``win'' premium tickets
without any interaction with the bank. While LIRA solves many pragmatic
shortcoming of BRAIDS, neither of these designs were constructed to handle
strong threats of client cheating~\cite{jansen2013lira}

\subsection{Transferable Benefits}

The next class of transferable benefits descibes those schemes which do not
directly confer weath to relays operators but nevertheless offers a product that
might hold value in a secondary resale market. In the general case, these
indirect financial incentives aim to attract a broader demand than in the
non-transferable case.

\textbf{TEARS} TEARS introduces a two-layer approach whereby \emph{shallots}
token are awarded by a central bank to relays who contribute to the
network. These shallots, which may be securely exchange, can then be redeemed
for BRAIDS-style \emph{priority pass} tokens. While fully exchangable shallots
represent an economic improvement to non-transferable priviledges, these tokens
are conceptually discrete and indivisible assets that are not as easily
exchanged as true currency. Furthermore, its reliance on a centralized bank and
bandwidth measurement authority place implicit limits on its scalability and
security~\cite{jansen2010recruiting}.

\textbf{TorPath to TorCoin} Perhaps the most radical redesign suggestion,
TorCoin proposes to revamp the Tor architecture into one which can serve as the
basis for a new cryptocurrency mined with proof-of-bandwidth. The networking
component specifies verifiable pseudorandom shuffling as a new method for user
circuit selection. This modified protocol would in theory provide a weakly
secure means for relays to mint new TorCoin tokens based on their bandwidth
contribution. In addition to several unresolved security issues enumerated by
the authors themselves, it is not entirely clear whether the earned TorCoins
would be able to hold real-world value~\cite{ghosh2014torpath}.

\textbf{Proof-of-work as anonymous micropayment} In this simple design, users
provide partial proof-of-work tokens that can then be redeemed by relays for
profit in real cryptocurrency mining pools. The drawback of this scheme is
simply an issue of unrealistic magnitude; the paper itself estimates that a user
who continuously mines with typical consumer CPU hardware will be able to make
only a few cents worth of payments every 24 hours while incuring a much higher
cost in electricity~\cite{biryukov2015proof}.

\subsection{Monetary Payments}

Proposals in this final class offer the most direct incentivization by outlining
strategies to pay relays with real, externally valuable money. Generally, this
tends to be the most economically sound approach and the one that moneTor
adopts.

\textbf{PAR} A pre-Bitcoin design, PAR utilizes a hybrid payment scheme that
implements inefficient but anonymous Chaumian e-cash protocols and efficient but
transparent probabilistic micropayments. PAR introduces the \emph{honest but
  curious} bank paradigm in which the bank cannot deanonymize client but is in
control of the reserve financial assets. As with BRAIDS, PAR suffers from an
unscalable centralized design~\cite{androulaki2008payment}.

\textbf{ORPay, PlusPay, CoinPay} These three protocols, part of the Chaumian
e-cash tradition of payments schemes, are a series of incremental improvements
released across two papers~\cite{chen2009xpay}~\cite{carbunar2012tipping}. In
each of the designs, the micropayment building block is derived from Payword
hash chains, as is ours. While the schemes offer practical advantages to many
aspects of PAR, they are limited by the same \emph{honest but curious} security
assumptions and their scalability remains capped by the need for all clients to
interact with the central bank for each payment chain.

\textbf{Orchid} Orchid is an alternative project to Tor altogether which we
include for thoroughness and some broad idealogical similarities
~\cite{salamon2018orchid}. Orchid is an Ethereum-based project which proposes to
construct a brand new decentralized and market-based anonymous routing
network. However, as its stated intent has a stronger focus on cenorship
resistance than strong anonymity, it does not have comparable privacy gurantees
with Tor. Indeed, the payment protocol enlisted for Orchid payments makes no
claims on anonymity itself~\cite{pass2015micropayments}.