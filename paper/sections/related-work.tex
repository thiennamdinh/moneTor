We classify incentivization schemes into three types of strategies:

	\emph{1. Non-transferable benefits:} these proposals aim to recruit relays by offering some
	 privileged status intended for personal use which cannot be \emph{securely} sold for reimbursement
	  of financial investment~\cite{dingledine2010building,jansen2010recruiting, jansen2013lira}. Current state-of-the art schemes for Tor incentives are in this category and the next one, and we focus our comparison to them.
	  
	\emph{2. transferable benefits:} these schemes offer
privileges or products intended to hold value in a secondary resale market. These
indirect financial incentives presume to attract a broader demand than in the
non-transferable case. Partial proof-of-work tokens that can then be redeemed by relays for
profit in real cryptocurrency mining pools~\cite{biryukov2015proof} and exchangeable shallots in TEARS~\cite{jansen2014onions} are part of this category.

	\emph{3. monetary payments:} these schemes offer rewards with what might be considered
real money that holds external value. PAR~\cite{androulaki2008payment}
allows clients to send direct payments to relays in a hybrid payment scheme
which makes use of inefficient but anonymous Chaumian e-cash
protocols~\cite{chaum1988untraceable} and efficient but transparent
probabilistic micropayments. PAR introduces the \emph{honest but curious} bank
paradigm in which the bank cannot deanonymize clients but is in control of their
deposited financial assets. PAR suffers from an unscalable
design owing to its centralized architecture. Other monetary payments schemes such as Xpay~\cite{chen2009xpay} and Carbunar~\textit{et al.}~\cite{carbunar2012tipping} are limited by the same \emph{honest but curious} security assumptions and their scalability
remains capped by the need for all clients to interact with the central bank for each payment chain. The moneTor scheme can be seen as an idealogical member of this family of solutions, yet we resolve the scalability issue inherent with approaches in this category and we relieve the need to rely on a \textit{honest but curious} bank.


%\footnote{Generally speaking, incentives provided by monetary payments
%  are more economically robust than transferable benefits, which in turn are
%  more robust than non-transferable benefits. Although we examine related works
%  through the prism of economic appeal, this is not a statement of superiority
%  so much as an observation on diverging motivation. Indeed, a number of
%  sociological and legal implications must be weighed before
%  introducing financial profitability into the Tor ecosystem. We consider these
%  to be out of scope for the purposes of our research.}

%\op{Need to explain why our solution is better/more interesting than all these.}

%\paragraph*{Non-transferable benefits - Gold Star} As one of the earliest incentive proposals, Gold Star
%introduces the notion of premium bandwidth. Premium, or \emph{gold star} status,
%is awarded exclusively to the fastest 7/8 fraction of relays, signalling to the
%rest of the network that these relays can enjoy prioritized traffic. While
%highly attractive for its conceptual simplicity, Gold Star leaks a considerable
%amount of user information since priortized traffic can now be trivially
%attributed to this relatively small set of gold star
%relays~\cite{dingledine2010building}, since gold star is by design non-transferable.

\paragraph*{Non-transferable benefits - BRAIDS/LIRA} The BRAIDS scheme introduces \emph{tickets} to represent
premium status. Users may wish to transfer tickets to other users, but this transfer must be done through a trusted-third party (hence not secure).  Small numbers of ephemeral tickets are freely distributed by a
central \emph{bank} to any client upon request or to relays that have
accumulated spent tickets from clients. Crucially, tickets can only be spent at
a single relay defined at the time of their minting to circumvent the double
spending problem~\cite{jansen2010recruiting}, yet those tickets can be exchanged at the bank for others bound to a new relay assuming that the exchange happens within the allowed time interval. BRAIDS suffers from four major concerns: 1) everyone connects to a central bank through Tor, which raises scalability issues; 2) the bank is not trustless: by design the bank can issue valid tickets for anyone, and guard relays can steal tickets in the distribution process; 3) verifying a blind signature on relay side for each payment is computationally intensive if the rate of payment is high, and current high-bandwidth Tor relays are already CPU-bound; 4) tickets are designed to be relay-specific to overcome the double-spending problem. As a consequence, this does not scale on the economic appeal: the larger the network becomes, the more the users stockpile tickets for all the relays or need to interact with the bank to exchange them.

LIRA is an ideological successor
to BRAIDS which reduces scalability problems imposed by the centralized
infrastructure and improves the relay side verification of ticket by a factor $\approx 80$. Clients in LIRA probabilistically ``win'' premium tickets
without any interaction with the bank. While LIRA improves the efficiency of
BRAIDS, and could even offer high fairness (i.e., high rate of payments), it suffers 
from the \textit{client} cheating incentive to continuously build circuits to try to 
win premium tickets~\cite{jansen2013lira, jansenblogpost}. Depending on the chosen 
value for the payment rate, this problem by itself could prevent the scheme to be a 
realistic option. If the payment rate 
is too low, then the incentive to cheat is increased because a large priority bandwidth would be rewarded between guesses. If a high payment rate is 
chosen, then guessers would have difficulties to maintain good guesses through their 
circuit lifetime (as the probability to maintain priority through guesses exponentially decreases), which makes the idea of the scheme to have increasing buyer anonymity through successful guessers to lose efficacy. Moreover, LIRA still have the major concerns 2) and 4) inherited from BRAIDS.

Compared to BRAIDS and LIRA, moneTor offers the ability to strongly prevent double-spending without suffering from an inefficient centralized exchange process~\cite{jansenblogpost}(BRAIDS) and from incentives to cheat to gain premium access and stockpile relay-specific information (LIRA). Moreover, moneTor offers high fairness with payment verification improved by a factor $\approx 6$ to LIRA (one $\hash$ operation instead of six) and $\approx 500$ to BRAIDS in a fair-exchange situation. Also, moneTor does not suffer from 2) if moneTor tokens act as a wrapper from an external cryptocurrency such as Bitcoin or Ethereum using inter-ledger protocols~\cite{back2014enabling, poon2017plasma}. Finally and more important, moneTor tokens are not relay-specific, which solves the scalability problem on the economic appeal (i.e., frozen funds are independent of the network size in moneTor).
%Finally, moneTor payments offer to the relays what we could consider real money, that can be transferred or used to buy premium bandwidth directly (while in LIRA and BRAIDS, an exchange process with the bank is needed).

\paragraph*{Transferable benefits - TEARS} TEARS introduces a two-layer approach whereby \emph{shallots}
token are awarded by a central bank to participating relays. These shallots,
which are securely exchanged tokens, can then be redeemed for BRAIDS-style
\emph{Priority Passes}. While fully exchangeable shallots represent an economic
improvement over non-transferable privileges, these tokens are conceptually
discrete and indivisible assets that are not as easily exchanged as true
currency. Our scheme improves on this model by offering arbitrarily high
transferability and divisibility of priority tokens without changing the
underlying Tor architecture, while also offering a transparent, trustless, incremental and distributed system. Also, TEARS is a discussion paper without detailed construction and experimentation. Yet, TEARS suggests to construct Priority Passes using blind signatures, which would make moneTor faster on payment verification by a factor $\approx n$ since we need one blind signature for up to $n$ nanopayments. Finally, a major difference with previous work (TEARS, LIRA, BRAIDS) is that moneTor does not require to audit relay's bandwidth to distribute tokens. Relays receive currency directly from client from the rate of provided bandwidth and potentially more from the Tor authorities after tax redistribution. The tax redistribution mechanism may involve bandwidth audits, yet this is not mandatory.

One disadvantage of moneTor's direct payments to the relays is that it offers 
an opportunity to the exit relay to inject junk traffic (e.g., 
padding cells) or to conspire with the destination to inject useless 
data to get more nanopayments. Junk traffic produced solely by the 
exit node is already a severe issue~\cite{rochet2018dropping}, and 
the Tor project produced several patches to observe bogus 
traffic and react to it. The conspirator problem is however 
intractable by nature since the junk data would appear legit, 
however our channel establishment procedure offer up to $n$ 
nanopayments which can be tuned to account for this problem. 
A conspirator can then be reduced to 
non-fair exchange problem, which is what previous works offer as a basis.

\paragraph*{Related work concluding remarks}
\flo{Maybe merging what follows with the conclusion}
Introducing a monetary system may have a social impact on relay operators and Tor users~\cite{jansenblogpost}. This social impact may also depends on the technical properties and user-convenience of the used monetary scheme (e.g., the social impact of client cheating should not be ignored). We showed in this paper that technical progresses are still possible and that successfully offering priority requires more than only using a local scheduler on each relay. Better technical achievements could likely influence the social impact. In summary, the barriers to deployment of a payment system are first a community-wide agreement on good technical properties, and then the social impact barrier. We progress on the first barrier, offering a new payment system with no client cheating incentives, anonymous, efficient, distributed, scalable with trustless entities and with secure ownership of funds.

%Furthermore, its reliance on a centralized bank and bandwidth
%measurement authority place implicit limits on its scalability and
%security~\cite{jansen2010recruiting}.

%\paragraph*{TorPath to TorCoin} TorCoin proposes to revamp the Tor
%architecture into one which can serve as the basis for a new cryptocurrency
%mined via a method called proof-of-bandwidth. The networking component specifies
%verifiable pseudorandom shuffling as the new method for user circuit
%selection. This modified protocol would in theory provide a weakly secure means
%for relays to mint new TorCoin tokens based on their bandwidth
%contribution~\cite{ghosh2014torpath}. Although the research is an ambitious
%attempt to simultaneously tackle many challenges including both the bandwidth
%measurement and relay incentivization problems, the authors are clear in stating
%that many questions remain regarding the security and feasibility of the scheme,
%which would in effect require a network-wide overhaul of the design of Tor.
%%While researching on TorCoin-like approach has the potential to help the Tor
%%project on one of the fundamental security problems (i.e., the bandwidth
%%measurement system), outcomes are not clear as many vulnerabilities and
%%deployment questions remain.

%\paragraph*{Transferable benefits - Proof-of-work as anonymous micropayment} Users in this simple design
%provide partial proof-of-work tokens that can then be redeemed by relays for
%profit in real cryptocurrency mining pools. The drawback of this scheme is
%simply an issue of unrealistic magnitude; the paper estimates that a user who
%continuously mines with typical consumer CPU hardware will be able to make only
%a few cents worth of payments every 24 hours while incurring a much higher cost
%in electricity~\cite{biryukov2015proof}.

%\paragraph*{Monetary Payment - PAR} A pre-Bitcoin design, PAR~\cite{androulaki2008payment}
%allows clients to send direct payments to relays in a hybrid payment scheme
%which makes use of inefficient but anonymous Chaumian e-cash
%protocols~\cite{chaum1988untraceable} and efficient but transparent
%probabilistic micropayments. PAR introduces the \emph{honest but curious} bank
%paradigm in which the bank cannot deanonymize clients but is in control of their
%deposited financial assets. As with BRAIDS, PAR suffers from an unscalable
%design owing to its centralized architecture.
%
%\paragraph*{Monetary Payment - ORPay, PlusPay, CoinPay} These three protocols, part of the
%Chaumian e-cash tradition of payments schemes, are a series of incremental
%improvements released across Xpay~\cite{chen2009xpay} and Carbunar~\textit{et al.}~\cite{carbunar2012tipping}.
% In each of the designs, the micropayment building block
%is derived from Payword hash chains~\cite{rivest1996payword}, as is ours. While
%the schemes offer practical advantages to many aspects of PAR, they are limited
%by the same \emph{honest but curious} security assumptions and their scalability
%remains capped by the need for all clients to interact with the central bank for
%each payment chain. The moneTor scheme can be seen as an idealogical member of
%this family of solutions. In the remainder of the paper, we will explain how our
%payment channel approach resolves the scalability problems inherent in the
%existing protocols and explore the auxillary implications in greater detail than
%in prior works.

%\subsubsection{Orchid} Orchid is an alternative project to Tor altogether which we
%include for thoroughness and some broad ideological similarities
%~\cite{salamon2018orchid}. It is an Ethereum-based project proposing to
%construct a brand new decentralized and market-based anonymous routing
%network. However, its stated intent focuses on censorship resistance rather
%than strong anonymity and therefore lacks comparable privacy guarantees with
%Tor. Indeed, the external payment protocol adopted by Orchid makes no claims on
%anonymity whatsoever~\cite{pass2015micropayments}.

%%% Local Variables:
%%% mode: latex
%%% TeX-master: "../main"
%%% End:
