We classify the previously proposed incentivization schemes into three groups:

\noindent	\emph{1. Non-transferable benefits:} these proposals aim to recruit relays by offering some
	 privileged status intended for personal use which cannot be \emph{securely} sold for reimbursement
	  of financial investment~\cite{dingledine2010building,jansen2010recruiting, jansen2013lira}. Current state-of-the art schemes for Tor incentives are in this category and the next one, and we focus our comparison to them.
	  
\noindent	\emph{2. Transferable benefits:} these schemes offer
privileges or products intended to hold value in a secondary resale market. These
indirect financial incentives presume to attract a broader demand than in the
non-transferable case. Partial proof-of-work tokens that can then be redeemed by relays for
profit in real cryptocurrency mining pools~\cite{biryukov2015proof} and exchangeable shallots in TEARS~\cite{jansen2014onions} are part of this category.

\noindent \emph{3. Monetary payments:} these schemes offer rewards
with what might be considered real money that holds external
value. PAR~\cite{androulaki2008payment} allows clients to send direct
payments to relays in a hybrid payment scheme which makes use of
inefficient but anonymous Chaumian e-cash
protocols~\cite{chaum1988untraceable} and efficient but transparent
probabilistic micropayments. PAR introduces the \emph{honest but
  curious} bank paradigm in which the bank cannot deanonymize clients
but is in control of their deposited financial assets. PAR suffers
from scalability issues owing to its strongly centralized
architecture. Other monetary payments schemes such as
Xpay~\cite{chen2009xpay} and the proposal of Carbunar~\textit{et
  al.}~\cite{carbunar2012tipping} are limited by the same \emph{honest
  but curious} security assumptions and their scalability remains
capped by the need for all clients to interact with the central bank
for each payment chain. The moneTor scheme can be seen as a member of
this family of solutions, with the following advances: moneTor removes
the need to rely on a \textit{honest but curious} bank, and offers a
solution to the scalability issue inherent to previous approaches.


%\footnote{Generally speaking, incentives provided by monetary payments
%  are more economically robust than transferable benefits, which in turn are
%  more robust than non-transferable benefits. Although we examine related works
%  through the prism of economic appeal, this is not a statement of superiority
%  so much as an observation on diverging motivation. Indeed, a number of
%  sociological and legal implications must be weighed before
%  introducing financial profitability into the Tor ecosystem. We consider these
%  to be out of scope for the purposes of our research.}

%\op{Need to explain why our solution is better/more interesting than all these.}

%\paragraph*{Non-transferable benefits - Gold Star} As one of the earliest incentive proposals, Gold Star
%introduces the notion of premium bandwidth. Premium, or \emph{gold star} status,
%is awarded exclusively to the fastest 7/8 fraction of relays, signalling to the
%rest of the network that these relays can enjoy prioritized traffic. While
%highly attractive for its conceptual simplicity, Gold Star leaks a considerable
%amount of user information since priortized traffic can now be trivially
%attributed to this relatively small set of gold star
%relays~\cite{dingledine2010building}, since gold star is by design non-transferable.

We now review these approaches in more details. 

\paragraph*{Non-transferable benefits - BRAIDS/LIRA} The BRAIDS scheme
introduces \emph{tickets} to represent premium status. Users may wish
to transfer tickets to other users, but this transfer must be done
through a trusted-third party.  Small numbers of ephemeral tickets are
freely distributed by a central \emph{bank} to any client upon request
or to relays that have accumulated tickets spent by clients. Crucially, tickets can only be spent at a single relay
defined at the time of their minting, in order to circumvent the
double spending problem~\cite{jansen2010recruiting}, yet those tickets
can be exchanged at the bank for others bound to a new relay assuming
that the exchange happens within the allowed time interval. BRAIDS
raises four major concerns: 1)~everyone connects to a central bank
through Tor, which raises scalability issues; 2) the bank is not
trustless: by design the bank can issue valid tickets for anyone, and
guard relays can steal tickets in the distribution process; 3)
verifying a blind signature on relay side for each payment is
computationally intensive if the rate of payment is high, and current
high-bandwidth Tor relays are already CPU-bound; 4) interactions with
the bank is increased by the fact that tickets are designed to be
relay-specific. As a consequence, BRAIDS can hardly scale as a result
of an economic appeal: as the size of the network increases, users
must either stockpile an increasing number of tickets for all the
relays they may use or need to interact more and more frequently with
the bank to exchange them.

LIRA is an ideological successor to BRAIDS which reduces scalability
issues and improves the efficiency of the relay side verification of
tickets by a factor $\approx 80$. Clients in LIRA probabilistically
``win'' premium tickets without any interaction with the bank. While
LIRA improves the efficiency of BRAIDS, and could even offer high
fairness (by supporting high payment rates), it suffers from a
\textit{client} cheating incentive to continuously build circuits to
try to win premium tickets~\cite{jansen2013lira,
  jansenblogpost}. Depending on the chosen value for the payment rate,
this problem by itself could prevent the scheme from offering a
realistic option. If the payment rate is too low, then the incentive
to cheat is increased because a large priority bandwidth would be
rewarded between guesses to the cheater. If a high payment rate is chosen, then
guessers would have difficulties to maintain good guesses through
their circuit lifetime (as the probability to maintain priority
through guesses exponentially decreases). As a result, the LIRA idea
to have increasing buyer anonymity through successful guessers is
losing its efficacy. Moreover, LIRA does not address Concerns 2)
and 4) inherited from BRAIDS.

Compared to BRAIDS and LIRA, moneTor offers the ability to prevent
double-spending without suffering from a centralized exchange process
(BRAIDS)~\cite{jansenblogpost}, or from incentives to cheat to gain
premium access and stockpile relay-specific information
(LIRA). Moreover, moneTor offers high fairness with payment
verification improved by a factor $\approx 6$ compared to LIRA (one
$\hash$ operation instead of six) and $\approx 500$ to BRAIDS. Also,
moneTor does not suffer from concern 2) if moneTor tokens act as a
wrapper from an external cryptocurrency such as Bitcoin or Ethereum
using inter-ledger protocols~\cite{back2014enabling,
  poon2017plasma}. \op{Do you mean that we need to be stuck to an
  existing crypto-currency if we want problem 2) to be solved? I do
  not see why, and this looks bad -- many people are likely to not
  want to be linked to an external currency.} Finally and more
important, moneTor tokens are not relay-specific, which solves the
scalability problem on the economic appeal (i.e., in moneTor, frozen
funds are independent of the network size).
%Finally, moneTor payments offer to the relays what we could consider real money, that can be transferred or used to buy premium bandwidth directly (while in LIRA and BRAIDS, an exchange process with the bank is needed).

\paragraph*{Transferable benefits - TEARS} TEARS introduces a two-layer approach whereby \emph{shallots}
token are awarded by a distributed semi-trusted bank to participating relays. These shallots,
which are securely exchanged tokens, can then be redeemed for BRAIDS-style
\emph{Priority Passes}. While fully exchangeable shallots represent an economic
improvement over non-transferable privileges, these tokens are conceptually
discrete and indivisible assets that are not as easily exchanged as true
currency. Our scheme improves on this model by offering arbitrarily high
transferability and divisibility of priority tokens without changing the
underlying Tor architecture, while also scaling on trustless entities (our intermediaries) instead of a distribution of semi-trusted banks that must be operated by trusted people. Also, TEARS is a discussion paper without detailed construction and experimentation. The moneTor scheme is also designed for fair-exchange with Bolt's ZKP construction to open and close the nanopayment channels but offering $n$ nanopayments for which the relay only has to perform $1$ $\hash$ operation to validate. This is a strong contrast to TEARS requiring a blind signature for each payment. Finally, a major difference with previous work (TEARS, LIRA, BRAIDS) is that moneTor does not require to audit relay's bandwidth to distribute tokens. Relays receive currency directly from client from the rate of provided bandwidth and potentially more from the Tor authorities after tax redistribution. The tax redistribution mechanism may involve bandwidth audits, yet this is not mandatory. The counterpart of moneTor's direct payments to the relays is
that it offers an opportunity for the exit relay to inject junk
traffic (e.g., padding cells) or to conspire with the destination to
inject useless data in order to receive more nanopayments. Junk
traffic produced solely by the exit node is already a severe
issue~\cite{rochet2018dropping}, and the Tor project produced several
patches to observe bogus traffic and react to it. The conspirator
problem is however intractable by nature since the junk data would
appear legit to the Tor circuit layer, however our channel establishment procedure offers up to
$n$ nanopayments which can be tuned to account for this problem.  A
conspirator can then be reduced to non-fair exchange problem, which is
what previous works offer as a basis.



%Furthermore, its reliance on a centralized bank and bandwidth
%measurement authority place implicit limits on its scalability and
%security~\cite{jansen2010recruiting}.

%\paragraph*{TorPath to TorCoin} TorCoin proposes to revamp the Tor
%architecture into one which can serve as the basis for a new cryptocurrency
%mined via a method called proof-of-bandwidth. The networking component specifies
%verifiable pseudorandom shuffling as the new method for user circuit
%selection. This modified protocol would in theory provide a weakly secure means
%for relays to mint new TorCoin tokens based on their bandwidth
%contribution~\cite{ghosh2014torpath}. Although the research is an ambitious
%attempt to simultaneously tackle many challenges including both the bandwidth
%measurement and relay incentivization problems, the authors are clear in stating
%that many questions remain regarding the security and feasibility of the scheme,
%which would in effect require a network-wide overhaul of the design of Tor.
%%While researching on TorCoin-like approach has the potential to help the Tor
%%project on one of the fundamental security problems (i.e., the bandwidth
%%measurement system), outcomes are not clear as many vulnerabilities and
%%deployment questions remain.

%\paragraph*{Transferable benefits - Proof-of-work as anonymous micropayment} Users in this simple design
%provide partial proof-of-work tokens that can then be redeemed by relays for
%profit in real cryptocurrency mining pools. The drawback of this scheme is
%simply an issue of unrealistic magnitude; the paper estimates that a user who
%continuously mines with typical consumer CPU hardware will be able to make only
%a few cents worth of payments every 24 hours while incurring a much higher cost
%in electricity~\cite{biryukov2015proof}.

%\paragraph*{Monetary Payment - PAR} A pre-Bitcoin design, PAR~\cite{androulaki2008payment}
%allows clients to send direct payments to relays in a hybrid payment scheme
%which makes use of inefficient but anonymous Chaumian e-cash
%protocols~\cite{chaum1988untraceable} and efficient but transparent
%probabilistic micropayments. PAR introduces the \emph{honest but curious} bank
%paradigm in which the bank cannot deanonymize clients but is in control of their
%deposited financial assets. As with BRAIDS, PAR suffers from an unscalable
%design owing to its centralized architecture.
%
%\paragraph*{Monetary Payment - ORPay, PlusPay, CoinPay} These three protocols, part of the
%Chaumian e-cash tradition of payments schemes, are a series of incremental
%improvements released across Xpay~\cite{chen2009xpay} and Carbunar~\textit{et al.}~\cite{carbunar2012tipping}.
% In each of the designs, the micropayment building block
%is derived from Payword hash chains~\cite{rivest1996payword}, as is ours. While
%the schemes offer practical advantages to many aspects of PAR, they are limited
%by the same \emph{honest but curious} security assumptions and their scalability
%remains capped by the need for all clients to interact with the central bank for
%each payment chain. The moneTor scheme can be seen as an idealogical member of
%this family of solutions. In the remainder of the paper, we will explain how our
%payment channel approach resolves the scalability problems inherent in the
%existing protocols and explore the auxillary implications in greater detail than
%in prior works.

%\subsubsection{Orchid} Orchid is an alternative project to Tor altogether which we
%include for thoroughness and some broad ideological similarities
%~\cite{salamon2018orchid}. It is an Ethereum-based project proposing to
%construct a brand new decentralized and market-based anonymous routing
%network. However, its stated intent focuses on censorship resistance rather
%than strong anonymity and therefore lacks comparable privacy guarantees with
%Tor. Indeed, the external payment protocol adopted by Orchid makes no claims on
%anonymity whatsoever~\cite{pass2015micropayments}.


%%% Local Variables:
%%% mode: latex
%%% TeX-master: "../popets_monetor"
%%% End:
