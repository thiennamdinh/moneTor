There have been a number of standalone incentivization schemes proposed over the
the past decade. We classify these into three types of strategies:
non-transferable benefits, transferable benefits, and monetary payments.

\subsection{Non-Transferable Benefits}

These proposals aim to recruit relays by offering some personal-use priviledged
status. These priviledges, however, are crucially non-transferable and
unexchangable for general financial profit.

\textbf{Gold Star} ~\cite{dingledine2010building}

\textbf{BRAIDS} ~\cite{jansen2010recruiting}

\textbf{LIRA} ~\cite{jansen2013lira}

\subsection{Transferable Benefits}

The intrinsic drawback in the non-transferable approach is the limited economic
appeal. In the general case, incentives that translate into financial value will
have broader demand than those that do not. The next class of transferable
benefits categorizes those schemes which do not directly confer weath to relays
operators but nevertheless offers a product that might hold value in a secondary
resale market.

\textbf{TEARS} ~\cite{jansen2014onions}

\textbf{TorPath to TorCoin} ~\cite{ghosh2014torpath}

\textbf{Proof-of-work as anonymous micropayment} ~\cite{biryukov2015proof}

\subsection{Monetary Payments}

Proposals in the final class offer the most direct incentivization by outlining
strategies to pay relays with real, externally valuable money. In theory, this
is the most economically robust approach and the one that MoneTor adopts.

\textbf{PAR} ~\cite{androulaki2008payment}

\textbf{XPay} ~\cite{chen2009xpay}

\textbf{Tipping pennies} ~\cite{carbunar2012tipping}

\textbf{Orchid} ~\cite{salamon2018orchid}