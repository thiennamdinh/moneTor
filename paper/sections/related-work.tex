We classify incentivization schemes into three types of strategies:
non-transferable benefits, transferable benefits, and monetary
payments.\footnote{Generally speaking, incentives provided by monetary payments
  are more economically robust than transferable benefits, which in turn are
  more robust than non-transferable benefits. Although we examine related works
  through the prism of economic appeal, this is not a statement of superiority
  so much as an observation on diverging motivation. Indeed, a number of
  sociological and legal implications must be weighed before
  introducing financial profitability into the Tor ecosystem. We consider these
  to be out of scope for the purposes of our research.}

\subsection{Non-Transferable Benefits}

These proposals aim to recruit relays by offering some privileged status
intended for personal use which cannot be securely sold for financial profit.

\subsubsection{Gold Star} As one of the earliest incentive proposals, Gold Star
introduces the notion of premium bandwidth. Premium, or \emph{gold star} status,
is awarded exclusively to the fastest 7/8 relays. While highly attractive
for its conceptual simplicity, Gold Star leaks a considerable amount of user
information as the set of relays in the Tor network is much smaller than the set
of users~\cite{dingledine2010building}.

\subsubsection{BRAIDS/LIRA} The BRAIDS scheme introduces \emph{tickets} to represent
premium status. Small numbers of ephemeral tickets are freely distributed by a
central \emph{bank} to any client upon request or to relays that have
accumulated spent tickets from clients. Crucially, tickets can only be spent at
a single relay defined at the time of their minting to circumvent the double
spending problem~\cite{jansen2010recruiting}. LIRA is an ideological successor
to BRAIDs which reduces scalability problems imposed by the centralized
infrastructure. Clients in LIRA probabilistically ``win'' premium tickets
without any interaction with the bank. While LIRA improves the efficiency of
BRAIDS, neither of these designs were constructed to prevent strong threats of
client cheating~\cite{jansen2013lira}.

\subsection{Transferable Benefits}

The next class of transferable benefits describes schemes which offers
privileges or products intended to hold value in a secondary resale market. These
indirect financial incentives presume to attract a broader demand than in the
non-transferable case.

\subsubsection{TEARS} TEARS introduces a two-layer approach whereby \emph{shallots}
token are awarded by a central bank to participating relays. These shallots,
which are securely exchanged tokens, can then be redeemed for BRAIDS-style
\emph{priority passes}. While fully exchangeable shallots represent an economic
improvement over non-transferable privileges, these tokens are conceptually
discrete and indivisible assets that are not as easily exchanged as true
currency. Furthermore, its reliance on a centralized bank and bandwidth
measurement authority place implicit limits on its scalability and
security~\cite{jansen2010recruiting}.

\subsubsection{TorPath to TorCoin} TorCoin proposes to revamp the Tor architecture into one which can serve as the
basis for a new cryptocurrency mined with proof-of-bandwidth. The networking
component specifies verifiable pseudorandom shuffling as the new method for user
circuit selection. This modified protocol would in theory provide a weakly
secure means for relays to mint new TorCoin tokens based on their bandwidth
contribution. In addition to several unresolved security issues enumerated by
the authors, it is not clear whether the earned TorCoins would be able to hold
real-world value~\cite{ghosh2014torpath}.

\subsubsection{Proof-of-work as anonymous micropayment} Users in this simple design
provide partial proof-of-work tokens that can then be redeemed by relays for
profit in real cryptocurrency mining pools. The drawback of this scheme is
simply an issue of unrealistic magnitude; the paper estimates that a user who
continuously mines with typical consumer CPU hardware will be able to make only
a few cents worth of payments every 24 hours while incurring a much higher cost
in electricity~\cite{biryukov2015proof}.
\subsection{Monetary Payments}
\label{sub:monetary}

This final class of incentivization offers rewards relays with externally
valuable money. Generally, this tends to be the most economically
sound approach and the one that moneTor adopts.

\subsubsection{PAR} A pre-Bitcoin design, PAR allow clients to send direct payments to
relays in a hybrid payment scheme which makes use of inefficient but anonymous
Chaumian e-cash protocols and efficient but transparent probabilistic
micropayments. PAR introduces the \emph{honest but curious} bank paradigm in
which the bank cannot deanonymize clients but is in control of their deposited
financial assets. As with BRAIDS, PAR suffers from an unscalable centralized
design~\cite{androulaki2008payment}.

\subsubsection{ORPay, PlusPay, CoinPay} These three protocols, part of the Chaumian
e-cash tradition of payments schemes, are a series of incremental improvements
released across two papers~\cite{chen2009xpay, carbunar2012tipping}. In
each of the designs, the micropayment building block is derived from Payword
hash chains, as is ours. While the schemes offer practical advantages to many
aspects of PAR, they are limited by the same \emph{honest but curious} security
assumptions and their scalability remains capped by the need for all clients to
interact with the central bank for each payment chain.

%\subsubsection{Orchid} Orchid is an alternative project to Tor altogether which we
%include for thoroughness and some broad ideological similarities
%~\cite{salamon2018orchid}. It is an Ethereum-based project proposing to
%construct a brand new decentralized and market-based anonymous routing
%network. However, its stated intent focuses on censorship resistance rather
%than strong anonymity and therefore lacks comparable privacy guarantees with
%Tor. Indeed, the external payment protocol adopted by Orchid makes no claims on
%anonymity whatsoever~\cite{pass2015micropayments}.
