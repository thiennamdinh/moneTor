
\subsection{Definitions}

\paragraph*{Anonymity and Security for nanopayment channel}

Our objective in this work is to provide an efficient, secure and privacy-preserving payment system for Tor network bandwidth. Our nanopayment channel is built on the top of an existing micropayment channel as designed by Green and Miers~\cite{}. Intuitively, the Pay protocol of their bidirectional channel is replaced by our set of Nano-Setup, Nano-Establish, Nano-Pay and Nano-Close protocols, which allows a finer-grained up to $n$ payments for a service at the cost of roughly~\footnote{Nano-Setup and Nano-Establish do not require that the relay and the intermediary cannot link them to the Nano-Close protection, because our nanopayment channel is inherently transparent} two Pay protocols.
%We require that the Tor client must be responsible for initiating and closing a nanopayment channel. 
We require that the intermediary does not learn more than the number of nanopayments realised between an unknown Tor client and a unknown relay. Moreover, we require that the nanopayment protocol always produce a correct outcome for each valid execution of the protocol.
Informally, the anonymity guarantees provided by the nanopayment channel states that any relay (except the Guard relay) of a circuit learns no information except that a valid nanopayment channel establishment, payment or closure has occurred over an open micropayment channel with an intermediary. Intuitively, it means that a particular relay cannot link any nanopayment channel operation for each of the circuit it handles, at the same moment or through time.

We reuse the payment anonymity and balance properties of Green and Miers~\cite{bolt-eprint} for an Anonymous Payment Channel (APC scheme) but we adapt them for our tripartite protocol. We have a privacy property that should hold against the Intermediary, and a privacy property that should hold against a relay.
We prove that if it exists an adversary able to break the anonymity and balance property, then this adversary is able to distinguish the Real experiment with the Ideal experiment of an APC scheme with non-negligible advantage.

\subsubsection{Payment anonymity with Intermediary:} 
Let $\adv$ an adversary playing the role of Intermediary. We consider an experiment involving P customers (a.k.a. Tor client) and Q relays, each interacting with the Intermediary as follows. First, $\adv$ is given $pp$, then outputs $T_\mdv$. Next $\adv$ issues the following queries in any order:\\

\textbf{Initialize channel for $\cdv_i$ and $\rdv_j$.} When $\adv$ makes this query on input $B^{cust}$, $B^{inter}$, it obtains the commitment $T^i_\cdv$ generated as 
$$(T^i_\cdv, csk^i_\cdv) \sample Init_\cdv(pp, B^{cust}, B^{inter})$$
where the customer might also be a relay. In this case, the Intermediary obtains the commitment $T^j_\rdv$ generated as $$(T^j_\rdv, csk^j_\rdv) \sample Init_\rdv(pp, B^{relay}, B^{inter})$$  

\textbf{Establish channel with $\cdv_i$ and $\rdv_j$.} In this query, $\adv$ executes the Establish protocol with $\cdv_i$ (resp. $\rdv_j$) as
$$Establish(\{\cdv(pp, T_\mdv, csk^i_\cdv)\},\{\adv(state)\})$$
Where $state$ is the adversary's state. We denote the custumer's output as $w_i$, where $w_i$ may be $\bot$.\\

\textbf{Nano-Setup from $\cdv_i$.} In this query, if $w_i \neq \bot$, then $\adv$ executes the Nano-Setup to escrow $\epsilon$ with $\cdv_i$ as:
$$Nano\-Setup(\{\cdv(pp, \epsilon, w^i_\cdv)\}, \{\adv(state)\})$$

Where $state$ is the adversary's state. We denote the custumer's output as $w^i_\cdv$, the hashchain root $nT$, the custumer's nanopayment secret $ncsk_\cdv$, the curtumer's state $nS_\cdv$ and the refound token $nrt_\cdv$, where any may be $\bot$.\\

\textbf{Nano-Establish from $\rdv_j$}. In this query, if $w^j_\rdv$ and $nT$ $\neq \bot$, then $\adv$ executes the Nano-Establish to register the nanopayment channel with the relay $\rdv_j$ as:
$$Nano\-Establish(\{\rdv(pp, w^j_\rdv, nT)\}, \{\adv(state)\})$$

Where $state$ is the adversary state. We denote the relay's output as $w^j_\rdv$, the refound token $nrt_\rdv$, the relay's nanopayment secret $ncsk_\rdv$ and the state of the relay's nanopayment channel $nS_\rdv$. \\

\textbf{Nano-Close from $\cdv_i$ and $\rdv_j$.} In this query, if $w^i_\cdv$, $nT$, $ncsk_\cdv$ and $nS_\cdv$ $\neq \bot$, then $\adv$ executes the Nano-Close to close the nanopayment channel and update the micropayment wallet with $\cdv_i$ (resp. $\rdv_j$).
$$Nano\-close(\{\cdv(pp, w^i_\cdv, nT, ncsk_\cdv, nS_\cdv)\}, \{\adv(state)\})$$
Where $state$ is the adversary's state. We denote the customer's and relay output as $w^i_\cdv$ (resp. $w^j_\rdv$), where it may be $\bot$.\\

\textbf{Finalize with $\cdv_i$ (resp. $\rdv_j$).} When $\adv$ makes this query, it obtains $rc^i_\cdv$, computed as $rc_\cdv \sample Refund(pp, w^i_\cdv)$.

We say that $\adv$ is $legal$ if $\adv$ never asks to spend from a wallet where $w^i_\cdv$ or $w^j_\rdv$ is $\bot$ or undefined, and where $\adv$ never asks $C_i$ to spend unless the customer has sufficient balance to complete the spend.

Let $pp'$ be an auxiliary trapdoor not available to the participants of the real protocol. We require the existence of a simulator $\sdv^{X-Y(\cdot)}(pp, pp', \cdot)$ such that for all $T_\mdv$, no allowed adversary $\adv$ can distinguish the following two experiments with non-negligible advantage:\\
\textbf{Real experiment.} In this experiment, all responses are computed as described in our Algorithms.\\
\textbf{Ideal experiment.} In this experiment, the micropayment operations are handled using the procedure above. However, for the nanopayment procedures, $\adv$ does not interact with $\cdv_i$ and $\rdv_j$ but instead interacts with $\sdv^{X-Y(\cdot)}(pp,pp',\cdot)$.

\subsubsection{Payment anonymity with the relay.}

Let $\adv$ an adversary playing the role of Relay. We consider an experiment involving P customers (a.k.a. Tor client), each interacting with the Relay as follows. First, $\adv$ establishes a micropayment channel with the Intermediary. Next, $\adv$ issues the following queries in any order:\\

\textbf{Nano-Establish from $\cdv_i$.} In this query, $nT$ may be $\bot$, then $\adv$ executes only the part of Nano-Establish which interacts with $\cdv_i$:
$$Nano\-Establish(\{\cdv(pp, nT)\}, \{\adv(state)\})$$

Where $state$ is the adversary state. We denote the custumer's output $nT$, which may be $\bot$.


\textbf{Nano-Pay from $\cdv_i$.} In this query, $nT \neq bot$ and $p_k$ may be $\bot$, then $\adv$ executes the Nano-Pay protocol for an amount $\delta$ with $\cdv_i$ as:
$$Nano\-Pay(\{\cdv(pp, \delta, p_k)\}, \{\adv(state)\})$$

Where $state$ is the adversary's state and $p_k$ is the preimage of the current hash stored in the adversary's state, or $\bot$.

We say that $\adv$ is $legal$ if $\adv$ never asks to spend more than $n*\delta$.

Let $pp'$ be an auxiliary trapdoor not available to the participants of the real protocol. We require the existence of a simulator $\sdv^{X-Y(\cdot)}(pp, pp', \cdot)$ such that for all $T_\mdv$, no allowed adversary $\adv$ can distinguish the following two experiments with non-negligible advantage:\\
\textbf{Real experiment.} In this experiment, all responses are computed as described in our Algorithms.\\
\textbf{Ideal experiment.} In this experiment, the micropayment operations and nanopayment operations with the Intermediary are handled using our algorithms. However, for the nanopayment procedures between the Tor client and the adversary relay, $\adv$ does not interact with $\cdv_i$ but instead interacts with $\sdv^{X-Y(\cdot)}(pp,pp',\cdot)$.

\subsubsection{Payment Security (Balance)}

Let $\adv$ an adversary playing the role of Relay. We consider an experiment involving $P$ honest Tor clients interacting with the relay. We consider the micropayment channels with the intermediary setup and established, as well as the Intermediary honestly interacting with the Tor client and the relay.
 

\subsection{Theorem}
The nanopayment channel scheme satisfies the property of anonymity under the restriction that the adversary does not abort before Nano-Close finished, and the assumptions that the commitment scheme is secure, the zero-knowledge system is simulation extractable and zero-knowledge, and the hash function used to create the hashchain and verify the preimage during the Nano-Pay is a cryptographic hash function.

\subsection{Proofs}

\subsubsection{Anonymity}

We prove that the nanopayment channel scheme satisfy our anonymity properties using a simulator $\sdv^{X-Y(\cdot)}(pp,pp',\cdot)$ such that no allowed adversary $\adv$ can distinguish the Real experiment from the Ideal experiment with non-negligible advantage. The way this proof proceeds is requiring honest run of the appropriate algorithms for the micropayment channel. When Nanopayment channel operations are called, the client side or relay side of the protocol is emulated by the simulator for the Ideal experiment. To prove that they are indistinguishable, we borrow Green and Miers's proof and extend it to our notion of payment anonymity to the intermediary, and to the relay. We start with the Real experiment and we create Games which modify elements of the protocol until we match the Ideal experiment conducted with the simulator $\sdv$.

Let be $\nu_1, \nu_2$ be negligible functions and let \textbf{Adv[Game i]} be $\adv$'s advantage in distinguishing the output of \textbf{Game i} from the Real Distribution. \\


\textbf{Game 0.} This is the Real experiment: Nano-Setup, Nano-Establish and Nano-Close between customers (Tor clients) and the Intermediary.

\textbf{Game 1.} This game is identical to \textbf{Game 0} except that we replace NIZK proofs generated by the customer at the Nano-Setup and Nano-Close with simulated proofs. If the proof system is zero-knowledge, then \textbf{Adv[Game 1] $\leq \nu_1 + \nu_2$}. With $\nu_i$ being the information given by the customers and $\nu_2$ the information given by the relay.

\textbf{Game 2.} This game is identical to \textbf{Game 1} except that we replace the commitments $nwCom_C$, $nwCom_R$, $wCom_C'$ and $wCom_R'$ by commitments on random messages. If the commitment scheme is computationally hiding, then \textbf{Adv[Game 2] $-$ Adv[Game 1]} $\leq \nu_1+\nu_2$.

\textbf{Game 3.} This game is identical to \textbf{Game 2} except that we replace  the root of the hashchain $HC[0]$ by a hash generated from Random(). Note that Random() was also used for the original value, therefore \textbf{Adv[Game 3] $-$ Adv[Game 2]} $= 0$.

\textbf{Game 4.} This game is identical to \textbf{Game 3} except that we replace $wpk_C, nwpk_C, wpk_R, nwpk_R$ with random keys using the KeyGen algorithm described for anonymous micropayment channels. Since the distribution is identical to the distribution of original values, \textbf{Adv[Game 4] $-$ Adv[Game 2]} $= 0$

Therefore, since \textbf{Game 4} is is identical to the Ideal experiment, the interaction between customers and Intermediary is anonymous.

Now, we have to prove the indistinguishably between the Real experiment and the Ideal experiment for the payment anonymity property with the relay.  We proceed with the same logic:\\

\textbf{Game 0'.} This is the Real experiment: Nano-Establish and Nano-Pay between  Tor clients and relays.

\textbf{Game 1'.} This game is identical to \textbf{Game 0'} except that we replace the root of the hashchain $HC[0]$ by a hash generated from Random() in the Nano-Establish interaction. Note that Random() was also used for the original value, therefore \textbf{Adv[Game 1'] $-$ Adv[Game 0']} $= 0$

\textbf{Game 2'.} This game is identical to \textbf{Game 1'} except that we replace the preimage $p_k$ sent to the relay by a hash generated from Random(). In the random oracle model, both original value and simulated one provide from the same distribution, hence \textbf{Adv[Game 2'] $-$ Adv[Game 1']} $= 0$

Since \textbf{Game 2'} is identical in the Ideal experiment, the interaction between Tor clients and relays is anonymous.

By showing that the interaction with the Intermediary and the Interaction with the relay through the nanopayment algorithms is anonymous, we conclude that our nanopayment channel is anonymous.

\subsection{Proof - Fund Security}