
\subsection{Definitions}

\paragraph*{Anonymity and Security for nanopayment channel}

Our objective in this work is to provide an efficient, secure and privacy-preserving payment system for Tor network bandwidth. Our nanopayment channel is built on the top of an existing micropayment channel as designed by Green and Miers~\cite{}. Intuitively, the Pay protocol of their bidirectional channel is replaced by our set of Nano-Setup, Nano-Establish, Nano-Pay and Nano-Close protocols, which allows a finer-grained up to $n$ payments for a service at the cost of roughly~\footnote{Nano-Setup and Nano-Establish do not require that the relay and the intermediary cannot link them to the Nano-Close protection, because our nanopayment channel is inherently transparent} two Pay protocols.
%We require that the Tor client must be responsible for initiating and closing a nanopayment channel. 
We require that the intermediary does not learn more than the number of nanopayments realised between an unknown Tor client and a unknown relay. Moreover, we require that the nanopayment protocol always produce a correct outcome for each valid execution of the protocol.
Informally, the anonymity guarantees provided by the nanopayment channel state that any relay (except the Guard relay) of a circuit learns no information except that a valid nanopayment channel establishment, payment or closure has occurred over an open micropayment channel with an intermediary. Intuitively, it means that a particular relay cannot link any nanopayment channel operation for each of the circuit it handles, at the same moment or through time.

We reuse the payment anonymity and balance properties of Green and Miers~\cite{bolt-eprint} for an Anonymous Payment Channel (APC scheme) but we adapt them for our tripartite protocol. We have a privacy property that should hold against the Intermediary, and a privacy property that should hold against a relay.
We prove that if it exists an adversary able to break the anonymity and balance property, then this adversary is able to distinguish the Real experiment with the Ideal experiment of an APC scheme with non-negligible advantage.

\subsubsection{Payment anonymity with Intermediary:} 
Let $\adv$ an adversary playing the role of Intermediary. We consider an experiment involving P customers (a.k.a. Tor client) and Q relays, each interacting with the Intermediary as follows. First, $\adv$ is given $pp$, then outputs $T_\mdv$. Next $\adv$ issues the following queries in any order:\\

\textbf{Initialize channel for $\cdv_i$ and $\rdv_j$.} When $\adv$ makes this query on input $B^{cust}$, $B^{inter}$, it obtains the commitment $T^i_\cdv$ generated as $(T^i_\cdv, csk^i_\cdv) \sample Init_\cdv(pp, B^{cust}, B^{inter})$ (resp. $\rdv_j$).\\

\textbf{Establish channel with $\cdv_i$ and $\rdv_j$.} In this query, $\adv$ executes the Establish protocol with $\cdv_i$ (resp. $\rdv_j$) as:
$$Establish(\{\cdv(pp, T_\mdv, csk^i_\cdv)\},\{\adv(state)\})$$
Where $state$ is the adversary's state. We denote the custumer's output as $w_i$, where $w_i$ may be $\bot$.\\

\textbf{Nano-Setup from $\cdv_i$.}. In this query, if $w_i \neq \bot$, then $\adv$ executes the Nano-Setup to escrow $\epsilon$ with $\cdv_i$ as:
$$Nano\-Setup(\{\cdv(pp, \epsilon, w^i_\cdv)\}, \{\adv(state)\})$$

Where $state$ is the adversary's state. We denote the custumer's output as $w^i_\cdv$, the hashchain root $nT$, the custumer's nanopayment secret $ncsk_\cdv$, the curtumer's state $nS_\cdv$ and the refound token $nrt_\cdv$, where any may be $\bot$.\\

\textbf{Nano-Establish from $\rdv_j$}. In this query, if $w^j_\rdv$ and $nT$ $\neq \bot$, then $\adv$ executes the Nano-Establish to register the nanopayment channel with the relay $\rdv_j$ as:
$$Nano\-Establish(\{\rdv(pp, w^j_\rdv, nT)\}, \{\adv(state)\})$$

Where $state$ is the adversary state. We denote the relay's output as $w^j_\rdv$, the refound token $nrt_\rdv$, the relay's nanopayment secret $ncsk_\rdv$ and the state of the relay's nanopayment channel $nS_\rdv$. \\

\textbf{Nano-Close from $\cdv_i$ and $\rdv_j$.} In this query, if $w^i_\cdv$, $nT$, $ncsk_\cdv$ and $nS_\cdv$ $\neq \bot$, then $\adv$ executes the Nano-Close to close the nanopayment channel and update the micropayment wallet with $\cdv_i$ (resp. $\rdv_j$).
$$Nano\-close(\{\cdv(pp, w^i_\cdv, nT, ncsk_\cdv, nS_\cdv)\}, \{\adv(state)\})$$
Where $state$ is the adversary's state. We denote the customer's and relay output as $w^i_\cdv$ (resp. $w^j_\rdv$), where it may be $\bot$.\\

\textbf{Finalize with $\cdv_i$ (resp. $\rdv_j$).} When $\adv$ makes this query, it obtains $rc^i_\cdv$, computed as $rc_\cdv \sample Refund(pp, w^i_\cdv)$.

We say that $\adv$ is $legal$ if $\adv$ never asks to spend from a wallet where $w^i_\cdv$ or $w^j_\rdv$ is $\bot$ or undefined, and where $\adv$ never asks $C_i$ to spend unless the customer has sufficient balance to complete the spend.

Let $pp'$ be an auxiliary trapdoor not available to the participants of the real protocol. We require the existence of a simulator $\sdv^{X-Y(\cdot)}(pp, pp', \cdot)$ such that for all $T_\mdv$, no allowed adversary $\adv$ can distinguish the following two experiments with non-negligible advantage:\\
\textbf{Real experiment.} In this experiment, all responses are computed as described in our Algorithms.\\
\textbf{Ideal experiment.} In this experiment, the micropayment operations are handled using the procedure above. However, for the nanopayment procedures, $\adv$ does not interact with $\cdv_i$ and $\rdv_j$ but instead interacts with $\sdv^{X-Y(\cdot)}(pp,pp',\cdot)$.

\subsubsection{Payment anonymity with the relay.}

Let $\adv$ an adversary playing the role of Relay. We consider an experiment involving P customers (a.k.a. Tor client), each interacting with the Relay as follows. First, $\adv$ establishes a micropayment channel with the Intermediary. Next, $\adv$ issues the following queries in any order:\\

\textbf{Nano-Establish from $\cdv_i$.}
\textbf{Nano-Pay from $\cdv_i$.}
