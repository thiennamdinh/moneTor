\subsection{Definitions}

\paragraph*{Anonymity and balance for nanopayment channel} Our objective in this work is to provide an efficient, correct, and privacy-preserving payment system for Tor network bandwidth.
Our nanopayment channel is built on the top of an existing micropayment channel as designed by Green and Miers~\cite{green2017bolt}.
Intuitively, we replace the Pay protocol of their bidirectional channel with our set of Nano-Setup, Nano-Establish, Nano-Pay and Nano-Close, protocols to allow high-granularity payments of up to $n$ iterations at the cost of roughly two Pay protocols.
We require that the intermediary does not learn more than the number of nanopayments realized between an unknown Tor client and an unknown relay.~\footnote{Due to the fact that moneTor nanopayment channels are inherently transparent, we do not require unlinkability between Nano-Setup/Nano-Establish and Nano-Close from the perspective of the relay and the intermediary.}
Moreover, we require that the nanopayment protocol always produce a correct outcome for each valid execution of the protocol.
Informally, the anonymity guarantees provided by the nanopayment channel states that any relay (except the guard relay) of a circuit learns no information except that a valid nanopayment channel establishment, payment, or closure has occurred over an open micropayment channel with some intermediary.
A particular relay should not be able to link any two nanopayment channels for separate circuits that it operates.

We reuse the payment anonymity and balance properties of Green and Miers~\cite{bolt-eprint} for an Anonymous Payment Channel (APC scheme), but we adapt them for our tripartite protocol.
The scheme requires a privacy property that holds against the intermediary, a privacy property that holds against a relay, and a balance property to define monetary security.
We prove that if there exists an adversary able to break the anonymity property, then this adversary can distinguish the Real experiment from the Ideal experiment of an APC scheme with a non-negligible advantage.
Furthermore, we prove that the only adversary able to break the balance property is an adversary able to break preliminary security assumptions.

\subsubsection{Payment anonymity with respect to the intermediary:}
\label{def:anon1}

Let $\adv$ be an adversary playing the role of the intermediary.
We consider an experiment involving P customers (a.k.a. Tor client) and Q relays, each interacting with the intermediary as follows.
First, $\adv$ is given $pp$, then outputs $T_\mdv$.
Next $\adv$ issues the following queries in any order:\\

\textbf{Initialize channel for $\cdv_i$ and $\rdv_j$.}
When $\adv$ makes this query on input $B^{cust}$, $B^{inter}$, it obtains the commitment $T^i_\cdv$ generated as $$(T^i_\cdv, csk^i_\cdv) \sample Init_\cdv(pp, B^{cust}, B^{inter})$$ where the customer might also be a relay.
In this case, the intermediary obtains the commitment $T^j_\rdv$ generated as $$(T^j_\rdv, csk^j_\rdv) \sample Init_\rdv(pp, B^{relay}, B^{inter})$$

\textbf{Establish channel with $\cdv_i$ and $\rdv_j$.}
In this query, $\adv$ executes the Establish protocol with $\cdv_i$ (resp.
$\rdv_j$) as $$Establish(\{\cdv(pp, T_\mdv, csk^i_\cdv)\},\{\adv(state)\})$$ Where $state$ is the adversary's state.
We denote the customer's output as $w_i$, where $w_i$ may be $\bot$.\\

\textbf{Nano-Setup from $\cdv_i$.}
In this query, if $w_i \neq \bot$, then $\adv$ executes the Nano-Setup to escrow $\epsilon$ with $\cdv_i$ as: $$\operatorname{Nano-Setup}(\{\cdv(pp, \epsilon, w^i_\cdv)\}, \{\adv(state)\})$$

Where $state$ is the adversary's state.
We denote the customer's output as $w^i_\cdv$, the hashchain root $hc^0$, the customer's nanopayment secret $ncsk_\cdv$, the customer's state $nS_\cdv$ and the refund token $nrt_\cdv$, where any may be $\bot$.\\

\textbf{Nano-Establish from $\rdv_j$}.
In this query, if $w^j_\rdv$ and $nT$ $\neq \bot$, then $\adv$ executes the Nano-Establish to register the nanopayment channel with the relay $\rdv_j$ as: $$\operatorname{Nano-Establish}(\{\rdv(pp, w^j_\rdv, nT)\}, \{\adv(state)\})$$

Where $state$ is the adversary state.
We denote the relay's output as $w^j_\rdv$, the refund token $nrt_\rdv$, the relay's nanopayment secret $ncsk_\rdv$ and the state of the relay's nanopayment channel $nS_\rdv$.
\\

\textbf{Nano-Close from $\cdv_i$ and $\rdv_j$.}
In this query, if $\epsilon^i_\cdv$, $nT$, $ncsk_\cdv$ and $nS_\cdv$ $\neq \bot$, then $\adv$ executes the Nano-Close to close the nanopayment channel and update the micropayment wallet with $\cdv_i$ (resp.
$\rdv_j$).
$$\operatorname{Nano-Close}(\{\cdv(pp, \epsilon^i_\cdv, nT, ncsk_\cdv, nS_\cdv)\}, \{\adv(state)\}) \rightarrow w^i_\cdv$$ Where $state$ is the adversary's state.
We denote the customer's and relay's output as $w^i_\cdv$ (resp.
$w^j_\rdv$), where it may be $\bot$.\\

\textbf{Finalize with $\cdv_i$ (resp.
  $\rdv_j$).}
When $\adv$ makes this query, it obtains $rc^i_\cdv$, computed as $rc_\cdv \sample Refund(pp, w^i_\cdv)$.

We say that $\adv$ is $legal$ if $\adv$ never asks to spend from a wallet where $w^i_\cdv$ or $w^j_\rdv$ is $\bot$ or undefined, and where $\adv$ never asks $C_i$ to spend unless the customer has sufficient balance to complete the spend.

Let $pp'$ be an auxiliary trapdoor not available to the participants of the real protocol.
We require the existence of a simulator $\sdv^{X-Y(\cdot)}(pp, pp',
\cdot)$ such that for all $T_\mdv$, no allowed adversary $\adv$ can
distinguish the following two experiments with non-negligible
advantage:\\

\textbf{Real experiment.}
In this experiment, all responses are computed as described in our Algorithms.\\

\textbf{Ideal experiment.}
In this experiment, the micropayment operations are handled using the procedure above.
However, for the nanopayment procedures, $\adv$ does not interact with $\cdv_i$ and $\rdv_j$ but instead interacts with a simulator $\sdv^{X-Y(\cdot)}(pp,pp',\cdot)$.

\subsubsection{Payment anonymity with respect to the relay.}
\label{def:anon2}

Let $\adv$ be an adversary playing the role of the relay.
We consider an experiment involving P customers (a.k.a.
Tor clients), each interacting with the relay as follows.
First, $\adv$ establishes a micropayment channel with the intermediary.
Next, $\adv$ issues the following queries in any order:\\

\textbf{Nano-Establish from $\cdv_i$.}
In this query, $nT$ may be $\bot$, then $\adv$ executes only the part of Nano-Establish which interacts with $\cdv_i$: $$\operatorname{Nano-Establish}(\{\cdv(pp, nT)\}, \{\adv(state)\})$$

Where $state$ is the adversary state.
We denote the customer's output $nT$, which may be $\bot$.

\textbf{Nano-Pay from $\cdv_i$.}
In this query, $nT \neq \bot$ and $p_k$ may be $\bot$, then $\adv$ executes the Nano-Pay protocol for an amount $\delta$ with $\cdv_i$ as: $$\operatorname{Nano-Pay}(\{\cdv(pp, \delta, p_k)\}, \{\adv(state)\})$$

Where $state$ is the adversary's state and $p_k$ is the preimage of the current hash stored in the adversary's state or $\bot$.

We say that $\adv$ is $legal$ if $\adv$ never asks to spend more than $n \times \delta$.

Let $pp'$ be an auxiliary trapdoor not available to the participants of the real protocol.
We say that a payment scheme offers anonymity if, for every legal $\adv$, there is a simulator $\sdv^{X-Y(\cdot)}(pp, pp', \cdot)$ such that the following two experiments cannot be distinguished with a non-negligible advantage:\\

\textbf{Real experiment.}
In this experiment, all responses are computed as described in our Algorithms.\\

\textbf{Ideal experiment.}
In this experiment, the micropayment operations and nanopayment operations with the intermediary are handled using our algorithms.
However, for the nanopayment procedures between the Tor client and the adversary relay, $\adv$ does not interact with $\cdv_i$ but instead interacts with $\sdv^{X-Y(\cdot)}(pp,pp',\cdot)$.

\subsubsection{Balance}
\label{def:balance}

Let $\adv$ be an adversary playing the role of the relay.
We consider an experiment involving $P$ honest Tor clients $\cdv_1,..., \cdv_P$ interacting with the relay.
We assume the micropayment channels are properly setup and established with the intermediary and that the intermediary continues to interact honestly with the client and relay.

Given the micropayment channel setup and established, parties hold funds valued at $B^{cust}$ and $B^{\adv}$.
Let $bal_\adv \leftarrow 0$ be the amount of funds the adversary may claim.
Now $\adv$ may issue the following queries in any order:\\

\textbf{Nano-Establish from $\cdv_i$.}
In this query, $nT$ may be $\bot$, then $\adv$ executes only the part of Nano-Establish which interacts with $\cdv_i$: $$\operatorname{Nano-Establish}(\{\cdv(pp, nT)\}, \{\adv(state)\})$$

Where $state$ is the adversary state.
The adversary obtains $nT$ and establishes the nanopayment channel with the intermediary.

\textbf{Nano-Pay from $\cdv_i$.}
The nanopayment channel has been correctly established before.
This query can executed up to $n$ times before \textbf{Nano-close} is called.
For each execution, $nT \neq \bot$ and $p_k$ may be $\bot$.
$\adv$ executes the Nano-Pay protocol for an amount $\delta$ with $\cdv_i$ as:

$$\operatorname{Nano-Pay}(\{\cdv(pp,\delta,p_k)\},\{\adv(state)\}) \rightarrow p_k$$

If $H(p_k)$ matches the hash stored in the adversary's state, then $bal_\adv = bal_\adv+\delta$ and $H(p_k)$ is stored in the internal state.
If it does not match, we output $\bot$.

\textbf{Nano-Close with intermediary.}
In this query, $\epsilon_\adv \leftarrow k \times \delta$ for k Nano-Pay executions.
$nT, ncsk_\adv, nS_\adv \neq \bot$, then $\adv$ executes the Nano-Close protocol to close its leg of the nanopayment channel and claim funds to the intermediary.

$$Nano\-close(\{\adv(pp, \epsilon^i_\adv, nT, ncsk_\adv, nS_\adv)\},$$ $$\{Intermediary(state)\}) \rightarrow w^i_\adv$$

We denote the adversary output $w^i_\adv$, where it may be $\bot$.
The Tor client also closes its leg of the nanopayment channel with the intermediary to transfer $k \times \delta$ and update its wallet accordingly.
At any point, all parties have the option to call Nano-Refund to initiate a partial or full refund of their escrowed fund and close the nanopayment channel.
We say that $\adv$ is $legal$ if it never agrees to execute the Nano-Pay protocol upon $nT = \bot$.
We further restrict $\adv$ to establish one nanopayment channel per micropayment channel established with any Tor client.
We say that a scheme guarantees a correct balance if no $\adv$ can complete, with non-neglible probability, the game described above in such a way that $bal_\adv > k \times \delta$.

\subsection{Theorem}

The nanopayment channel scheme satisfies the properties of anonymity (\ref{def:anon1}, \ref{def:anon2}) and security (\ref{def:balance}) under the restriction that the adversary does not abort before Nano-Close finishes, the restrictions that at most one nanopayment channel can be open per micropayment channel, the assumptions that the commitment scheme is secure, the zero-knowledge system is simulation extractable and zero-knowledge, and the hash function used to create the hashchain and verify the preimage during the Nano-Pay is a cryptographic hash function.

\subsection{Proofs}

\subsubsection{Anonymity}

\sloppy We prove that the nanopayment channel scheme satisfies our anonymity properties using a simulator $\sdv^{X-Y(\cdot)}(pp,pp',\cdot)$ such that no allowed adversary $\adv$ can distinguish the Real experiment from the Ideal experiment with non-negligible advantage.
The way this proof proceeds requires honest runs of the appropriate algorithms for the micropayment channel.
When Nanopayment channel operations are called, the client side or relay side of the protocol is emulated by the simulator for the Ideal experiment.
To prove that they are indistinguishable, we borrow Green and Miers's proof and extend it to our notion of payment anonymity to the intermediary and the relay.
We start with the Real experiment, and we create Games that modify elements of the protocol until we match the Ideal experiment conducted with the simulator $\sdv$.
When $\adv$ calls the simulator $\sdv$ on legal interactions, the simulator emulates the Tor client or relay part of the protocol, depending on which step of the protocol we perform.

Let be $\nu_1, \nu_2$ be negligible functions and let \textbf{Adv[Game i]} be $\adv$'s advantage in distinguishing the output of \textbf{Game i} from the Real Distribution.
\\

\textbf{Game 0.}
This is the Real experiment: Nano-Setup, Nano-Establish, and Nano-Close between customers (Tor clients) and the intermediary.

\textbf{Game 1.}
This game is identical to \textbf{Game 0} except that we replace NIZK proofs generated by the customer at the Nano-Setup and Nano-Close with simulated proofs (we assume the existence of a ZK simulation algorithm which can extract a simulated proof).
If the proof system is zero-knowledge, then \textbf{Adv[Game 1] $\leq \nu_1$}.

\textbf{Game 2.} This game is identical to \textbf{Game 1} except that we replace the commitments $nwCom_C$, $nwCom_R$, $wCom_C'$ and $wCom_R'$ by commitments on random messages. If the commitment scheme is computationally hiding, then \textbf{Adv[Game 2] $-$ Adv[Game 1]} $\leq \nu_2$.

\textbf{Game 3.} This game is identical to \textbf{Game 2} except that we replace the root of the hashchain $HC[0]$ with a value generated from Random(). Note that Random() was also used for the original value, therefore \textbf{Adv[Game 3] $-$ Adv[Game 2]} $= 0$.

\textbf{Game 4.} This game is identical to \textbf{Game 3} except that we replace $wpk_C, nwpk_C, wpk_R, nwpk_R$ with random keys using the KeyGen algorithm described for anonymous micropayment channels. Since the distribution is identical to the distribution of original values, \textbf{Adv[Game 4] $-$ Adv[Game 2]} $= 0$

We have started with the Real experiment and modified elements of the protocols from a series of Games to come up with a computationally indistinguishable experiment conducted by $\sdv$ from the Real experiment. Since $\adv$ cannot distinguish the real experiment from the Ideal experiment obtained in \textbf{Game 4.} with non-negligible advantage, the interaction between customers and intermediary is anonymous.

Now, we have to prove the indistinguishablility between the Real experiment and the Ideal experiment for the relay's payment anonymity property. We proceed with the same logic:\\

\textbf{Game 0'.} This is the Real experiment: Nano-Establish and Nano-Pay between Tor clients and relays.

\textbf{Game 1'.} This game is identical to \textbf{Game 0'} except that we replace the root of the hashchain $HC[0]$ with a value generated from Random() in the Nano-Establish interaction. Note that Random() was also used for the original value, therefore \textbf{Adv[Game 1'] $-$ Adv[Game 0']} $= 0$

\textbf{Game 2'.} This game is identical to \textbf{Game 1'} except
that we replace the preimage $p_k$ sent to the relay by a value
generated from Random(). In the random oracle model, both the original
value and the simulated one provide from the same distribution, hence \textbf{Adv[Game 2'] $-$ Adv[Game 1']} $= 0$

Since \textbf{Game 2'} is identical in the Ideal experiment, the interaction between Tor clients and relays is anonymous.

By showing that the interaction with the intermediary and the interaction with the relay through the nanopayment algorithms are anonymous, we conclude that our nanopayment channel is anonymous.

\subsubsection{Balance}

We prove that the Nanopayment channel guarantees correct balance if the micropayment channel is itself secure, the hash function behaves like a random oracle, and the signature scheme is EU-CMA secure (i.e. Existential Unforgeability under a Chosen Message Attack).

To win, $\adv$ must claim more money than the agreed-upon price between an honest client and the adversary.
The adversary must make this claim while running a protocol that is indistinguishable from the honest protocol.
The Nano-Setup protocol borrows the same structure as the provably secure Pay protocol.
Properties included the soundness of the zero-knowledge proof, the binding property of the commitment scheme, and the unforgeability of the signature scheme.
At this step, the adversary cannot win against the Tor client and claim more than $k \times \delta$ where $k$ is 0 since no Nano-Pay has been executed yet.
For the following steps of the protocol, we proceed by showing that $\adv$ cannot diverge from the protocol and claim more than $k \times \delta$ with our classical security assumptions.
If $\adv$ could succeed this game, it would mean that there exists an indistinguishable experiment from the Real experiment where $\adv$ ends up with more than $k \times \delta$.

\textbf{Game 0.} This is the Real experiment.

\textbf{Game 1.}
This game is identical to Game 0 except that we replace $hc^0$ in $nT$ by a value chosen by $\adv$ from a hashchain created by $\adv$.
From this hashchain, $\adv$ creates $nT'$.
If the intermediary is honest, the nanopayment cannot be established because $nT'$ is unknown to the intermediary for this micropayment channel.
If the intermediary is dishonest, then it can accept $nT'$ but cannot prove, under the assumption that the signature scheme is unforgeable in the usual sense (EU-CMA secure), that the client holds a refund token with $nT'$ instead of $nT$.
Hence, \textbf{Adv[Game 1]} $\leq \nu_1$.

\textbf{Game 2.}
This game is identical to Game 1 except that $\adv$ tries in the Nano-Pay protocol to find herself a preimage to the stored hashchain, and claim more than $\delta$.
Assuming the hash function is a cryptographic hash function, then the adversary cannot find a preimage unless the Tor client sends it to issue a payment.
Hence, \textbf{Adv[Game 2]} $\leq \nu_2$.

Finally, the Nano-Close protocol borrows the micropayment Pay protocol to update the micropayment wallet according to the number $k$ of preimages the adversary received from the Tor client.
The Pay protocol has been proved secure by Green and Miers; hence we observe that the adversary cannot win the game with a non-negligible probability (claim more than $k \times \delta$).

%%% Local Variables:
%%% mode: latex
%%% TeX-master: "../popets_monetor"
%%% End:
