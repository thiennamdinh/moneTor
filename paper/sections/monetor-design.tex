Where we previously assumed the existence of an ideal payment procedure, we now
describe our construction for the MoneTor payment scheme. We first summarize the
algorithmic design of our payment protocols and proceed to describe integration
details with the Tor networking layer and codebase.

\subsection{Security Model}

The privacy threat model is derived from the local active adversary
model ubiquitously found in Tor research~\cite{dingledine2004tor}. Informally, we claim the following
anonymity guarantees:

\begin{enumerate}
\item No additional party needed to operate the MoneTor system (i.e. ledgers
  and intermediaries) will know any more about a given user than is already
  known by an existing relay.
\item In typical usage, users do not leak any more information other than the
  single bit needed to differentiate premium and nonpremium users.
\end{enumerate}

Our threat model for payment security is similar to those found in prior works
in blockchain micropayment channels~\cite{poon2016bitcoin}. In such models, the
user is protected from malicious intermediaries by the ability to prove
misbehavior to a global ledger. The security guarantees on the ledger will be
subject to the guarantees made by the cryptocurrency interface~\cite{back2014enabling}\cite{poon2017plasma}. This is in direct contrast to prior to Chaumian e-cash
proposals that employ an \emph{honest but curious} model for the central
bank. In MoneTor, we prioritize payment security and ensure that nobody,
including the ledger, ever has direct control of user funds.

\subsection{Payment Protocols}

In this section we specify formal protocols that comprise the payment
scheme. All protocols are either two or three party interactions between a
subset of the following roles: $C$ (client), $R$ (relay), $E$ (end user: either
a client or relay) $I$ (intermediary), and $L$ (ledger). At this level, we assume that
one client is paying one relay through a single channel and that all network
connections have been perfectly anonymized.

The full MoneTor payment scheme is a superset of Bolt three-party bidirectional
micropayment channels and, as such, we adopt its nomenclature where possible but
adapt it to our notions of ledgers, clients, relays, etc. The following is brief
outline of the prerequisite micropayment channel procedures defined in Bolt. We
refer the reader to the original paper for more detailed
specifications~\cite{green2017bolt}.

\begin{itemize}
\item $KeyGen$: Any party generates a cryptographic keypair
\item $Init_E$: $E$ initializes her half of a micropayment channel by escrowing
  initial funds on $L$
\item $Init_I$: $I$ initializes his half of a micropayment channel by
  escrowing intial funds on $L$
\item $Establish$: $E$ and $I$ interact to establish a new, functional
  micropayment channel
\item $Pay$: $E_1$ interacts with $I$ and $E_2$ to send a single micropayment to $E_2$
\item $Refund$: $E$ closes a micropayment channel on $L$ and makes a claim on
  the escrowed balance.
\item $Refute$: $I$ closes a micropayment channel on $L$ and makes a claim on
  the escrowed balance.
\item $Resolve$: $L$ determines the final balance of funds awared to
  each party.
\end{itemize}

MoneTor makes use of the existing anonoymous micropayment structure to construct
a new layer for \emph{locally transparent nanopayments}. In this scheme,
\emph{nanopayment channels} are established in place of single micropayment
operations. Nanopayment channels allow the client to send a $n$ number of
undirectional nanopayments to the relay. Each payment represents a fixed value
$\delta$, established at the start of the nanopayment channel. Payments are
\emph{locally transparent} in the sense that each nanopayment in the same
channel is trivially linked to eachother. However, each nanopayment channel is
anonymous with respect to other nanopayment channels and micropayment
operations.

Assuming that $C$ and $R$ have both completed $Establish$ with $I$, we introduce
the following set of protocols to operate a nanopayment channel.

\begin{itemize}
\item $Nano$-$Setup$: $C$ and $I$ interact to prepare a nanopayment channel on top
  of their existing micropayment channel.
\item $Nano$-$Establish$: $C$ sends her nanopayment channel information to $R$,
  who establishes it with $I$ on top of their existing micropayment channel.
\item $Nano$-$Pay$: $C$ sends a single nanopayment channel to $R$. This is
  repeatable for up to $n$ operations.
\item $Nano$-$Close_R$: $R$ closes his nanopayment channel with $I$.
\item $Nano$-$Close_C$: $C$ closes his nanopayment channel with $I$. Note that
  this must happen after $Nano$-$Close_R$.
\end{itemize}

We also specify the following modified channel conflict resolution procedures to
ensure secure closure properties for the nanopayment scheme.

\begin{itemize}
\item $Nano$-$Refund$: $E$ closes channel on $L$.
\item $Nano$-$Refute$: $I$ closes channelon $L$.
\item $Nano$-$Resolve$: $L$ makes final determination on both micropayment and
  outstanding nanopayment balances.
\end{itemize}

The core of our nanopayment scheme is inspired by the classic \emph{Payword}
two-party micropayment scheme in which payments are encoded by successively
revealed preimages in a precomputed hash chain~\cite{rivest1996payword}. Hash
chains are perhaps the most efficient known method for representing
payments. Whereas a single Bolt micropayments require expensive zero-knowledge
proofs and at least seven protocol steps hash payments can be computed on the
order of of millions per second and each payment can be relayed with a single
256 bit message.

The challenge in this construction is to securely integrate the hash chain
concept into the existing three-party anonymous micropayment channel setup such
that all parties maintain cryptographic ownership of their funds. At the same
time, we must ensure that no deanonymizing information is leaked outside of the
nanopayment channel context.

\subsection{Nanopayment Protocols}

In this section, we provide a summarized intuition for the basic steps in the
payment protocol. For a more formal algorithmic definition, we refer the reader
to [algorithms] and [Appendix].

\subsubsection{Nano-Setup}

At the start of this protocol, $C$ has access to a micropayment wallet $w$ that
enables her to operate her micropayment channel with $I$ as well as a refund
token $rt$ that entitles her to claim her current funds on $L$ should $I$ misbehave
or go offline.

To construct a nanopayment channel, $C$ first generates an array of values $hc$
of length $n$ where $hc_i = H(hc_{i+1})$ and $hc_n$ is a random number. The root
of the hash chain $hc_0$ is used to create a globally unique nanopayment
token $nT$ that encodes the public parameters of the nanopayment channel
including the length $n$ and the per-payment value $\delta$.

$C$ sends $I$ a commitment to a fresh nanopayment channel parametrized by $nT$ along
with a zero-knowledge proof of the following statements:

\begin{enumerate}
\item The nanopayment wallet $nw$ is well-formed from $w$
\item $C$ has ownership of a micropayment channel containing at least $n *
  \delta$ funds.
\end{enumerate}

$I$ verifies these messages and blindly supplies $C$ with a new refund token $nrt$
that entitles $C$ to cash out the full balance of the micropayment channel along
with any nanopayments to $L$. $C$, now protected against misbehavior by $I$,
agrees to send a revocation token $\sigma_w$, which revokes her right to use $w$
or $rt$. $I$, now protected against double spending by $C$, can now safely
inform $C$ that the nanopayment channel has been setup.

\subsubsection{Nano-Establish}

At this point, $C$ sends $R$ the same $nT$ token used to setup the channel with
$I$. $R$ uses this token to initiate her end of the nanopayment channel with $I$
with essentially the same procedure that $C$ used in $Nano$-$Setup$. The
nanopayment channel is fully established and ready to be used.

A key observation is that both ends of the channel ($C$-$I$ and $R$-$I$)
are rooted at the same hash chain root $hc_0$. As a result, the scheme ensures
a correct-by-construction notion of \emph{atomicity} whereby both legs of the
protocol are complete at the same time.

\subsubsection{Nano-Pay}

To make the next payment $i$, $C$ simply sends the next hash preimage $hc_i$ to
$R$. Knowledge of this preimage $hc_i$ is sufficient for $R$ to prove posession
of a nanopayment. At any given time, $R$ can broadcast the tuple ($nrt$, $hc_i$)
to $L$ to prove ownership of a certain value of funds. Notice that this action
simultaneously reveals $hc_i$ to $I$, who can then claim an equivalent value of
funds from $C$.

\subsubsection{Nano-Close}

After some number of payments $k < n$ has transpired, $C$ and $R$ will both wish
to close their nanopayment channels through $I$. In this process, the $R$-$I$ leg
must be closed before the $C$-$I$ leg. This is due to the unidirectional nature
of nanopayment channels. Since payments are flowing form $C$ to $R$, $I$ must
first determine its debt to $R$ in order to know how much it can claim from $C$.

$R$ first sends to $I$ a commitment to a new micropayment wallet $w'$ and a
zero-knowledge proof of the following statements:

\begin{enumerate}
\item $w'$ is well-formed from $w$
\item The balance of $w'$ is equal to the sum of the previous wallet $w$ and
  $\delta * k$
\end{enumerate}

Once verified, $I$ issues a refund token $rt'$ on the new funds. $R$ agrees to
invalidate the nanopayment channel by issuing a revocation token $\sigma_{nw}$
to $I$. $I$ and $R$ proceed to create a blind signature on $w'$ thus validating
the wallet for fuure use.

Once $I$ has closed his nanopayment channel leg with $R$, $I$ and $C$ are free
to complete the exact same close protocol. All parties are now reverted to the
original state prior to $Nano$-$Setup$ with a securely updated balance.

\subsection{Network Protocols}

\subsection{Implementation}
