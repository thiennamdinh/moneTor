Anonymous traffic routing through Tor remains one of the most popular and
well-studied low-latency networks for censorship evasion and privacy protection
~\cite{dingledine2004tor}. In this setup, clients protect their TCP/IP metadata
by routing their traffic through an onion encryption protocol with three
randomly selected volunteer relay nodes. This path is refered to as a
circuit. The network currently consists of approx. 6,000 relays contributing
over 200 Gbit/s of bandwidth globally~\cite{portal2018tormetrics}. While Tor has
proven to be a highly effective option for privacy-seeking users, it suffers
from two orthogonal issues that are relevant to this work. The first is the
large family of collusion attacks. These threats are relevant in scenarios where
an attacker controls multiple node and is probabilistically controls more than
one hop along a single client's circuit.
path~\cite{wright2004predecessor}~\cite{murdoch2005low}. The second problem is
overall performance. Although the multi-hop overlay protocol itself implies an
inherit overhead on efficiency, Tor suffers from further traffic congestion
leading to less-than-optimal network
performance~\cite{portal2018tormetrics}[need better evidence/citations here?].

One approach to mitigated these problems is to address them separately from a
networking standpoint. Indeed, a significant portion of recent research in Tor
proposesses modifications to the core protocol itself, such as replacing the TCP+TLS part of the stack~\cite{} or designing a better kernel congestion-aware scheduler~\cite{}. A second approach
is to observe that the anonymity and performance of the network are both
proportional to the number of nodes and users. Under this perspective, the problem becomes
a largely economic question: how can we incentivize more relay participation?
There is a long line of research that explores various strategies for relay
incentivization dating back about a decade. Unfortunately, relay incentivization
faces several unique obstacles and no single proposal has thusfar proven to be
overwhelmingly convincing.

\textbf{Present Landscape} We argue that the lack of a breakthrough in this
niche research area is not a matter of insufficient ingenuity but rather one of
timing. Since the latest incentive schemes have been published, there has been
an explosion of academic research within the domain of cryptocurrencies. As of
this writing, Nakamoto's original Bitcoin whitepaper has garnered some 3000
citations~\cite{nakamoto2008bitcoin}. This body of work introduces a multitude
of new concepts that would appear indespensible to our area of research. Our
aim then is to leverage the full range of payment protocols that have emerged in
recent years into a practical next-generation incentivization strategy for Tor.

\textbf{Contributions} MoneTor is a novel full-stack scheme for monetary relay
incentivization. We make the following contributions in this paper:

\begin{enumerate}
\item Design an incentive scheme to favor the network diversity and offer enough market control to the Tor project to decide which notion of diversity they want to promote
\item Introduce new highly-efficient nanopayment protocols, which may be of some
  independent interest for other continuous payment applications
\item Specify an integration strategy into the existing Tor architecture and networking layer
\item Collect privacy-preserving client usage data to justify design parameters
  and argue for efficacy
\item Implement a proof-of-concept partial prototype extending the Tor protocol in the core Tor's daemon, featuring more than 15k lines of code
\item Conduct network-scale simulations on our design with Shadow~\cite{jansen2011shadow}, and criticize the performance impact of a secure and anonymous payment system embedded in a low-latency anonymity system
\end{enumerate}

The MoneTor design either leverages peer-reviewed designs or introduces fully
specified designs for all of its components. In effect, we argue that the
theoretical MoneTor framework is one that can be convincingly developed within
Tor today, from a technical standpoint.