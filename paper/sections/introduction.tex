Anonymous traffic routing through Tor remains one of the most popular low-latency methods for censorship evasion and privacy protection ~\cite{dingledine2004tor}.
In this setup, clients protect their TCP/IP metadata and content by routing their traffic through an onion-encrypted path with three randomly selected volunteer relay nodes, referred to as a circuit.
The network presently consists of $\approx 6,400$ relays contributing over 160 Gbit/s of bandwidth globally~\cite{portal2018tormetrics}.
While Tor has proven to be a highly effective option for privacy-seeking users, it suffers from two orthogonal issues that are relevant to this work.
The first is the broad family of traffic correlation attacks.
These threats are relevant in scenarios where an attacker, who controls multiple nodes or network vantage points, probabilistically occupies key roles along a single circuit, opening a much easier path to client deanonymization~\cite{wright2004predecessor,murdoch2005low}.
The second problem is the performance.
Independently of the overlay protocol itself, which generates an inherent overhead in network resources, Tor suffers from traffic congestion that leads to suboptimal network performance~\cite{portal2018tormetrics, alsabah2016performance}.

One approach to mitigate these problems is to address them separately from a networking standpoint.
Indeed, a significant portion of recent research in Tor proposes modifications to the core protocol itself, such as reengineering the TCP+TLS part of the stack~\cite{reardon2009improving} or designing a better kernel-aware scheduler~\cite{jansen2014never}.
A second approach, complementary to the first one, observes that the anonymity and performance of the network are both proportional to the number of nodes and users.
From this perspective, the problem becomes a primarily economic question: how can we incentivize more relays to participate?

To this end, we propose to deploy a flexible tokenization strategy and present moneTor: a monetary incentive scheme which allows relays to offer a \emph{premium bandwidth} product to Tor users in exchange for tokens.
Under this framework, financially willing users send payments directly to each relay along their circuits and indirectly through ``taxes'' to the Tor Project in exchange for higher internet bandwidth and faster download speeds relative to unpaid users.
Notionally, we design moneTor tokens to serve as a bona fide form of currency which satisfies standard definitions of \textit{scarcity}, \textit{fungibility}, \textit{divisibility}, \textit{durability}, and \textit{transferability}~\cite[p.3]{crump2011phenomenon}.
Our work aims to solve the nontrivial challenge of applying this strategy in the presence of the following compulsory constraints:

\begin{itemize}
\item \emph{Anonymity}: The payment scheme may not, in any way, compromise Tor's primary mission to protect user anonymity.
\item \emph{Payment Security}: Financial transactions within a semi-trusted system cannot transpire without strong guarantees of cryptographic security.
\item \emph{Efficiency}: A robust payment system must handle extremely lightweight, low-latency, and scalable payments to accommodate the dynamic and often brief activities of these clients.
  Tor services millions of concurrent users, all of whom maintain largely short-term relationships.
\end{itemize}

Importantly, we acknowledge that the addition of monetary incentives into Tor would involve delicate legal, economic, and sociopolitical considerations.
Final policy recommendations are out of scope for our work, but we do provide some discussion in Section~\ref{sec:discussion} on the risk, merits, and versatility of tokenization.
Our goal is to construct an adaptable and value-agnostic design that shows how incentives can improve Tor.

\label{sec:Contributions} \medskip \noindent \textbf{Contributions.}
This work describes a full-stack framework for tokenized Tor incentives.
Our construction, called \emph{moneTor}, addresses the challenge of applying theoretical advances in cryptocurrency research to the concrete constraints and complexities of the live Tor network.
Concretely, we introduce highly-efficient payment protocols which extend recent work to facilitate the novel concept of \emph{locally transparent nanopayment channels}.
During data exchange, our distributed payment processing procedure completely shifts expensive CPU operations off of the critical path, incurring negligible computational costs (a single hash operation) per payment.
State-of-the-art throughput is made possible by a global payment infrastructure that utilizes trustless \emph{Intermediaries} to handle the added CPU load, potentially in exchange for monetary rewards.
The moneTor scheme adheres to the standard Tor security model and conforms to our domain-specific constraints of \emph{Anonymity}, \emph{Payment Security}, and \emph{Efficiency}.

We provide a prototype of our payment layer and an extension to the existing routing protocol, resulting in approximately 15k lines of new code in C within the Tor codebase.
Our networking experiments demonstrate thousands of transactions per second without any significant issue and achieve optimal latency through engineered mechanisms like preemptive channel creation.
Finally, we discover through experimental analysis that scheduling approaches~\cite{dovrolis1999case, tang2010improved} used in previous Tor incentives systems would likely fail to offer the expected priority under the current state of the network.
Our response is to design and analyze an alternative network prioritization strategy that combines scheduling modifications with a new method based on control-flow windows.

Our Github account~\cite{monetor-github} contains all repositories needed to reproduce the research and corresponding results.

%%% Local Variables:
%%% mode: latex
%%% TeX-master: "../popets_monetor"
%%% End:
