Anonymous traffic routing through Tor remains one of the most popular
low-latency methods for censorship evasion and privacy protection
~\cite{dingledine2004tor}. In this setup, clients protect their TCP/IP metadata
and content by routing their traffic through an onion encrypted path with three
randomly selected volunteer relay nodes, referred to as a circuit. The network
presently consists of $\approx 6,400$ relays contributing over 230 Gbit/s of
bandwidth globally~\cite{portal2018tormetrics}. While Tor has proven to be a
highly effective option for privacy-seeking users, it suffers from two
orthogonal issues that are relevant to this work. The first is the broad family
of collusion attacks. These threats are relevant in scenarios where an attacker,
who controls multiple nodes or network vantage points, is probabilistically
placed in key roles along a single circuit, opening a much easier path to client
deanonymization~\cite{wright2004predecessor,murdoch2005low}. The second problem
is performance. Although the overlay protocol itself generates an inherent
overhead in network resources, Tor suffers from additional traffic congestion
that leads to suboptimal network performance~\cite{portal2018tormetrics,
  alsabah2016performance}.

One approach to mitigate these problems is to address them separately from a
networking standpoint. Indeed, a significant portion of recent research in Tor
proposes modifications to the core protocol itself, such as reengineering the
TCP+TLS part of the stack~\cite{reardon2009improving} or designing a better
kernel-aware scheduler~\cite{jansen2014never}. A second approach observes that
the anonymity and performance of the network are both proportional to the number
of nodes and users. From this perspective, the problem becomes a largely
economic question: how can we incentivize more relay participation? There is a
long line of research that explores various strategies for incentivization
spanning over the past decade. Progress in this field faces a multitude of
challenges. Consider the approach most relevant to our work: monetary
incentives. Aside from the analytically out-of-scope set of legal and
sociopolitical obstacles, monetary payments in this environment must overcome a
trifecta of challenging constraints:

\begin{itemize}
\item \emph{Anonymity}: The paramount mission of Tor is user privacy. This
  cannot be compromised or reduced by transparent money transaction trails.
\item \emph{Payment Security}: Financial transactions within an anonymity system cannot transpire
  without strong guarantees of cryptographic security.
\item \emph{Efficiency}: Tor services millions of concurrent users, all of whom
  maintain largely short-term relationships. A robust payment system must handle
  extremely lightweight and scalable payments so as to accommodate the dynamic
  and often short-lived activities of these clients.
\end{itemize}

\textbf{Present Landscape} While a number of prior works satisfy some subset of
these constraints, no single proposal has thus far been sufficiently convincing
so as to warrant further development. We speculate that the lack of a
breakthrough in this niche area is not a matter of insufficient ingenuity but
rather one of timing. In recent years, there has been an explosion of academic
research within the domain of cryptocurrencies. As of this writing, Satoshi
Nakamoto's Bitcoin whitepaper has already garnered over 4,000 references in
citing publications~\cite{nakamoto2008bitcoin}.  \footnote{The recency of this
  development is highlighted by the fact that about half of this work is dated
  after 2017.}  This body of work has sparked the development of a multitude of
new techniques that will prove indispensable to our own area of research. Our
objective then, is to leverage the full extent of these innovations into a
practical next-generation incentivization strategy for Tor.

\label{sec:Contributions}
\textbf{Contributions} The moneTor scheme is a novel full-stack framework for
Tor incentives. We make the following contributions in this paper:

\begin{enumerate}
\item Discuss economic considerations for the Tor network % to favor network
  % diversity and
  to support Tor as a public good;
  %market control such that the Tor project can decide which notion of diversity
  %to promote
\item Introduce new highly-efficient nanopayment protocols which satisfy all
  \emph{Anonymity}, \emph{Payment Security} and \emph{Efficiency}
  constraints. These protocols are % formally  verified
  analyzed and may be of some independent interest for
  other high-frequency payment applications as well;
\item Detail our integration strategy to add a payment layer
  which extends the existing Tor architecture;
  % idk I kinda liked the previous wording better.
\item Collect privacy-preserving client usage data in order to justify design parameters
  and argue for their efficacy;
\item Identify an efficient method of throttling to give priority users an
  advantage;
\item Implement a prototype of our nanopayment protocols extending the Tor protocol
  in the core C software, featuring more than 15k lines of code; %\op{Add somewhere in the paper a paragraph explaining what is in those 15k lines (everything, but crypto replaced by delays and coin management). }
\item Conduct large scale simulations to analyze the performance impact of the
  embedded payment and incentivization scheme.
\end{enumerate}

The moneTor design leverages fully-specified algorithms for some components and
well-studied existing research for all others. On a technical level, we believe
that moneTor could be feasibly developed today within a deployed system.
% previous wording was kinda weird because it implied moneTor could be used
% somewhere other than the Tor network

\paragraph*{Roadmap.} In Section~\ref{sec:background}, we draw on technical
preliminaries from two distinct fields: applied Tor research and cryptographic
payment channels. Section~\ref{sec:related_work} presents the related work. Our
contributions begin formally with Section~\ref{sec:economic}, where we first
describe a high-level economic model for the flow of money through the
network. In Section~\ref{sec:payment}, we outline the technical construction of
our nanopayment scheme at the payment protocol level. Section~\ref{sec:network}
expands the technical construction of moneTor to cover modifications at the
network level. In section~\ref{sec:analysis} we justify our design decisions
with real-world data collection from live Tor users. We continue in
Section~\ref{sec:experimentations} with a validation of our technical design,
carried out through experiments performed on a native proof-of-concept
implementation. Finally, we discuss limitations and future work in
Section~\ref{sec:limitations_futurework}, reference our source code in
Section~\ref{sec:code}, and present concluding remarks in
Section~\ref{sec:conclusion}.

%%% Local Variables:
%%% mode: latex
%%% TeX-master: "../main"
%%% End:
