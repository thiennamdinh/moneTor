Anonymous traffic routing through Tor remains one of the most popular
low-latency methods for censorship evasion and privacy protection
~\cite{dingledine2004tor}. In this setup, clients protect their TCP/IP metadata
by routing their traffic through an onion encrypted path with three
randomly selected volunteer relay nodes, referred to as a circuit. The network
presently consists of $\approx 6,400$ relays contributing over 230 Gbit/s of
bandwidth globally~\cite{portal2018tormetrics}. While Tor has proven to be a
highly effective option for privacy-seeking users, it suffers from two
orthogonal issues that are relevant to this work. The first is the broad family
of collusion attacks. These threats are relevant in scenarios where an attacker,
who controls multiple nodes or network vantage points, is probabilistically placed at more than one hop
along a single circuit, opening a much easier path to client
deanonymization~\cite{wright2004predecessor,murdoch2005low}. The second
problem is overall performance. Although the overlay protocol itself generates
an inherent overhead in network resources, Tor suffers from additional traffic
congestion that leads to suboptimal network
performance~\cite{portal2018tormetrics, alsabah2016performance}.

One approach to mitigated these problems is to address them separately from a
networking standpoint. Indeed, a significant portion of recent research in Tor
proposes modifications to the core protocol itself, such as reengineering the
TCP+TLS part of the stack~\cite{reardon2009improving} or designing a better
kernel-aware scheduler~\cite{jansen2014never}. A second approach observes that
the anonymity and performance of the network are both proportional to the number
of nodes and users. From this perspective, the problem becomes a largely
economic question: how can we incentivize more relay participation? There is a
long line of research that explores various strategies for incentivization
spanning over the past decade. Progress in this field faces a multitude of
challenges. Consider the approach most relevant to our work: monetary
incentives. Aside from the analytically intractable set of legal and
sociopolitical obstacles, monetary payments in this environment must overcome a
trifecta of challenging constraints:

\begin{itemize}
\item \emph{Anonymity}: The paramount mission of Tor is user privacy. This
  cannot be compromised or reduced by transparent money transactions trail.
\item \emph{Payment Security}: Anonymity also prohibits the formation of large
  credit and trust systems. As such, financial transactions cannot transpire
  without strong guarantees of cryptographic security.
\item \emph{Efficiency}: Tor services millions of concurrent users, all of whom
  maintain largely short-term relationships. A robust payment system must handle
  extremely lightweight and scalable payments so as to accommodate the dynamic
  and often short-lived activities of these clients.
\end{itemize}

\textbf{Present Landscape} While a number of prior works satisfy some subset of
these constraints, no single proposal has thus far been sufficiently convincing
so as to warrant further development. We speculate that the lack of a
breakthrough in this niche area is not a matter of insufficient ingenuity but
rather one of timing. In recent years, there has been an explosion of academic
research within the domain of cryptocurrencies. As of this writing, Satoshi
Nakamoto's Bitcoin whitepaper has already garnered over 3,000 references in
citing publications~\cite{nakamoto2008bitcoin}\footnote{The recency of this
  development is highlighted by the fact that about half of this work is dated
  after 2017}. This body of work introduces a multitude of new concepts that
will prove indispensable to our own area of research. Our objective then, is to
leverage the full extent of these innovations into a practical next-generation
incentivization strategy for Tor.

\label{sec:Contributions}
\textbf{Contributions} The moneTor scheme is a novel full-stack framework for
Tor incentives. We make the following contributions in this paper:

\begin{enumerate}
\item Design an economic model to favor network diversity, offering enough
  market control such that the Tor project can decide which notion of diversity
  to promote
\item Introduce new highly-efficient nanopayment protocols which may be of some
  independent interest for other high-frequency payment applications
\item Specify an integration strategy to integrate payments and user incentives
  into the existing Tor architecture and networking layer
\item Collect privacy-preserving client usage data to justify design parameters
  and argue for efficacy
\item Implement a proof-of-concept partial prototype extending the Tor protocol
  in the core daemon, featuring more than 15k lines of code
\item Conduct network-scale simulations to analyzing the performance impact of
  our embedded payment and incentivization scheme
\end{enumerate}

The moneTor design leverages peer-reviewed research for some components and
novel, fully-specified techniques in all others. In effect, we claim that the
entire technical stack is accounted for and that moneTor can be feasibly
developed today.
\paragraph*{Roadmap.} In Section~\ref{sec:background}, we draw on technical preliminaries from two distinct fields: applied Tor research and payment channels. Section~\ref{sec:related_work} presents the related work. In Section~\ref{sec:economic} we describe how we engineer the flow of money in a way that is both economically stable yet still adheres to the core mission of the Tor project. In Section~\ref{sec:payment} we 
describe the technical construction for the moneTor payment scheme at the payment protocol level. We pursue in Section~\ref{sec:network} with the technical construction of moneTor at the network level. In section~\ref{sec:analysis} we  validate our design decisions with real-world data collection from live Tor users. 
In Section~\ref{sec:experimentations}, we validate our technical design via experiments performed on a proof-of-concept software implementation within the native Tor codebase. We finally address concerncs for limitations and future work in Section~\ref{sec:limitations_futurework} and conclude in Section~\ref{sec:conclusion}.