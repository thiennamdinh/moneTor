%While we cover as much ground as possible, it has been evident from the initial
%stages that a production-ready construction of any monetary incentive scheme for
%Tor would be a monumental collaborative effort. In this final chapter, we
%discuss the promise and limitations of our results and speculate on the many
%interesting research avenues going forward.

%\subsection{Evaluation}
%\label{subsec:evaluation}
%Recall from Chapter~\ref{chap:Intro} that we are broadly concerned with three
%constraints: anonymity, economic security, and efficiency. As described in
%Chapter~\ref{chap:payment} and Chapter~\ref{chap:network}, moneTor was
%deliberately constructed to account for anonymity and economic security. We did
%not uncover any notable design flaws relating to these constraints during our
%extensive implementation phase.
%
%The results from Chapter~\ref{chap:simulation} indicate that moneTor performs
%overwhelmingly well in all efficiency measurements. This true for both global
%throughput and effective user latency where moneTor appears to achieve
%near-optimal performance in both measures. Unfortunately, we cannot produce
%meaningful comparisons with related works dealing in monetary Tor payments, as
%in all other case the researchers choose to analyze their respective algorithms
%in isolation rather than as part of a simulated Tor network. Meanwhile,
%performance comparisons with non-transferable benefit schemes and transferable
%benefits are not necessarily meaningful either, as they employ vastly different
%tokens models and should be evaluated in context. We nevertheless note that our
%decision to coordinate payments through third-party intermediaries implies that
%moneTor is conceptually more scalable than most previous proposals. A
%back-of-the-envelope calculation shows that moneTor can theoretically
%accommodate on the order of 10 million paying clients, if
%necessary.\footnote{This simple analysis conservatively assumes that the moneTor
%  ledger can handle 200 channel initiations per second (one order of magnitude
%  greater than inefficient decentralized cryptocurrencies), users are willing to
%  escrow one month's worth of payments, and users need no more than 10 concurrent
%  channels at any given time}. We are careful to acknowledge that scales
%of this this magnitude may pose any number of unseen complications which we fail
%to anticipate at the theoretical level.
%
%Finally, the paper describing LIRA is, to the best of our knowledge, the most
%recent related work that explores similar concepts of traffic prioritization in
%Tor. While it is difficult to make definitive claims comparing different
%networking simulations, we observe that our flow-control method achieves similar
%levels of traffic differentiation for the median user at lower traffic
%load~\cite{jansen2013lira}.

\subsection{Limitations}
\label{subsec:limitations}

The moneTor framework faces a number of technical and economic
limitations. Firstly, the mere presence of paid traffic leaks at least a single
bit of information to relays and intermediaries regarding the browsing habit of
the user. In practice, we expect that this information might be positively
correlated with certain user applications such as bittorrent or video
streaming. Secondly, our nanopayment protocols leak the number and value of
payments made to the intermediary. While this information cannot by itself be
used to identify clients or relays, we expect that it could provide useful
contextual knowledge when combined with other inputs. Finally, although we
ensure payment privacy within the Tor ecosystem, we can make no such guarantees
at the token exchange interface. Users who practice inadequate operational
security risk leaking information about the net value of payments they make in
Tor with their real-world identities. This problem may be of greatest concern for
users residing in political regimes that are politically opposed to the access
of Tor altogether.

A separate financial limitation arises from our our tripartite payment channel
architecture and the need for intermediaries to escrow large amounts of
funds. This opportunity cost of the frozen capital in terms of forgone
investments will be passed on as financial overhead to paying users in the form
of intermediary fees.

\subsection{Future Work}
\label{subsec:future_work}

\paragraph*{Prioritized Traffic} Although we were able to successfully demonstrate
clear prioritized premium traffic using flow control windows, there is
considerable room to analyze prioritization under more realistic settings. Among
the possibilities, we maintain that scheduling provides the most conceptually
elegant way to achieve our goals. In the absence of effective scheduling,
extension and further analysis of our flow-control technique provides an
alternative path forward. This future work might study methods for ensuring a
more robust response to local variations in the fraction of premium users as
well total network load.

\paragraph*{Side-channel Attacks} We assumed that our protocol construction would
not give more information to an observer than the single bit needed to
differentiate a premium user. However in practice, the protocol may leak
information through side-channels, such as the time needed to exchange cells in
order to open/close channels. An intermediary, or an exit relay may reason about
the client's location by evaluating the round-trip-times of the multi-rounds
protocol. Although often easily mitigated once discovered, these side channels
pose an interesting problem to our expanded attack surface.

\paragraph*{Tax Redistribution and Adversary Reactions}
The understanding of the right policies for tax redistribution is an ongoing
research question and is out of scope of this paper. Further, likely
interdisciplinary, research might seek to discover the best incentivization
strategies for desired user behaviors and while mitigating adversarial abuse.

\paragraph*{Channel Management} Given the dynamic activity of payment
channels that we have shown, further research might explore more sophisticated
policies to manage moneTor payment channels. Such policies would make decisions
concerning micropayment channel initial value, nanopayment channel size,
preemptive channel creation, and the selection of available channels to apply to
each new circuit. An ideal policy would be responsive to network load and
predicted user behavior. In addition to the classical optimization objectives of
computation, memory, and bandwidth, this research effectively considers a brand
new vector as it attempts to streamline the flow of money through the network.

\paragraph*{Extended Applications} Existing privacy-centric currencies are
limited in that they are only effective when used in conjunction with an
anonymizing network such as Tor~\cite{sasson2014zerocash}. Our currency scheme
more explicitly integrates the two layers and. This, combined with the featured
efficiency and scalability in particular makes moneTor an attractive option for
such applications as metered video viewing, metered video gaming, and
pay-per-page websites. The study and implementation of these applications in an
anonymous setting provides another interesting opportunity for multidisciplinary
research.
