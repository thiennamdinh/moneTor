In this final discussion, we attempt to frame the social nuances of incentivization that accompany our technical proposal.
Today, the Tor network owes its success to voluntary contributors.
These relay operators incur hardware, bandwidth, and labor expenses to participate in the Tor network for predominantly intrinsic reasons, among them political, philosophical, and philanthropic.
In implementing a system like moneTor, the Tor Project must consider the consequences of financial incentives on these potentially fragile value systems.
Critics have raised several questions concerning the social consequences of incentives~\cite{jansenblogpost}.
For instance, the ``crowding out effect'', describes a psychological mechanism whereby the introduction of extrinsic motivations displaces previously dependable intrinsic motivations~\cite{10.1257/jep.25.4.191}.
At the sociological level, empirical studies on prosocial behaviour~\cite{10.1257/aer.96.5.1652} have shown that explicit incentives can reduce participation.
This phenomenon suggests that introducing incentives would risk degrading the average social quality of Tor nodes without necessarily growing the network.
Nevertheless, the literature does not reject the natural intuition that incentives at least have the potential to be effective given careful consideration to the nuances of multi-faceted motivational factors~\cite{10.1257/aer.96.5.1652}.
Our work does not presume to offer an authoritative opinion of the best social incentive design.
Instead, we show that the versatility of our token-centric technical design is sufficient to support a wide diversity of potential strategies.

By design, we have left many variables to be decided by the Tor Project.
Chief among these are the premium bandwidth price, premium bandwidth advantage, tax rate, monetary policy, and redistribution policy.
Enforcement mechanisms include a combination of network monitoring, parameter broadcast, ledger constraints, and organizational policy.
Together, these options give the Tor Project the freedom to implement a large variety of incentivization strategies.
We present a few theoretical paradigms below.

\begin{enumerate}

\item \textbf{Benevolent Ruler}: The Tor Project is a nonprofit with altruistic organizational intentions.
  In this hypothetical approach, the authorities set the tax rate to 100\%, turning moneTor into a revenue stream for purely centralized efforts to improve the network.

\item \textbf{Crypto-Libertarian}: On the other end of the spectrum, a simple na\"{\i}ve strategy is to implement moneTor as unregulated market where relays sell bandwidth for money, perhaps trustlessly backed by a cryptocurrency like Bitcoin or Ethereum.

\item \textbf{In-Kind Rewards}: If financial rewards turn out to be prohibitive, we can mimic several of the purely in-kind non-transferable and transferable schemes described in Section~\ref{sec:related_work} by limiting the number of on-ledger transactions to zero or one, respectively.
  For instance, at zero allowed transactions, relays would be unable to use their tokens for any purpose other than to buy premium traffic from other relays.

\item \textbf{Subsistence Relaying}: Here, although relays would receive real money, we would limit the expected value rewards to the break-even cost of running a relay.
In this paradigm, every relay operator would effectively act as an individual NGO to avoid the potential liabilities that come with financial profit.
In practice, moneTor could support this design by adjusting global parameters such that the incomes of any relay does not exceed the cost-to-bandwidth ratio in the cheapest geographic region.
The tax redistribution policy could cover any discrepancy from this income floor due to location or other factors.
Previously, studies have indicated that incentives can be useful when sparingly applied to ``concrete'' task~\cite{10.1257/jep.25.4.191, 10.1086/431263}.
Provided careful public communication of intent, we might expect the same results here.

\item \textbf{Decentralized Grant-making}: Finally, the Tor Project can implement moneTor as a form of \emph{altruistic money}, which payees can donate an approved list of prosocial projects, reminiscent of participatory budgeting in local governments~\cite{cabannes2004participatory}.
Rather than cashing out tokens, the ledger would only allow relays to contribute to their choice of special programs that, for instance, increase the number of relays in region $X$, fund research on topic $Y$, or advocate for privacy-related issue $Z$.
The result is a fully fungible asset that is extrinsically worthless to profit-driven entities but intrinsically rewarding for parties invested in the core mission of Tor.

\end{enumerate}

Much future work remains to implement some of these methods.
On the technical side, robust monitoring must be implemented to mitigate inflated bandwidth for tax redistribution and junk traffic insertion for direct payments.
Economically, we require a better understanding of topics such as location-based incentives and the elasticity of supply and demand for various types of relay incentives and client premium bandwidth, respectively.

The purpose of these extreme paradigms is to illustrate the versatility moneTor as a technical infrastructure.
In practice, the Tor Project can implement nearly any combination of our listed approaches, and many others as well.
The mindset we wish to instill is this: an overly flexible design can always be constrained afterward.
Until we have a consensus for the right social approach to incentivization, the most useful technical base is one that can adapt to many models of human behavior.

%%% Local Variables:
%%% mode: latex
%%% TeX-master: "../popets_monetor"
%%% End:
