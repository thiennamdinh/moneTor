Like any network, Tor network can benefit from improvements in both performance and diversifying its participants.
While the idea of introducing incentive mechanisms for relay operators has been studied for almost 10 years, existing implementations leave numerous questions open.

Our scheme, moneTor, overcomes several technical limitations in the previous proposals.
We developed new approaches and also integrated recent contributions from payment technologies within the Tor architecture.
For example, moneTor supports highly efficient anonymous transactions in the absence of a trusted third party, is considerably more scalable than previous works, and offers integrated methods that would support the implementation of a broad range of future incentive policies through its token-centric approach and taxation mechanism.
We tested the moneTor design through extensive simulation efforts to show the technical efficacy of our solution.
During this process, we pivoted to challenge and document the inner working of Tor's packet scheduling mechanism.

An essential set of questions remain open concerning on the legal, political, and social aspects of introducing monetary mechanisms into Tor.
We certainly do not dismiss the importance of these issues, which existing literature have discussed at length~\cite{jansenblogpost}.
Nevertheless, we believe that gaining a good understanding of what \emph{can} be done, in terms of its technical impact on networking and security, is a central part of the debate on what \emph{should} be done.

%%% Local Variables:
%%% mode: latex
%%% TeX-master: "../popets_monetor"
%%% End:
