The Tor network can, as any network, benefit from improvements in both performance and diversity in the positioning of participants.
While the idea of introducing incentive mechanisms for relay operators has been recognized and studied for almost 10 years, existing implementations leave numerous questions open.

Our scheme makes several important steps to overcome technical limitations identified in the previous proposals.
We developed new approaches and also integrated recent contributions from payment technologies within the Tor architecture.
As a result, moneTor supports highly efficient anonymous transactions in the absence of trusted third party, is considerably more scalable than previous proposals and offers integrated methods that would support the implementation of a broad range of future incentive policies through its token-centric approach and taxation mechanism.
We tested the moneTor design through extensive simulation efforts which showed the technical efficacy of our solution and, along our way, were led to challenge and document the inner working of Tor's packet scheduling mechanism.

An important set of open questions, which can hardly be addressed in a laboratory setting, focuses on the legal, political and social aspects of introducing monetary mechnisms into Tor.
We certainly do not underrate the importance of these issues, which have been discussed at length in the literature~\cite{jansenblogpost}.
Nevertheless, we believe that gaining a good understanding of what \emph{can} be done, in terms of its technical impact on networking and security, is a central part of the debate on what \emph{should} be done.

%%% Local Variables:
%%% mode: latex
%%% TeX-master: "../popets_monetor"
%%% End:
