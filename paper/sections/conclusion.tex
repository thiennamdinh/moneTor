The Tor network suffers from concerns in both performance and diversity. In
accordance with the general law that large networks are better than small
networks, we assert that the quality of Tor along both of these vectors can be
simultaneously improved by incentivizing more relay participation, while
promoting some diversity notion. To this end, we present moneTor, a
comprehensive framework for incentivizing Tor relay participation through true
monetary payments. Our design broadly covers every major level of implementation
from economic policies to the protocol payment layer and integration into the
Tor networking architecture. Along the way, we developed novel protocols for
highly efficient and cryptographically secure nanopayment channels as well as
novel techniques in networking integration. A small venture into empirical data
collection reinforced our intuition that existing user behavior is compatible
with a light-weight payment incentive scheme. This led to our more involved
round of experimentation in which we tested our natively integrated moneTor
prototype. The results of this investigation were highly encouraging, indicating
low latencies, negligible throughput overhead, and upwards of 100\% benefits for
paying users in the simulated environment.

Legal, political, and sociological questions concerning the prudency of
introducing money into the Tor network are difficult to answer in a laboratory
setting. However, this work indicates that our solution to Tor incentivizaiton
offers promising properties on all measured technical and economic
features. Pending further refinement, we optimistcally conclude that moneTor is
the first scalable monetary payment framework for Tor that can be feasibly
developed today.