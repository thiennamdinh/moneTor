The Tor network suffers from concerns in both performance and
diversity. In accordance with the general law that large networks are better than
small networks, we assert that the quality of Tor along both of these vectors
can be simultaneously improved by incentivizing more relay participation, while promoting some diversity notion. To this end, we
present moneTor, a comprehensive framework for incentivizing Tor relay
participation through true monetary payments. Our design broadly covers every
major level of implementation from general economic model to the protocol
payment layer and integration into the Tor networking architecture. Along the
way, we developed novel protocols for highly efficient and cryptographically
secure nanopayment channels as well as novel techniques in networking
integration.

A small venture into empirical data collection reinforced our intuition that
existing user behavior is compatible with a light-weight payment incentive
scheme. This led to our more involved round of experimentation in which we
tested our natively integrated moneTor prototype. The results of this
investigation were highly encouraging, indicating low latencies, negligible
throughput overhead, and upwards of 100\%-200\% benefits for paying users in the
simulated environment.

Legal, political, and sociological questions concerning the prudency of
introducing money into the Tor network are difficult to answer in a laboratory
setting. However, our work indicates that moneTor is feasible from a
technical and economical perspective.

We optimistically conclude that the built-in efficiency and scalability in particular makes our tripartite micro+nanopayment layers an attractive option for other usage than Tor bandwidth prioritization, such as metered video viewing, metered video gaming, and pay-per-page websites.
%moneTor an attractive option
%We optimistically conclude with an
%overarching statement that moneTor is the first true monetary incentive scheme for
%Tor which is ready for serious production-level development today.
