
The Tor network can, as any network, benefit from improvements in both
performance and diversity in the positioning of participants. While the idea of
introducing incentive mechanisms for relay operators has been recognized and
studied for almost 10 years, existing implementations leave numerous questions
open.

Our scheme makes several important steps to overcome technical limitations
identified in the previous proposals. We developed new approaches and also
integrated recent contributions from payment technologies within the Tor
architecture. As a result, moneTor supports highly efficient anonymous
transactions in the absence of trusted third party, is considerably more
scalable than previous proposals and offers integrated methods that would
support the implementation of a broad range of future incentive policies through
its token-centric approach and taxation mechanism. We tested the moneTor design
through extensive simulation efforts which showed the technical efficacy of our
solution and, along our way, were led to challenge and document the inner
working of Tor's packet scheduling mechanism.

An important set of open questions, which can hardly be addressed in a
laboratory setting, focuses on the legal, political and social aspects of
introducing monetary mechnisms into Tor. We certainly do not underrate the
importance of these issues, which have been discussed at length in the
literature~\cite{jansenblogpost}. Nevertheless, we believe that gaining a good
understanding of what \emph{can} be done, in terms of its technical impact on
networking and security, is a central part of the debate on what \emph{should}
be done.



% \flo{Maybe merging what follows with the conclusion} Introducing a monetary
% system may have a social impact on relay operators and Tor
% users~\cite{jansenblogpost}. This social impact may also depends on the
% technical properties and user-convenience of the used monetary scheme (e.g.,
% the social impact of client cheating should not be ignored). We showed in this
% paper that technical progresses are still possible and that successfully
% offering priority requires more than only using a local scheduler on each
% relay. Better technical achievements could likely influence the social impact.
% In summary, the barriers to deployment of a payment system are first a
% community-wide agreement on good technical properties, and then the social
% impact barrier. We progress on the first barrier, offering a new payment
% system with no client cheating incentives, anonymous, efficient, distributed,
% scalable with trustless entities and with secure ownership of funds.

% The Tor network suffers from concerns in both performance and diversity. In
% accordance with the general law that large networks are better than small
% networks, we assert that the quality of Tor along both of these vectors can be
% simultaneously improved by incentivizing more relay participation, while
% promoting some diversity notion. To this end, we present moneTor, a
% comprehensive framework for incentivizing Tor relay participation through true
% monetary payments. We make contributions to each broadly technical component
% of the stack including economic policies, payment transactions, and
% integration into the Tor networking architecture. Along the way, we developed
% novel protocols for highly efficient and cryptographically secure nanopayment
% channels as well as novel techniques in networking design. A small venture
% into empirical data collection reinforced our intuition that existing user
% behavior is compatible with a light-weight payment incentive scheme. This led
% to our more involved round of experimentation in which we tested our natively
% integrated moneTor prototype. The results of this investigation were highly
% encouraging, indicating low latencies, negligible throughput overhead, and
% upwards of 100\% benefits for paying users in the simulated environment.

% Legal, political, and sociological questions concerning the prudency of
% introducing money into the Tor network are difficult to answer in a laboratory
% setting. However, this work indicates that our solution to Tor incentivizaiton
% offers promising properties on all measured technical and economic features.
% Pending further refinement, we optimistcally conclude that moneTor is the
% first scalable monetary payment framework for Tor that can be feasibly
% developed today.

%%% Local Variables:
%%% mode: latex
%%% TeX-master: "../popets_monetor"
%%% End:
