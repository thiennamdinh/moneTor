This work introduces moneTor, a practical system that uses tokenized payments to incentivize participation in the Tor network.
Like any network, Tor can benefit from improving its performance and diversifying its participants.
While Tor incentivization has been studied for almost 10 years, existing implementations have many technical limitations.

Our work directly addresses these shortcomings.
At the payment level, we leveraged recent advances from the cryptocurrency research space to inform the design of our payment system.
At the network level, we investigated a critical issue in the existing scheduling mechanism and introduce a new flow-control-based approach for network prioritization.
The resulting scheme features sufficiently anonymous transactions in the absence of a trusted third party and considerably greater scalability.
We tested moneTor through extensive simulation efforts to demonstrate its technical feasibility.

An essential set of questions remain concerning the legal, political, and social aspects of incentives for Tor.
We certainly do not dismiss the importance of these issues, which existing literature have discussed at length~\cite{jansenblogpost}.
Nevertheless, the ability to show what \emph{can} be done, in terms of its technical impact on networking and security, is a central step toward determining what \emph{should} be done.

%%% Local Variables:
%%% mode: latex
%%% TeX-master: "../popets_monetor"
%%% End:
