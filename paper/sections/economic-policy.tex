We first summarize the economic layer of the incentivization scheme. At this
level, we assume the existence of an ideally secure and efficient payment layer
and proceed to outline the high-level policy design.

\subsection{Currency Design}

The moneTor framework proposes a native payment layer for the Tor ecosystem. To
qualify as a true monetary payment design, our payment tokens should satisfy the
standard properties of \textit{scarcity}, \textit{fungibility},
\textit{divisibility}, \textit{durability}, and
\textit{transferability}~\cite[p.3]{crump2011phenomenon} Furthermore, we follow
the Bitcoin paradigm in which all users maintain full and exclusive control of
their monetary wealth through use of public key cryptography.

Tokens must accrue some form of intrinsic value and one option is to launch the
moneTor token with an ICO maintained by decentralized mining.\footnote{Initial
  Coin Offering} There are two drawbacks to this approach.

\begin{enumerate}
\item Decentralized blockchains are inefficient
\item New currencies introduce undue economic complexity
\end{enumerate}

The efficiency of decentralized blockchains stems from the necessity to reach
agreement among an unbounded number of nodes through expensive consensus
mechanisms. Fortunately, the Tor network benefits in some ways from a more
centralized infrastructure comprised from the list of trusted authorities. We
propose to take advantage of this setup by introducing a new set of ledger
authorities who are responsible for maintaining publicly audited global
currency data.

To avoid the economic and financial nuances associated with a new currency, we
suggest a two-way value peg to fix moneTor tokens to a high-market cap
cryptocurrency such as Bitcoin or Ethereum~\cite{back2014enabling,
  poon2017plasma}. We recommend engineering the system under the Ethereum Plasma
model~\cite{poon2017plasma}. Aside from the efficiency improvements, it would
likely be feasible in the Plasma design to integrate smart contracts such that
the Ethereum blockchain becomes the authority ledger of last resort in the event
that the native Tor ledger goes offline. Effectively, this means we guarantee
that honest users will never completely lose access to their money so long as
the Ethereum blockchain is accessible.

Finally, moneTor requires a base layer payment protocol to anonymously move
large sums of money. To do this, we suggested either
Zerocash~\cite{sasson2014zerocash} or Cryptonote~\cite{van2013cryptonote}.

\subsection{Incentive Model}
The moneTor framework follows the same philosophy as prior incentivization
proposals by offering a paid \emph{premium} bandwidth product to Tor
users. Under this scheme, financially willing users send direct payments to each
relay along their circuits in exchange for higher internet bandwidth relative to
unpaid users. While we cannot strictly enforce that all relays will correctly
following the protocol in a decentralized network, the client can monitor her
own bandwidth and only make payments when it appears to be beneficial. This
setup can be viewed from a game theoretic tit-for-tat relationship in which the
relay has no apparent incentive to deviate.

In our setup, a key economic question to address is the issue of price
determination. While it would be tempting to enlist any number of market-based
mechanisms to set premium bandwidth prices, any price differentiation between
clients or relays inevitably leaks information. We therefore impose the
constraint that all users should pay a single uniform price for premium
bandwidth at any time $t$. This price may be centralized calculation by the
authorities or a more dynamical consensus vote reached by the network.

\textbf{Death and Taxes} A more convoluted problem is the issue of wealth
distribution. Given that Tor is a fundamentally nonprofit ecosystem, it is not
quite clear that an optimal profit-seeking network will closely aligned with the
core mission of the Tor Project. We address this problem with the introduction
of a taxation element. From every payment, a percentage of funds is anonymously
diverted to an account controlled by network authories and transparently
redistributed to relays or even other users with desirable properties. In
essence, taxation provides a tunable mechanism for the Tor Project to shape the
topology of the network towards better diversity and performance.
