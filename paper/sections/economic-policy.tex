We first summarize the economic layer of the incentivization scheme. At this
level, we assume the existence of an ideally secure and efficient payment layer
and proceed to outline the high-level design strategy.

\subsection{Currency Design}

The moneTor framework describes native currency for the Tor ecosystem. To
qualify as a true monetary payment design, our currency should satisfy the
standard properties of \textit{scarcity}, \textit{fungibility},
\textit{divisibility}, \textit{durabililty}, and
\textit{transferability}~\cite[p. 3]{crump2011phenomenon} Furthermore, we follow
the Bitcoin paradigm in which all users maintain full and exclusive control of
their monetary wealth through use of public key cryptography.

A natural initial choice might be to introduce a new token modeled after the
growing number of independent cryptocurrencies. This approach has two drawbacks:

\begin{enumerate}
\item Decentralized blockchains are inefficient
\item New currencies introduce undue economic complexity
\end{enumerate}

The efficiency of decentralized blockchains stems from the necessity to reach
agreement among an unbounded number of nodes through expensive consensus
mechanisms. Fortunately, the Tor network uses a more centralized infrastructure
that currently features nine partially trusted directory authorities. We propose
to take advantage of this setup by introducing a set of ledger authorities who are
responsible for maintaining publically audited global currency data.

To avoid the economic and financial nuances associated with a new currency, we
suggest a two-way value peg to fix moneTor tokens to a high-market cap
cryptocurrency such as Bitcoin or Ethereum. While a designed rooted on Bitcoin
sidechains~\cite{back2014enabling} is acceptable, we recommend engineering the
system under the Ethereum Plasma model~\cite{poon2017plasma}. Aside from the
efficiency improvements, it would likely be feasible in the Plasma design to
integrate smart contracts such that the Ethereum blockchain becomes the
authority ledger of last resort in the event that the native Tor ledger goes
offline. Effectively, this means we guarantee that users will never completely
lose honestly obtained money so long as the Ethereum blockchain is accessible.

Finally, we would like to provide user privacy as much as possible before and
after relay payments are made. To do this, we suggested either
Zerocash~\cite{sasson2014zerocash} or Cryptonote~\cite{van2013cryptonote} as
base layer payment protocols for channel escrow and other large transactions.

\subsection{Incentive Model}
The moneTor framework follows the same philosophy as prior incentivization
proposals by offering a choice of paid \emph{premium} bandwidth to Tor
users. Under this scheme, financially willing users send direct payments to each
relay along their circuits to purchase priority status in the traffic scheduling
algorithm. While there is no way to enforce priority scheduling in a
decentralized network, the setup is secure in a game-theoretic model by virtue
of a multi-round tit-for-tat relationship between the client and the relay.

In our setup, a key economic question to address is the issue of price
determination. While it would be tempting to enlist any number of market-based
mechanisms to set premium bandwidth prices, any price differentiation between
clients or relays inevitably leaks information. We therefore impose the
constraint that all users should pay a single uniform price for premium
bandwidth at any time $t$. This price may be obtained through any number of
transparent mechanisms ranging from a centralized market determination by the
Tor Project to a more dynamical consensus vote reached by the network. For example, Tor relays may agreed on a global unique price per MB of bandwidth in order to out-compete self called ``anonymous" VPNs claiming better performance.


\textbf{Death and Taxes} A more convoluted problem is the issue of wealth
distribution. Given that Tor is a fundamentally nonprofit ecosystem, it is not
quite clear that an optimal profit-seeking network will closely aligned with the
core mission of the Tor Project. To address this problem, we introduce a
taxation element into the design. From every payment, a percentage of funds is
anonymously diverted to an account controlled by the network authority. The
intent is that these funds can be algorithmically redistributed to relays in
desirable locations or roles, sustaining desirable notitons of network diversity
as defined and executed by the Tor project.
