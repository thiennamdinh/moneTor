We first summarize the economic layer of the incentivization scheme. At this
level, we assume the existence of an ideally secure and efficient payment layer
and instead outline the high-level design strategy.

\subsection{Currency Design}

MoneTor proposes to implement a native currency to the Tor ecosystem. Unlike in
many previous Tor proposals thats feature ``bulkier'' notions of incentives and
priviledges, our currency should satisfy the standard properties of
\textit{scarcity}, \textit{fungibility}, \textit{divisibility},
\textit{durabililty}, and \textit{transferability}. \flo{We need proper
  definitions of these properties. In background section or here?}\td{These
  properties are pretty standard as far as I know so better to find a good
  reference... did some initial search but nothing came up yet} Furthermore, we
follow the Bitcoin paradigm in which all users maintain full and exclusive
control of their monetary wealth through use of public key cryptography.

A natural initial choice might be to introduce a new token modeled after the
growing number of independent cryptocurrencies. This approach has two drawbacks:

\begin{enumerate}
\item Decentralized blockchains are inefficient
\item Standalone altcoin introduces undue economic complexity
\end{enumerate}

The efficiency of decentralized blockchains stems from the necessity to reach
consensus among an unbounded number of nodes through expensive proof-of-work
mining or proof-of-stake. Fortunately, the Tor network uses a more centralized
infrastructure that currently features nine directory authorities. We propose to
make use of this model by introducing a set of ledger authorities who are
responsible for maintaining publically audited global currency data.

To avoid the economic and financial nuisances of the second problem, we suggest
a two-way value peg to fix MoneTor tokens to a high-market cap cryptocurrency
such as Bitcoin or Ethereum. Two-way pegs were originally proposed for Bitcoin
as ``pegged sidechains'' and further developed for more flexible use cases as
``plasma child chains'' in Ethereum~\cite{back2014enabling, poon2017plasma}. In
the Ethereum design, it would likely be feasible to integrate smart contracts
such that the Ethereum blockchain can be used as an authority ledger of last
resort in the event that the native Tor ledger goes offline.

Finally, we would like to provide user privacy as much as possible before and
after relay payments are made. To do this, we suggested a transactional payment
layer constructed from existing zk-SNARK or ring signature based anonymous
payment schemes~\cite{sasson2014zerocash, van2013cryptonote}.

\subsection{Incentive Model}
MoneTor follows the same philosophy as prior incentivization proposals by
offering a choice of paid \emph{premium} bandwidth to Tor users. Under this
scheme, financially willing users send direct payments to each relay along their
circuits to purchase priority status in the traffic scheduling
algorithm. While there is no way to enforce relays to honor this priority
scheduling, the setup is secure in a game-theoretic model by virtue of the
multi-payment tit-for-tat relationship between the client and the relay.

In our setup, a key economic question to address is the issue of price
determination. While it would be tempting to enlist any number of market-based
mechanisms to set premium bandwidth prices, any price differentiation between
clients or relays effectively leaks information. We therefore impose the
constraint that all users should pay a single uniform price for
premium bandwidth at any time $t$. This price may be obtained by any number of
transparent mechanisms ranging from a centralized market determination by the
Tor Project to a more dynamical consensus vote reached by the network.

A more convoluted problem is the issue of wealth distribution. Given that Tor is
fundamentally a nonprofit ecosystem, it is not quite clear that an optimal
profit-seeking network will closely aligned with the core mission of the Tor
Project. To address this problem, we introduce a taxation element into the
design. From every payment, a percentage of funds is anonymously diverted to an
account controlled by the network authority. The intent is that these funds can
be algorithmically redistributed to relays in desirable locations or roles to sustain the network diversity, with a notion of diversity decided by the Tor project.