
% Discuss first the problem
% Discussing conservative deployment of moneTor with Tax collection
% Discuss impact on usability, size and distribution of Tor users if only available for pay 
% Discuss incentives to game the system
% Discuss Liability impact to run relays
% Discuss auditing, taxation and payment distribution?
% Careful about offering more money for exits
% Comparing Tor incentives with incentives for studenship
% Big difference between providing premium service -- and setting % the relay in the network in a good place

Tor is a distributed network based on volunteered contribution: 
physical machines are bought or rented by people all around the 
world and connected through the Tor routing protocol. Those 
people are taking time and spending part of their income 
contributing to Tor for various kind of reasons, among 
philosophical, political and philantropical intrinsic 
motivations. That is, relay operators do not expect to make money 
from their contributions. An important consideration to have is 
to maintain the strength of intrinsic motivations of the current 
community and to prevent monetary rewards to harm it.

Several questions~\cite{jansenblogpost} were raised covering 
potential social issues from previous attempts to design incentives, 
and among them the ``crowding out'' effect~\cite{,}, showing how 
explicit incentives such as monetary rewards can lead to a decrease 
of intrinsic motivations. Social experimentation in the litterature 
covering prosocial behaviours~\cite{} has shown that rewarding the 
behaviour could backfire and decrease the amount of participants to the 
task instead of the expected increase, due to psychological effects on 
self-representation, on the image we want to send to others, or to 
mismatching agreement on the reward. 

Rewards are usually linked to profit or to compensation. However, monetary incentives does not necessarly need to be about that. A 
conservative approach could be taken by not rewarding relays for relaying 
bandwidth, but by offering an
in-kind support -- that is, the received money cannot exceed the cost of 
running relays, and in the best case, it would support the estimated 
cost of the infrastructure. The received money would not compensate the 
cost of running the infrastructure, which is fueled by the operator's intrinsic 
motivations.  In practice, moneTor could support this design by ajusting the 
global price for direct payment such that the direct payment incomes of the 
cheapest place in the Internet to run relays does not exceed the cost of running a relay at this place.