
% Discuss first the problem
% Discussing conservative deployment of moneTor with Tax collection
% Discuss impact on usability, size and distribution of Tor users if only available for pay 
% Discuss incentives to game the system
% Discuss Liability impact to run relays
% Discuss auditing, taxation and payment distribution?
% Careful about offering more money for exits
% Comparing Tor incentives with incentives for studenship
% Big difference between providing premium service -- and setting % the relay in the network in a good place

Tor is a distributed network based on volunteered contribution: physical
machines are bought or rented by people all around the world and connected
through the Tor routing protocol. Those people are taking time and spending part
of their income contributing to Tor for various kind of reasons, among
philosophical, political and philantropical intrinsic motivations. That is,
relay operators do not expect to make money from their contributions. An
important consideration to have is to maintain the strength of intrinsic
motivations of the current community and to prevent monetary rewards to harm it.
In this spirit, several questions~\cite{jansenblogpost} were raised covering
potential social issues from previous attempts to design incentives, and among
them the ``crowding out'' effect~\cite{10.1257/jep.25.4.191}, showing how
explicit incentives such as monetary rewards can lead to a decrease of intrinsic
motivations. Social experimentation in the litterature covering prosocial
behaviours~\cite{10.1257/aer.96.5.1652} has shown that rewarding the behaviour
could backfire and decrease the amount of participants to the task instead of
the expected increase, due to psychological effects on self-representation, on
the image we want to send to others, or to mismatching expectation on what the
reward should be.

Rewards in the prosocial studies were usually linked to profit or to
compensation. However, monetary incentives does not necessarly need to be about
that. A conservative approach could be taken by not rewarding relays for
relaying bandwidth, but by offering an in-kind support with moneTor tokens --
that is, the expected value of the received tokens cannot exceed the cost of
running relays, and is expected to support the estimated cost of the relay. In
this reasoning, every relay operator could act as nonprofit and not suffer from
potential liability of what running relays for profit could create. The received
money would not compensate the cost of running the infrastructure, which is
fueled by the operator's intrinsic motivations. In practice, moneTor could
support this design by ajusting the global price for direct payments such that
the direct incomes of the cheapest place in the Internet to run relays does not
exceed the cost of running a relay at this place, also weakening the incentives
of exit relays to inject junk traffic. Then, the tax redistribution could be a
function accounting for relay location and relay bandwidth that would try to
reimburse the marginal cost of running the relays in the network. Many more
questions are tied to tax redistribution, such as preventing operators to game
the system (security), evaluating the right price for a given location
(economic) or protecting the trust relationship~\cite{10.1257/aer.96.5.1611}
between Tor authorities and relays (social). Could the tax redistribution be a
community decision? Answering those questions would yield interesting further
work, with the goal to incentive operators with intrinsic motivations to deploy
more relays, and potentially at more diverse locations (usually more costly). It
has been shown in incentives for Education~\cite{10.1257/jep.25.4.191,
10.1086/431263}, that when some support was offered for a concrete task (e.g.,
course enrollment), the social experiment was showing positive results. We
expect that offering support for operators with intrinsic motivation to run
relays should yield similar results than offering support to student with
intrinsic motivation to complete studies.

Applying an in-kind support should not mix operators attracted by extrinsic
values with operators holding intrinsic motivation, yet this remains to be
verfied in a concrete experimentation. However, it could be interesting to
motivate the Intermediary role with extrinsic value. The Tor network would
require in practice a limited amount of Intermediaries, but they must offer
strong availability to maintain the micropayment channel states for all peers
and enable robustness in the payment layer. Running Intermediary nodes requires
the operator to have more CS abilities than operators running classical Tor
relays. It is likely that observing, understanding and helping the Tor project
with technical limitations of the implementation would be an important task for
a limited amount of operators handling a new role within the network. However,
an Intermediary does not require a trustworthy operator, but someone
incentivized to provide good services. Those intermediaries are by design not
tasked to relay user streams, hence those operators are not to be mixed with the
relay operator community.

