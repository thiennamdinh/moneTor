
% Discuss first the problem
% Discussing conservative deployment of moneTor with Tax collection
% Discuss impact on usability, size and distribution of Tor users if only available for pay 
% Discuss incentives to game the system
% Discuss Liability impact to run relays
% Discuss auditing, taxation and payment distribution?
% Careful about offering more money for exits
% Comparing Tor incentives with incentives for studenship
% Big difference between providing premium service -- and setting % the relay in the network in a good place

In this final discussion, we attempt to frame the social nuances of
incentivization describe the applicablity of our technical framework. Today, the
Tor network owes its success to voluntary contributors. These relay operators
incur tangible hardware and labor expenses to participate in the Tor network for
largely intrinsic reasons, among them political, philosophical, and
philanthropic. In implementing a system like moneTor, the Tor Project must
consider the consequences of financial incentives on these these potentially
fragile value systems. Several critical questions have been raised concerning
the social design of incentives~\cite{jansenblogpost}. For instance, the
``crowding out effect'', describes a psychological mechanism whereby the
introduction of extrinsic motivations displaces previously dependable intrinsic
motivations~\cite{10.1257/jep.25.4.191}. At the sociological level, empirical
studies on prosocial behaviour~\cite{10.1257/aer.96.5.1652} have shown that
explicit incentives can have an adverse effect on participation. This phenomena
suggests that introducing incentives would risk of degrading the average social
quality of Tor nodes without necessarily growing the network. Nevertheless, the
literature does not reject the natural intuition that incentives at least have
the potential to be effective when careful consideration is given to the nuances
of multi-faceted motivational factors~\cite{10.1257/aer.96.5.1652}. Our work
does not presume to offer an authoritative opinion of the best social incentive
design. Instead, we show that the versatility of our token-centric technical
design is sufficient to support a wide diversity of potential strategies.

By design, we have left many variables to be decided either directly or
indirectly by the Tor Project. Chief among these are the premium bandwidth
price, premium bandwidth advantage, tax rate, monetary policy, and
redistribution policy. Methods to enforce these choices involve network
monitoring, network parameter broadcasting, ledger constraints, and
organizational policy. Together, these components comprise a hypothetical
control panel for the Tor Project to implement one of an essentially limitless
array of incentivzation options, a few of which are described below.

\begin{enumerate}

\item \textbf{Crypto-Libertarian}: On one extreme, perhaps the simplest naive
approach is to implement moneTor as perfectly unregulated market where bandwidth
is exchanged for money, perhaps trustlessly backed by a cryptocurrecny like
Bitcoin or Ethereum.

\item \textbf{Benevolent Ruler}: On the other end of the spectrum, the Tor
Project is itself a nonprofit with altruistic organization intentions. In this
extreme view, the authorities can simply set the tax rate to 100\% to turn
moneTor into a revenue stream for centralized efforts to improve the network.

\item \textbf{In-Kind Only}: If financial rewards turn out to be prohibitive, we can
mimic several of the purely in-kind nature of non-transferable and transferable
benefits described in Section~\ref{sec:related_work} by limiting the number of
on-ledger transactions to zero or one, respectively. For instance, at zero
allowed transactions, relays would be unable to use their tokens for any purpose
other than to buy premium traffic from other relays in a model that is
superficially reminiscent of BRAIDS~\cite{jansen2010recruiting}.

\item \textbf{Subsistence Relaying}: A conservative monetary approach is to limit the
expected value of the tokens to the cost of running a relay. In this paradigm,
every relay operator would effectively act as a individual NGO to avoid
potential liabilities that come with financial profit. In practice, moneTor
could support this design by adjusting global parameters such that the direct
incomes of any relay does not exceed the cost of relay operation in the cheapest
geographic region. Any descrepancy from this income floor due to location and
bandwidth would be covered in the tax redistribution policy. Previously, studies
have indicated that incentives can be effective when limited support is offered
for a ``concrete'' task~\cite{10.1257/jep.25.4.191, 10.1086/431263}. Provided
careful publication communication of intent, we might expect the same results
here.

\item \textbf{Decentralized Grant-making}: Finally, the Tor Project can
implement moneTor tokens as a form of \emph{altruistic money} to contribute to a
curated list of proposed projects. Relay operators who accumulate tokens might
chose to contribute a program to increase the number of relays in country $X$,
to fund Tor research on topic $Y$, or to increase advocacy for a pro-privacy
issue $Z$. Concretely, the Tor Project would implement this strategy by
witholding the original collatoral value collected by clients and let relays
vote by burning coins on the ledger. The result is a fungible asset that is
extrinsically worthless to profit-driven entities but intrinsically meaningful
to parties interested in the future direction of Tor.

\end{enumerate}

These toy examples are extreme strategies meant to illustrate the flexibility of
moneTor as a technical intrastruture. In practice, the Tor Project would be free
to implement essentially any combination of these approaches. The picture we
wish to illustrate is this: in the absence of a clear consensus on the right
mode of social incentivization, it is far more practical to build a versatile
system and constrain it afterwards than it is to design a constrained system
from the beginning.

Much future work remains to implement some of these methods. On the technical
side, robust monitoring must be implemented to mitigate inflated bandwidth for
tax redistribution and junk traffic insertion for direct payments. Economically,
we require a better understanding of topics such as location-based incentives
and the elasticity of supply and demand for various types of relay incentives
and client premium bandwidth, respectively.

%% ---------------------------------------------
%% Tor is a distributed network based on volunteered contribution: physical
%% machines are bought or rented by people all around the world and connected
%% through the Tor routing protocol. Those people are taking time and spending part
%% of their income contributing to Tor for various kind of reasons, among
%% philosophical, political and philantropical intrinsic motivations. That is,
%% relay operators do not expect to make money from their contributions. An
%% important consideration to have is to maintain the strength of intrinsic
%% motivations of the current community and to prevent monetary rewards to harm it.
%% In this spirit, several questions~\cite{jansenblogpost} were raised covering
%% potential social issues from previous attempts to design incentives, and among
%% them the ``crowding out'' effect~\cite{10.1257/jep.25.4.191}, showing how
%% explicit incentives such as monetary rewards can lead to a decrease of intrinsic
%% motivations. Social experimentation in the literature covering prosocial
%% behaviours~\cite{10.1257/aer.96.5.1652} has shown that rewarding the behaviour
%% could backfire and decrease the amount of participants to the task instead of
%% the expected increase, due to psychological effects on self-representation, on
%% the image we want to send to others, or to mismatching expectation on what the
%% reward should be.

%% Rewards in the prosocial studies were usually linked to profit or to
%% compensation. However, monetary incentives does not necessarly need to be about
%% that. A conservative approach could be taken by not rewarding relays for
%% relaying bandwidth, but by offering an in-kind support with moneTor tokens --
%% that is, the expected value of the received tokens cannot exceed the cost of
%% running relays, and is expected to support the estimated cost of the relay. In
%% this reasoning, every relay operator could act as nonprofit and not suffer from
%% potential liability of what running relays for profit could create. The received
%% money would not compensate the cost of running the infrastructure, which is
%% fueled by the operator's intrinsic motivations. In practice, moneTor could
%% support this design by ajusting the global price for direct payments such that
%% the direct incomes of the cheapest place in the Internet to run relays does not
%% exceed the cost of running a relay at this place, also weakening the incentives
%% of exit relays to inject junk traffic. Then, the tax redistribution could be a
%% function accounting for relay location and relay bandwidth that would try to
%% reimburse the marginal cost of running the relays in the network. Many more
%% questions are tied to tax redistribution, such as preventing operators to game
%% the system (security), evaluating the right price for a given location
%% (economic) or protecting the trust relationship~\cite{10.1257/aer.96.5.1611}
%% between Tor authorities and relays (social). Could the tax redistribution be a
%% community decision? Answering those questions would yield interesting further
%% work, with the goal to incentive operators with intrinsic motivations to deploy
%% more relays, and potentially at more diverse locations (usually more costly). It
%% has been shown in incentives for Education~\cite{10.1257/jep.25.4.191,
%% 10.1086/431263}, that when some support was offered for a concrete task (e.g.,
%% course enrollment), the social experiment was showing positive results. We
%% expect that offering support for operators with intrinsic motivation to run
%% relays should yield similar results than offering support to student with
%% intrinsic motivation to complete studies.

%% Applying an in-kind support should not mix operators attracted by extrinsic
%% values with operators holding intrinsic motivation, yet this remains to be
%% verfied in a concrete experimentation. However, it could be interesting to
%% motivate the Intermediary role with extrinsic value. The Tor network would
%% require in practice a limited amount of Intermediaries, but they must offer
%% strong availability to maintain the micropayment channel states for all peers
%% and enable robustness in the payment layer. Running Intermediary nodes requires
%% the operator to have more CS abilities than operators running classical Tor
%% relays. It is likely that observing, understanding and helping the Tor project
%% with technical limitations of the implementation would be an important task for
%% a limited amount of operators handling a new role within the network. However,
%% an Intermediary does not require a trustworthy operator, but someone
%% incentivized to provide good services. Those intermediaries are by design not
%% tasked to relay user streams, hence those operators are not to be mixed with the
%% relay operator community.

