\subsection{Tor}

\subsubsection{Tor Architecture}
The Tor network is composed of multiple different components, and each of which
run the same code base. Volunteers can run relays and enable specific roles or
tasks such as \textit{network consensus directories}, \textit{HSDirs} and
\textit{Exit policies}. \textit{Directory authorities} and \textit{Bandwidth
  authorities} are the most important components of the network, and are only
operated by trustworthy core contributors. There are currently 9 directory
authorities which periodically reach agreement over the state of the network,
called the \textit{consensus document}. This consensus document holds the
identification information related to all available relays inside the network,
as well as the result of the authority vote (e.g., a set of \textit{flags}
associated to each relay). The bandwidth authorities constantly measure every
relay and provide to the directory authorities a measurement value for each of
them, which plays a critical role in the path selection algorithm. This paper
extends this architecture by adding two new roles needed for our monetary relay
incentivization: Intermediary and Ledger.

\subsubsection{Traffic Analysis}
Tor's threat model assumes a local adversary who can observe some fraction of
the network and can operate or compromise a number of onion routers. Tor also
assumes a local adversary who can manipulate user's streams. The adversary can,
insert, modify, delete or delay data to create observable
perturbations. Typically, by observing both ends of an anonymous stream, an
attacker can infer the participating parties using statistical correlation. The
adversary's precision can also be improved by traffic flow perturbations. This
attack is called \textit{end-to-end correlation}. Tor does not try to implement
countermeasures to this attack but strives to minimize its impact. Recently,
Rochet and Pereira~\cite{popets-dropping} showed that a silent near to perfect
and instantaneous active traffic confirmation attack exists in Tor, leveraging
an essential property of distributed system: forward compatibility. Those
results show the need to strive to reduce their impact. The approach presented
here is to induce more diversity to the Tor network, which would make traffic
analysis against a large fraction of Tor users more costly. The central goal of
this paper is to give a tool to the Tor project to shape the Tor network
diversity through monetary incentivization.

\subsubsection{Circuit handling on Tor clients}

\subsubsection{Evaluating Tor's performance}
Shadow~\cite{jansen2011shadow} is a discrete event networking simulator that allows
real, unmodified applications to run within a virtual network. Its primary
advantage is the ability to run native networking application code that
interfaces with the simulator via an external application-specific
plugin. Within the simulation environment itself, Shadow faithfully mimics the
real Tor network conditions including bandwidth and latency. As a result,
experiments conducted with this tool tends to be more accurate than ones
conducted over alternatives such as private universities networks or
PlanetLab~\cite{Chun:2003:POT:956993.956995}.

In this paper, we use Shadow to evaluate our payment layer extension of the Tor
protocol in order to measure its networking impact and its feasibility.
\subsubsection{Tor's scheduling}


\subsection{Cryptocurrencies}

The modern generation of decentralized digital currencies traces its roots to
Nakamoto's Bitcoin protocol~\cite{nakamoto2008bitcoin}. This family of
cryptocurrencies are characterized by use of a public distributed ledger, often
a blockchain, and a notion identity and ownership based in public key
cryptography. A wide range of base-layer payment protocols have emerged, most
notably for us the fully programmable smart contract platform
Ethereum~\cite{wood2014ethereum} and anonymity-focused schemes such as
Zerocash~\cite{sasson2014zerocash} and Cryptonote~\cite{van2013cryptonote}. In
this second class of anonymous currencies, the general security model requires
that user is able to upload verfiable and irreversible proof of payment to the
ledger without leaking sender identity, recipient identity, or payment
value. Cryptonote achieves a weaker probablistic guarantee of this privacy using
a combination of lightweight ring signatures and stealth addresses. Zerocash
achieves anonymity by leveraging more flexible but expensive non-interactive
zero-knowledge proofs.

As more niches have emerged for types of cryptocurrencies, one active area of
research is ledger interoperability. Back et al. published the first primitive
proposal for sidechains, which specifies how bitcoins can be migrated from the
main blockchain onto a secondary chain~\cite{back2014enabling}.  More recently,
Poon and Buterin explained how Ethereum can support multiple levels of
arbirarily configured \emph{child chains} that inherit many notions of security
from the main Ethereum he.

The core cryptocurrency component featured in moneTor are multi-party
bidirectional micropayment channels. Base layer cryptocurrency protocols suffer
from severe scalability limits as they typical are capped on the order of tens
of transactions per second~\cite{team2017blockchain}. As base layer scaling
solutions inevitably face fundamental limitations, the most actively pursued
path thusfar if off-chain payment channel networks~\cite{poon2016bitcoin}. In
this setup, a single ledger transaction is used to escrow funds by two parties
$A$ and $B$. $A$ and $B$ can then proceed to make bidirectional micropayments to
each other \emph{without ledger interaction} through the exchange of signed IOU
tokens. By themselves, channels are useful for reducing ledger transactions
between parties with reoccurring interactions. More importantly, however,
micropayment channels can be extended such that $A$ pays $B$ through some
\emph{intermediary} party $I$ to which they both have active micropayment
channels. Pragmatically, $A$ and $B$ might occupy the roles of a customer and
merchant who are registered with a well-known financial service $I$, enabling
remarkable scaling. Informally multi-party channels are secure if the following
properties can be guaranteed:

\begin{enumerate}
\item At every step of the protocol, all parties possess proof of execution of
  the last finalized payment state
\item Given two proofs of payment states, the network can unambigiously identify
  the more recent state.
\item When $A$ agrees to pay $B$ through $I$, the payment is atomic. That is,
  there is never a situation in which $I$ pays $B$ but is unable to extract the
  agreed-upon payment from $A$.
\end{enumerate}

If all of these properties can be cryptographically ensured, then it becomes a
matter of network policy to ensure game-theoretic stability whereby which all
parties are incentivized behave honestly. The Bitcoin Lightning Network and
other similar protocols utilize simple hash commitments and transaction delays
to construct a secure scheme.

Finally, the micropayment channel concept has been extended to support anonymity
design goals. Z-Channel specifies an implementation designed specifically for
Zerocash which only supports two-party channels~\cite{zhang2017z}. Green and
Miers designed Bolt, which does support multi-party bidirectional
channels~\cite{green2017bolt}. In this setup, the anonymity set is defined with
respect to the users connected to the intermediary. In other words, given a set
of end users $E_{all} = \{E_1, E_2, ... E_n\}$ who all have active channels with
$I$, $E_a$ should be able to send secure payment to $E_b$ where $I$ cannot
identify $E_a$ or $E_b$ from $E_{all}$ nor can $I$ determine the payment
value. Of course, $I$ should still be able to verify (using zero-knowledge
proofs) that the payment is valid and that its internal channel states have been
updated accordingly. Several nuances arise concerning end user privacy in the
micropayment channel setup phase and in the event that $I$ maliciously aborts,
but we do not consider it necessary to discuss caveats here.