The technical prelimaries in constructing our system are broadly divided into
applied onion routing research in the Tor network and recent developments
in cryptocurrency payment protocols.

\subsection{Tor}

\subsubsection{Tor Architecture}
The Tor network is composed of multiple different components, each of which run
the same code base. Volunteers can run relays and enable specific roles or tasks
such as \textit{network consensus directories}, \textit{HSDirs} and \textit{Exit
  policies}. \textit{Directory authorities} and \textit{Bandwidth authorities}
are the most important components of the network, and are only operated by
trustworthy core contributors. There are currently nine directory authorities
which periodically reach agreement over the state of the network, called the
\textit{consensus document}. This consensus document holds the identification
information related to all available relays inside the network, as well as the
result of the authority vote (e.g., a set of \textit{flags} associated to each
relay). The bandwidth authorities constantly measure every relay and provide to
the directory authorities a measurement value for each of them, which plays a
critical role in the path selection algorithm. This paper extends the
architecture by adding two new roles needed for our monetary relay
incentivization: Intermediary and Ledger.

\subsubsection{Traffic Analysis}
Tor's threat model assumes a local adversary who can observe some fraction of
the network and can operate or compromise a number of onion routers. Tor also
assumes a local adversary who can manipulate user's streams. The adversary can,
insert, modify, delete or delay data to create observable
perturbations. Typically, by observing both ends of an anonymous stream, an
attacker can infer the participating parties using statistical correlation. The
adversary's precision can also be improved by traffic flow perturbations. This
attack is called \textit{end-to-end correlation}. Tor does not try to implement
countermeasures to this attack but strives to minimize its impact. Recently,
Rochet and Pereira~\cite{rochet2018dropping} showed that a silent near to perfect
and instantaneous active traffic confirmation attack exists in Tor, leveraging
an essential property of distributed system: forward compatibility. Those
results show the need to strive to reduce their impact. The approach presented
here is to induce more diversity to the Tor network, which would make traffic
analysis against a large fraction of Tor users more costly. The central goal of
this paper is to give a tool to the Tor project to shape the Tor network
diversity through monetary incentivization.

\subsubsection{Circuit handling on Tor clients}
%
%\td{TODO: This section will provide a more detailed explanation of how circuits are
%created and how that affects moneTor's high initial latency and overhead design}

When an application uses a Tor client, it creates a stream. Tor protects the privacy of the user by routing its streams inside Tor circuits. Therefore, as soon as a stream is created, it must directly be routed through an available Tor circuit to speed up the time to fist byte (i.e., the time to get the first byte of the response to a request we sent through Tor). In order to allow such scenario, Tor circuits are built pre-emptively once the clients obtain directory information (i.e., a consensus and router descriptors), and Tor tries to predict the needed circuits depending on the client usage. Indeed, every second the Tor client estimates how many clean circuits (not yet used) it predicts to use in the forthcoming moment, and decides to build new ones if it believes not to have enough.

For our moneTor design, we are going to follow the same strategy when building our payment channels: we pre-emptively build them to speed-up the time to first payment.

\subsubsection{Evaluating Tor's performance}
Shadow~\cite{jansen2011shadow} is a discrete event networking simulator that allows
real, unmodified applications to run within a virtual network. Its primary
advantage is the ability to run native networking application code that
interfaces with the simulator via an external application-specific
plugin. Within the simulation environment itself, Shadow faithfully mimics the
real Tor network conditions including bandwidth and latency. As a result,
experiments conducted with this tool tends to be more accurate than ones
conducted over alternatives such as private universities networks or
PlanetLab~\cite{Chun:2003:POT:956993.956995}.

In this paper, we use Shadow to evaluate our payment layer extension of the Tor
protocol in order to measure its networking impact and its feasibility.

\subsubsection{Tor's Scheduling}

Tor handles multiple queues of cells, for each circuit, and manages to write
cells in the outbound connection while favoring bursty over bulky traffic. The
main idea is to prioritize circuit handling interactive data streams, like chats
or web browsing. Tor uses an heuristic called EWMA~\cite{ccs10-scheduling}
(i.e., computes the exponentially weighted moving average for the number of
cells sent on each circuit) to decide which circuit to prioritize. Recently, the
efficiency of EWMA has been improved with the Kist~\cite{jansen14-kist}
scheduler used to reduce the congestion on the kernel outbound queue and push
back this delicate problem on the Tor layer. Thanks to Kist, EWMA should better
fulfill Tor's intended interactive streams prioritization. \td{should we still
  talk about this if scheduling is not used?}

\subsubsection{Flow Control}

To maintain a managable flow of cells between nodes, Tor implements an overlay
flow-control mechanism. The design caps the rate of cell transmission at two
different levles.

\begin{itemize}
\item Circuit-level: transmission window between two tor nodes (default: 1000 cells)
\item Stream-level: end-to-end transmission window between a client and the
  edge connection at the exit node (default: 500 cells)
\end{itemize}

Separate windows are maintained for both the inbound and outbound
directions. Tor attempt to keep the windows close to full at all times as long
as there is traffic to process. When a relay has flushed some number of cells,
it considers whether it has free space in its output buffes. If so, it sends a
special RELAY\_COMMAND\_SENDME cell to initialize the next batch of cells. In
networking design of moneTor, we will exploit this flow control functionality to
generate priortized traffic.

\subsection{Cryptocurrencies}

\subsubsection{Currency}

The modern generation of decentralized digital currencies traces its roots to
Nakamoto's Bitcoin protocol~\cite{nakamoto2008bitcoin}. This family of payment
protocols is characterized by use of a public distributed ledger, often a
blockchain, and a notion of identities and ownership based in public key
cryptography. A wide range of base-layer protocols have emerged; most
relevant for our purposes are the fully programmable smart contract platform
Ethereum~\cite{wood2014ethereum} and anonymity-focused schemes such as
Zerocash~\cite{sasson2014zerocash} and Cryptonote~\cite{van2013cryptonote}. In
this second class of anonymous currencies, the general security model requires
that user is able to upload verfiable and irreversible proof of payment to the
ledger without leaking sender identity, recipient identity, or payment
value. Cryptonote achieves a weaker probablistic guarantee of privacy using
a combination of lightweight ring signatures and stealth addresses. Zerocash
achieves anonymity by leveraging more flexible but expensive non-interactive
zero-knowledge proofs.

As more niches have emerged for varying types of cryptocurrencies, one active
area of research is ledger interoperability. Back et al. published the first
primitive proposal for sidechains, which specifies how bitcoins can be migrated
from the main blockchain onto a secondary chain~\cite{back2014enabling}.  More
recently, Poon and Buterin explained how Ethereum can support multiple levels of
arbirarily configured \emph{child chains} that inherit many security properties
from the main Ethereum blockchain.

\subsubsection{Payment Channels}

The core cryptocurrency component featured in moneTor is the multi-party
bidirectional micropayment channel. Base layer cryptocurrency protocols suffer
from severe scalability limits as they are typically capped on the order of tens
of transactions per second~\cite{team2018blockchain}. As base layer scaling
solutions inevitably face fundamental limitations, the most actively pursued
path thusfar is toward off-chain payment channel
networks~\cite{poon2016bitcoin}. In this setup, a single ledger transaction is
used to escrow funds by two parties $A$ and $B$. These parties may then proceed
to make bidirectional micropayments to each other \emph{without ledger
  interaction} through the exchange of signed IOU tokens. By themselves,
channels are useful for reducing ledger transactions between parties with
reoccurring interactions. More importantly, however, micropayment channels can
be extended such that $A$ has the ability to pay $B$ through some
\emph{intermediary} party $I$ to which they both have active micropayment
channels. Pragmatically, $A$ and $B$ might occupy the roles of a customer and
merchant who are registered with a well-known financial service provider $I$,
enabling remarkable scaling potential. Informally multi-party channels are
secure if the following properties can be guaranteed:

\begin{enumerate}
\item At every step of the protocol, all parties possess proof of execution of
  the last finalized payment state
\item Given two proofs of payment states, the network can unambigiously identify
  the more recent state.
\item When $A$ agrees to pay $B$ through $I$, the payment is atomic. That is,
  there is never a situation in which $I$ pays $B$ but is unable to extract the
  agreed-upon payment from $A$.
\end{enumerate}

If all of these properties can be cryptographically ensured, then it becomes a
matter of network policy to ensure game-theoretic stability whereby which all
parties are incentivized behave honestly. The Bitcoin Lightning Network and
other similar protocols utilize simple hash commitments and transaction delays
to achieve security.

\subsubsection{Anonymous Payment Channels}

Finally, the micropayment channel concept has been extended to support anonymity
design goals. Z-Channel specifies an implementation designed specifically for
Zerocash which only supports two-party channels~\cite{zhang2017z}. Green and
Miers designed Bolt~\cite{green2017bolt}, which accomodates multi-party channel construction as
well. In such a setup, the anonymity set is defined with
respect to the set of users connected to the intermediary. In other words, given
a set of end users $E_{all} = \{E_1, E_2, ... E_n\}$ who each have an active
channel with $I$, $E_a$ should be able to send a secure payment to $E_b$ even
though $I$ cannot identify $E_a$ or $E_b$ from $E_{all}$ nor can $I$ infer the
payment value. Of course, $I$ must still be able to verify that the payment is
valid and that its internal channel states have been updated
accordingly.~\footnote{This is achieved using a combination of zero-knowledge
  proofs and blind signatures, in the case of~\cite{green2017bolt}} Several
nuances arise concerning end user privacy in certain situations, such as during
the micropayment setup escrow phase and and in the event that $I$ maliciously
aborts. We do not consider it necessary to discuss these caveats here as they
are not critical to our later treatment of anonymous payment channels.
