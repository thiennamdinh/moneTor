\subsection{Tor}

\subsubsection{Tor Architecture}
The Tor network is composed of multiple different components, and each of which
run the same code base. Volunteers can run relays and enable specific roles or
tasks such as \textit{network consensus directories}, \textit{HSDirs} and
\textit{Exit policies}. \textit{Directory authorities} and \textit{Bandwidth
  authorities} are the most important components of the network, and are only
operated by trustworthy core contributors. There are currently 9 directory
authorities which periodically reach agreement over the state of the network,
called the \textit{consensus document}. This consensus document holds the
identification information related to all available relays inside the network,
as well as the result of the authority vote (e.g., a set of \textit{flags}
associated to each relay). The bandwidth authorities constantly measure every
relay and provide to the directory authorities a measurement value for each of
them, which plays a critical role in the path selection algorithm. This paper
extends this architecture by adding two new roles needed for our monetary relay
incentivization: Intermediary and Ledger.
\subsubsection{Traffic Analysis}
Tor's threat model assumes a local adversary who can observe some fraction of
the network and can operate or compromise a number of onion routers. Tor also
assumes a local adversary who can manipulate user's streams. The adversary can,
insert, modify, delete or delay data to create observable
perturbations. Typically, by observing both ends of an anonymous stream, an
attacker can infer the participating parties using statistical correlation. The
adversary's precision can also be improved by traffic flow perturbations. This
attack is called \textit{end-to-end correlation}. Tor does not try to implement
countermeasures to this attack but strives to minimize its impact. Recently,
Rochet and Pereira~\cite{popets-dropping} showed that a silent near to perfect
and instantaneous active traffic confirmation attack exists in Tor, leveraging
an essential property of distributed system: forward compatibility. Those
results show the need to strive to reduce their impact. The approach presented
here is to induce more diversity to the Tor network, which would make traffic
analysis against a large fraction of Tor users more costly. The central goal of
this paper is to give a tool to the Tor project to shape the Tor network
diversity through monetary incentivization.

\subsubsection{Circuit handling on Tor clients}

\subsubsection{Evaluating Tor's performance}
Shadow~\cite{shadow-ndss12} is a discrete event networking simulator that allows
real, unmodified applications to run within a virtual network. It was originally
developed to conduct more accurate and large scale Tor experimentations in a
private and controlled environment. Shadow's primary advantage is the ability to
run native networking application code that interfaces with the simulator via an
external application-specific plugin. \td{first three sentences seem somewhat
  redundant, maybe shorten to two?}  Within the simulation environment itself,
Shadow faithfully mimics the real Tor network conditions including bandwidth and
latency. As a result, experiments conducted with this tool tends to be more
accurate than ones conducted over alternatives such as private universities
networks or PlanetLab~\cite{Chun:2003:POT:956993.956995}.

In this paper, we use Shadow to evaluate our payment layer extension of the Tor
protocol in order to measure its networking impact and its feasibility.
\subsubsection{Tor's scheduling}


\subsection{Cryptocurrencies}



\begin{itemize}
\item bitcoin/ethereum~\cite{nakamoto2008bitcoin}~\cite{wood2014ethereum}
\item anonymous cryptocurrencies: zcash + monero
\item payment channels
\item anonymous payment channels
\item side chains/plasma
\end{itemize}

\flo{I think we just need a comprehensive view of payment channels, and what problems they solve? I am going to talk a bit about Tor's scalability issue in Tor's background, and that's exactly for what payment channels are awesome :)}