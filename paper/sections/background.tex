\subsection{Tor}

\subsubsection{Tor Architecture}
The Tor network is composed of multiple different components, each of which run
the same code base. Volunteers can run relays and enable specific roles or tasks
such as \textit{network consensus directories}, \textit{HSDirs} and \textit{Exit
  policies}. \textit{Directory authorities} and \textit{Bandwidth authorities}
are the most important components of the network, and are only operated by
trustworthy core contributors. There are currently nine directory authorities
which periodically reach agreement over the state of the network, called the
\textit{consensus document}. This consensus document holds the identification
information related to all available relays inside the network, as well as the
result of the authority vote (e.g., a set of \textit{flags} associated to each
relay). The bandwidth authorities constantly measure every relay and provide to
the directory authorities a measurement value for each of them, which plays a
critical role in the path selection algorithm. This paper extends the
architecture by adding two new roles needed for our monetary relay
incentivization: Intermediary and Ledger.

\subsubsection{Traffic Analysis}
Tor's threat model assumes a local adversary who can observe some fraction of
the network and can operate or compromise a number of onion routers. Tor also
assumes a local adversary who can manipulate user streams by inserting,
modifying, deleting or delaying data to create observable
perturbations. Typically, by observing both ends of an anonymous stream, an
attacker can infer the participating parties using statistical correlation. The
adversary's precision can also be improved by traffic flow perturbations. This
attack is called \textit{end-to-end correlation}. Tor does not implement
explicit countermeasures to this attack but does strive to minimize its
impact. Still, Rochet and Pereira recently showed that Tor's essential
forward-compatibility feature can be exploited in a silent near to perfect and
instantaneous active traffic confirmation attack~\cite{rochet2018dropping}.These
developments imply a serious need to induce more diversity to the Tor network
which would make such traffic analysis against a large fraction of Tor users
more costly. One such solution is presented in this paper.

\subsubsection{Circuit handling on Tor clients}
%
%\td{TODO: This section will provide a more detailed explanation of how circuits are
%created and how that affects moneTor's high initial latency and overhead design}

Application traffic in Tor is handled through streams
which are multiplexed with other streams within a circuit. In order to reduce
latency in the user experience, Tor attempts to build circuits preemptively as
soon as the client obtains the necessary directory information. Once a stream is
created by a user application, it can be immediately routed through the idle
circuit, dramatically improving latency. The anticipated number of required
preemptive circuits is dynamically calculated every second by the Tor client. We
exploit this same strategy for moneTor's circuit setup routine.

\subsubsection{Evaluating Tor's performance}
Shadow~\cite{jansen2011shadow} is a discrete event networking simulator that
allows real, unmodified applications to run within a virtual network. Its
primary advantage is the ability to run native networking application code that
interfaces with the simulator via an external application-specific
plugin. Within the simulation environment itself, Shadow faithfully mimics the
real Tor network conditions including bandwidth and latency. As a result,
experiments conducted with this tool tends to be more accurate than ones
conducted over alternatives such as private universities networks or
PlanetLab~\cite{chun2003planetlab}. We utilize the shadow framework to
evaluate the costs and performance of our algorithms over network-level
simulations.

%\subsubsection{Tor's Scheduling}

\subsubsection{Flow Control} Tor seeks to maintain an optimal flow of cells in each
circuit by capping the rate of transmission to fixed-sized windows. The
CIRCWINDOWSTART value dictates flow for the overall circuit while
STREAMWINDOWSTART dictates each end-to-end flow between a client and the edge
connection at an exit relay.\footnote{Default values, expressed in number of
  cells, are CIRCWINDOWSTART = 1000 and STREAMWINDOWSTART = 500} The number
of cells going in each direction are tracked separately. In effect, Tor attempts
to keep the flow of cells on each circuit as high as possibly which ensuring
that the number of cells in transition do not exceed the limit. In the network
design of moneTor, we will exploit these flow control windows to differentiate
preferred traffic.

\subsection{Payment Channels}

%\textbf{Secure Payments} The modern generation of decentralized digital
%currencies traces its roots to Nakamoto's Bitcoin
%protocol~\cite{nakamoto2008bitcoin}. This family of payment protocols is
%characterized by use of a public distributed ledger, often a blockchain, and a
%notion of identities and ownership based in public key cryptography. A wide
%range of base-layer payment protocols have emerged; most relevant for our
%purposes is the fully programmable smart contract platform
%Ethereum~\cite{wood2014ethereum} and anonymity-focused schemes such as
%Zerocash~\cite{sasson2014zerocash} and Cryptonote~\cite{van2013cryptonote}. In
%this second class of anonymous currencies, the general security model requires
%that the payment sender is able to upload verifiable and irreversible proof of
%payment to the ledger without leaking sender identity, recipient identity, or
%payment value. Cryptonote achieves a probabilistic guarantee of privacy using a
%combination of lightweight ring signatures and stealth addresses but may be
%vulnerable in typical use~\cite{miller2017empirical}. Zerocash achieves stronger
%theoretical anonymity by leveraging more expensive non-interactive
%zero-knowledge proofs which require a trusted setup phase.
%
%Researchers have explored interoperability between ledgers . Back et al.\
%published the first primitive proposal for \emph{sidechains}, which
%conceptualizes a secondary blockchain ledger that can send and receive assets
%from the primary Bitcoin blockchain~\cite{back2014enabling}. More recently, Poon
%and Buterin explain how Ethereum can support multiple levels of arbitrarily
%configured \emph{child chains} that inherit many security properties from the
%original blochain.

\subsubsection{Payment Channels} The core cryptocurrency component featured in moneTor
is the tripartite bidirectional micropayment channel. Base layer cryptocurrency
protocols are typically capped on the order of tens of transactions per
seconds~\cite{team2018blockchain}. The most actively pursued path toward better
scalability thus far is work in off-chain payment channel
networks~\cite{poon2016bitcoin}. In this setup, a single ledger transaction is
used to escrow funds by two parties $A$ and $B$. These parties may then proceed
to make bidirectional micropayments to each other \emph{without ledger
  interaction} through the exchange of signed IOU tokens. \op{What does IOU stand for?} By themselves,
channels are useful for reducing the number of ledger interactions for parties
with reoccurring interactions. More consequential for the scalability problem is
the tripartite channel paradigm in which $A$ pays $B$ through some
\emph{intermediary} party $I$ with which they both maintain active channels. In
practice, $A$ and $B$ might occupy the roles of a customer and merchant who are
registered with a well-known financial service provider $I$. $A$, $B$, and $I$
need only interact with the ledger periodically to deposit and withdraw large sums of
money, improving the network capacity by multiple orders of
magnitude. Informally, tripartite channels are secure if the following
requirements are met.

\begin{enumerate}
\item At every step of the protocol, all parties possess proof of execution of
  the last finalized payment state
\item Given two proofs of payment state, the network can unambiguously identify
  the more recent state.
\item When $A$ agrees to pay $B$ through $I$, the payment is atomic. That is,
  there is never a situation in which $I$ pays $B$ but is unable to extract the
  agreed-upon payment from $A$.
\end{enumerate}

The guarantees are achieved in the Bitcoin Lightning network and other similar
protocols through simple hash commitment and transaction delay primitives. The
end result is a game-theoretic notion of security whereby all parties are
incentivized to behave honestly.

\subsubsection{Anonymous Payment Channels} The micropayment channel concept has since
been extended to support anonymity design goals. Z-Channel specifies an
implementation designed specifically for Zerocash which only supports two-party
channels~\cite{zhang2017z}. Green and Miers designed Bolt, which is the only
known anonymous micropayment scheme to date that supports three-party
intermediary channels~\cite{green2017bolt}. In their framework, the anonymity
set is defined with respect to the collection of users connected to the same
intermediary. In other words, given a set of end users
$E_{all} = \{E_1, E_2, ... E_n\}$ who each have an active channel with $I$,
$E_a$ should be able to send a secure payment to $E_b$ even though $I$ cannot
identify $E_a$ or $E_b$ from $E_{all}$ nor can $I$ infer the payment value. Of
course, $I$ must still be able to verify that the payment is valid and that its
internal channel states have been updated accordingly.
\footnote{This is achieved
  using a combination of zero-knowledge proofs and blind signatures, in the case
  of Bolt} 
Several nuances arise concerning end user privacy in certain
situations, such as during the micropayment setup escrow phase and in the event
that $I$ maliciously aborts. We do not consider it necessary to discuss these
caveats here as they are not critical to our later treatment of anonymous
payment channels.

%%% Local Variables:
%%% mode: latex
%%% TeX-master: "../main"
%%% End:
