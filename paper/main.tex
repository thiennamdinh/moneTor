\documentclass[sigconf, anonymous]{acmart}

\usepackage{amsmath}
\usepackage{algorithm}
\usepackage{algorithmicx}
\usepackage[noend]{algpseudocode}
\algrenewcommand{\algorithmiccomment}[1]{{\color{gray}$\triangleright$ #1}}

\fancyhf{} % Remove fancy page headers
\fancyhead[C]{Anonymous submission \#9999 to ACM CCS 2018} % TODO: replace 9999 with your paper number
\fancyfoot[C]{\thepage}

\setcopyright{none} % No copyright notice required for submissions
\acmConference[Anonymous Submission to ACM CCS 2018]{ACM Conference on Computer and Communications Security}{Due 19 May 2018}{Toronto, Canada}
\acmYear{2018}

\settopmatter{printacmref=false, printccs=true, printfolios=true} % We want page numbers on submissions
\usepackage{subcaption}
%%\ccsPaper{9999} % TODO: replace with your paper number once obtained
\newcommand{\flo}[1]{ {\color{red} FR: #1}}
\newcommand{\td}[1]{ {\color{blue} TD: #1}}

\begin{document}
\title{Monetary Tor incentives with efficient nanopayment channels} % TODO: replace with your title

\begin{abstract}

  Although it is the most widely-used secure traffic anonymization network in
  the world, Tor nevertheless suffers from shortcomings in network diversity and
  performance. One strategy to mitigate both problems entails the engineering of
  incentives in order to encourage more nodes to act as network relays. In this
  paper, we present moneTor: the first relay incentivization scheme which
  provides sufficiently secure, efficient, and economically robust payments. Our
  approach is constructed from very recent developments in the cryptocurrency
  domain and is designed to be directly implemented into the existing Tor
  architecture. Using a combination of live network data collection and analysis
  of our simulated design, we show that moneTor is demonstratively feasible,
  offering upwards of [X] improved premium bandwidth for paying clients at only
  a [X] cost in network overhead.

  \td{The following (including this abstract) is some combination of incomplete
    and temporary pre-draft reading material. Comments are marked in blue (like this)}

\end{abstract}

\begin{CCSXML}
<ccs2012>
<concept>
<concept_id>10002978.10002991.10002994</concept_id>
<concept_desc>Security and privacy~Pseudonymity, anonymity and untraceability</concept_desc>
<concept_significance>300</concept_significance>
</concept>
<concept>
<concept_id>10002978.10003014.10003015</concept_id>
<concept_desc>Security and privacy~Security protocols</concept_desc>
<concept_significance>300</concept_significance>
</concept>
</ccs2012>
\end{CCSXML}

\ccsdesc[300]{Security and privacy~Pseudonymity, anonymity and untraceability}
\ccsdesc[300]{Security and privacy~Security protocols}

\keywords{Tor, cryptocurrency, payment channels} % TODO: replace with your keywords

\maketitle

\section{Introduction}
\label{sec:introduction}
Anonymous traffic routing through Tor remains one of the most popular
low-latency methods for censorship evasion and privacy protection
~\cite{dingledine2004tor}. In this setup, clients protect their TCP/IP metadata
and content by routing their traffic through an onion encrypted path with three
randomly selected volunteer relay nodes, referred to as a circuit. The network
presently consists of $\approx 6,400$ relays contributing over 230 Gbit/s of
bandwidth globally~\cite{portal2018tormetrics}. While Tor has proven to be a
highly effective option for privacy-seeking users, it suffers from two
orthogonal issues that are relevant to this work. The first is the broad family
of collusion attacks. These threats are relevant in scenarios where an attacker,
who controls multiple nodes or network vantage points, is probabilistically
placed in key roles along a single circuit, opening a much easier path to client
deanonymization~\cite{wright2004predecessor,murdoch2005low}. The second problem
is performance. Although the overlay protocol itself generates an inherent
overhead in network resources, Tor suffers from additional traffic congestion
that leads to suboptimal network performance~\cite{portal2018tormetrics,
  alsabah2016performance}.

One approach to mitigate these problems is to address them separately from a
networking standpoint. Indeed, a significant portion of recent research in Tor
proposes modifications to the core protocol itself, such as reengineering the
TCP+TLS part of the stack~\cite{reardon2009improving} or designing a better
kernel-aware scheduler~\cite{jansen2014never}. A second approach observes that
the anonymity and performance of the network are both proportional to the number
of nodes and users. From this perspective, the problem becomes a largely
economic question: how can we incentivize more relay participation? There is a
long line of research that explores various strategies for incentivization
spanning over the past decade. Progress in this field faces a multitude of
challenges. Consider the approach most relevant to our work: monetary
incentives. Aside from the analytically intractable set of legal and
sociopolitical obstacles, monetary payments in this environment must overcome a
trifecta of challenging constraints:

\begin{itemize}
\item \emph{Anonymity}: The paramount mission of Tor is user privacy. This
  cannot be compromised or reduced by transparent money transaction trails.
\item \emph{Payment Security}: Anonymity prohibits the formation of large
  credit and trust systems. As such, financial transactions cannot transpire
  without strong guarantees of cryptographic security.
\item \emph{Efficiency}: Tor services millions of concurrent users, all of whom
  maintain largely short-term relationships. A robust payment system must handle
  extremely lightweight and scalable payments so as to accommodate the dynamic
  and often short-lived activities of these clients.
\end{itemize}

\textbf{Present Landscape} While a number of prior works satisfy some subset of
these constraints, no single proposal has thus far been sufficiently convincing
so as to warrant further development. We speculate that the lack of a
breakthrough in this niche area is not a matter of insufficient ingenuity but
rather one of timing. In recent years, there has been an explosion of academic
research within the domain of cryptocurrencies. As of this writing, Satoshi
Nakamoto's Bitcoin whitepaper has already garnered over 3,200 references in
citing publications~\cite{nakamoto2008bitcoin}.  \footnote{The recency of this
  development is highlighted by the fact that about half of this work is dated
  after 2017}  This body of work has sparked the development of a multitude of
new techniques that will prove indispensable to our own area of research. Our
objective then, is to leverage the full extent of these innovations into a
practical next-generation incentivization strategy for Tor.

\label{sec:Contributions}
\textbf{Contributions} The moneTor scheme is a novel full-stack framework for
Tor incentives. We make the following contributions in this paper:

\begin{enumerate}
\item Discuss economic considerations for the Tor network to favor network
  diversity and to help supporting Tor as a public good
  %market control such that the Tor project can decide which notion of diversity
  %to promote
\item Introduce new highly-efficient nanopayment protocols which may be of some
  independent interest for other high-frequency payment applications
\item Detail an integration strategy to add payment
  into the existing Tor architecture and networking layer while being compatible
  with Tor's anonymity properties and efficiency
\item Collect privacy-preserving client usage data to justify design parameters
  and argue for efficacy
\item Implement a prototype extending the Tor protocol
  in the core C software, featuring more than 15k lines of code
\item Conduct network-scale simulations to analyze the performance impact of the
  embedded payment and incentivization scheme
\item Identify an efficient method of throttling to give priority users an
  advantage
\end{enumerate}

The moneTor design leverages fully-specified algorithms for some components and
well-studied existing research for all others. Consequently, we claim that the
entire technical stack is accounted for and that moneTor can be feasibly
developed today.

\paragraph*{Roadmap.} In Section~\ref{sec:background}, we draw on technical
preliminaries from two distinct fields: applied Tor research and cryptographic
payment channels. Section~\ref{sec:related_work} presents the related work. Our
contributions begin formally with Section~\ref{sec:economic}, where we first
describe a high-level economic model for the flow of money through the
network. In Section~\ref{sec:payment}, we outline the technical construction of
our nanopayment scheme at the payment protocol level. Section~\ref{sec:network}
expands the technical construction of moneTor to cover modifications at the
network level. In section~\ref{sec:analysis} we justify our design decisions
with real-world data collection from live Tor users. We continue in
Section~\ref{sec:experimentations} with a validation of our technical design,
carried out through experiments performed on a native proof-of-concept
implementation. Finally, we discuss limitations and future work in
Section~\ref{sec:limitations_futurework}, reference our source code in
Section~\ref{sec:code}, and present concluding remarks in
Section~\ref{sec:conclusion}.

%%% Local Variables:
%%% mode: latex
%%% TeX-master: "../main"
%%% End:


\section{Background}
\label{sec:background}
\subsection{Tor}

\subsubsection{Tor Architecture}
The Tor network is composed of multiple different components, and each of which
run the same code base. Volunteers can run relays and enable specific roles or
tasks such as \textit{network consensus directories}, \textit{HSDirs} and
\textit{Exit policies}. \textit{Directory authorities} and \textit{Bandwidth
  authorities} are the most important components of the network, and are only
operated by trustworthy core contributors. There are currently 9 directory
authorities which periodically reach agreement over the state of the network,
called the \textit{consensus document}. This consensus document holds the
identification information related to all available relays inside the network,
as well as the result of the authority vote (e.g., a set of \textit{flags}
associated to each relay). The bandwidth authorities constantly measure every
relay and provide to the directory authorities a measurement value for each of
them, which plays a critical role in the path selection algorithm. This paper
extends this architecture by adding two new roles needed for our monetary relay
incentivization: Intermediary and Ledger.
\subsubsection{Traffic Analysis}
Tor's threat model assumes a local adversary who can observe some fraction of
the network and can operate or compromise a number of onion routers. Tor also
assumes a local adversary who can manipulate user's streams. The adversary can,
insert, modify, delete or delay data to create observable
perturbations. Typically, by observing both ends of an anonymous stream, an
attacker can infer the participating parties using statistical correlation. The
adversary's precision can also be improved by traffic flow perturbations. This
attack is called \textit{end-to-end correlation}. Tor does not try to implement
countermeasures to this attack but strives to minimize its impact. Recently,
Rochet and Pereira~\cite{popets-dropping} showed that a silent near to perfect
and instantaneous active traffic confirmation attack exists in Tor, leveraging
an essential property of distributed system: forward compatibility. Those
results show the need to strive to reduce their impact. The approach presented
here is to induce more diversity to the Tor network, which would make traffic
analysis against a large fraction of Tor users more costly. The central goal of
this paper is to give a tool to the Tor project to shape the Tor network
diversity through monetary incentivization.

\subsubsection{Circuit handling on Tor clients}

\subsubsection{Evaluating Tor's performance}
Shadow~\cite{shadow-ndss12} is a discrete event networking simulator that allows
real, unmodified applications to run within a virtual network. It was originally
developed to conduct more accurate and large scale Tor experimentations in a
private and controlled environment. Shadow's primary advantage is the ability to
run native networking application code that interfaces with the simulator via an
external application-specific plugin. \td{first three sentences seem somewhat
  redundant, maybe shorten to two?}  Within the simulation environment itself,
Shadow faithfully mimics the real Tor network conditions including bandwidth and
latency. As a result, experiments conducted with this tool tends to be more
accurate than ones conducted over alternatives such as private universities
networks or PlanetLab~\cite{Chun:2003:POT:956993.956995}.

In this paper, we use Shadow to evaluate our payment layer extension of the Tor
protocol in order to measure its networking impact and its feasibility.
\subsubsection{Tor's scheduling}


\subsection{Cryptocurrencies}

The modern generation of decentralized digital currencies traces its roots to
Nakamoto's Bitcoin protocol~\cite{nakamoto2008bitcoin}. This family of
cryptocurrencies are characterized by use of a public distributed ledger, often
a blockchain, and a notion identity and ownership based in public key
cryptography. A wide range of base-layer payment protocols have emerged, most
notably for us the fully programmable smart contract platform
Ethereum~\cite{wood2014ethereum} and anonymity-focused schemes such as
Zerocash~\cite{sasson2014zerocash} and Cryptonote~\cite{van2013cryptonote}. In
this second class of anonymous currencies, the general security model requires
that user is able to upload verfiable and irreversible proof of payment to the
ledger without leaking sender identity, recipient identity, or payment
value. Cryptonote achieves a weaker probablistic guarantee of this privacy using
a combination of lightweight ring signatures and stealth addresses. Zerocash
achieves anonymity by leveraging more flexible but expensive non-interactive
zero-knowledge proofs.

As more niches have emerged for types of cryptocurrencies, one active area of
research is ledger interoperability. Back et al. published the first primitive
proposal for sidechains, which specifies how bitcoins can be migrated from the
main blockchain onto a secondary chain~\cite{back2014enabling}.  More recently,
Poon and Buterin explained how Ethereum can support multiple levels of
arbirarily configured \emph{child chains} that inherit many notions of security
from the main Ethereum he.

The core cryptocurrency component featured in moneTor are multi-party
bidirectional micropayment channels. Base layer cryptocurrency protocols suffer
from severe scalability limits as they typical are capped on the order of tens
of transactions per second~\cite{team2017blockchain}. As base layer scaling
solutions inevitably face fundamental limitations, the most actively pursued
path thusfar if off-chain payment channel networks~\cite{poon2016bitcoin}. In
this setup, a single ledger transaction is used to escrow funds by two parties
$A$ and $B$. $A$ and $B$ can then proceed to make bidirectional micropayments to
each other \emph{without ledger interaction} through the exchange of signed IOU
tokens. By themselves, channels are useful for reducing ledger transactions
between parties with reoccurring interactions. More importantly, however,
micropayment channels can be extended such that $A$ pays $B$ through some
\emph{intermediary} party $I$ to which they both have active micropayment
channels. Pragmatically, $A$ and $B$ might occupy the roles of a customer and
merchant who are registered with a well-known financial service $I$, enabling
remarkable scaling. Informally multi-party channels are secure if the following
properties can be guaranteed:

\begin{enumerate}
\item At every step of the protocol, all parties possess proof of execution of
  the last finalized payment state
\item Given two proofs of payment states, the network can unambigiously identify
  the more recent state.
\item When $A$ agrees to pay $B$ through $I$, the payment is atomic. That is,
  there is never a situation in which $I$ pays $B$ but is unable to extract the
  agreed-upon payment from $A$.
\end{enumerate}

If all of these properties can be cryptographically ensured, then it becomes a
matter of network policy to ensure game-theoretic security of payments. The
Bitcoin Lightning Network and other similar protocols utilize simple hash
commitments and transaction delays to construct a secure scheme.

Finally, the micropayment channel concept has been extended to support anonymity
design goals. Z-Channel specifies an implementation designed specifically for
Zerocash which only supports two-party channels~\cite{zhang2017z}. Green and
Miers designed Bolt, which does support multi-party bidirectional
channels~\cite{green2017bolt}. In this setup, the anonymity set is defined with
respect to the users connected to the intermediary. In other words, given a set
of end users $E_{all} = \{E_1, E_2, ... E_n\}$ who all have active channels with
$I$, $E_a$ should be able to send secure payment to $E_b$ where $I$ cannot
identify $E_a$ or $E_b$ from $E_{all}$ nor can $I$ determine the payment
value. Of course, $I$ should still be able to verify (using zero-knowledge
proofs) that the payment is valid and that its internal channel states have been
updated accordingly. Several nuances arise concerning end user privacy in the
micropayment channel setup phase and in the event that $I$ maliciously aborts,
but we do not consider it necessary to discuss caveats here.

\section{Related Work}
\label{sec:related_work}
\subsection{Tor Incentive Schemes}

\subsection{MoneTor Preliminaries}

\section{Economic Policy}
\label{sec:tor_incentives}
We first summarize the economic layer of the incentivization scheme. At this
level, we assume the existence of an ideally secure and efficient payment layer
and proceed to outline the high-level policy design.

\subsection{Currency Design}

The moneTor framework proposes a native payment layer for the Tor ecosystem. To
qualify as a true monetary payment design, our payment tokens should satisfy the
standard properties of \textit{scarcity}, \textit{fungibility},
\textit{divisibility}, \textit{durability}, and
\textit{transferability}~\cite[p.3]{crump2011phenomenon} Furthermore, we follow
the Bitcoin paradigm in which all users maintain full and exclusive control of
their monetary wealth through use of public key cryptography.

Tokens must accrue some form of intrinsic value and one option is to launch the
moneTor token with an ICO maintained by decentralized mining.\footnote{Initial
  Coin Offering} There are two drawbacks to this approach.

\begin{enumerate}
\item Decentralized blockchains are inefficient
\item New currencies introduce undue economic complexity
\end{enumerate}

The efficiency of decentralized blockchains stems from the necessity to reach
agreement among an unbounded number of nodes through expensive consensus
mechanisms. Fortunately, the Tor network benefits in some ways from a more
centralized infrastructure comprised from the list of trusted authorities. We
propose to take advantage of this setup by introducing a new set of ledger
authorities who are responsible for maintaining publicly audited global
currency data.

To avoid the economic and financial nuances associated with a new currency, we
suggest a two-way value peg to fix moneTor tokens to a high-market cap
cryptocurrency such as Bitcoin or Ethereum~\cite{back2014enabling,
  poon2017plasma}. We recommend engineering the system under the Ethereum Plasma
model~\cite{poon2017plasma}. Aside from the efficiency improvements, it would
likely be feasible in the Plasma design to integrate smart contracts such that
the Ethereum blockchain becomes the authority ledger of last resort in the event
that the native Tor ledger goes offline. Effectively, this means we guarantee
that honest users will never completely lose access to their money so long as
the Ethereum blockchain is accessible.

Finally, moneTor requires a base layer payment protocol to anonymously move
large sums of money. To do this, we suggested either
Zerocash~\cite{sasson2014zerocash} or Cryptonote~\cite{van2013cryptonote}.

\subsection{Incentive Model}
The moneTor framework follows the same philosophy as prior incentivization
proposals by offering a paid \emph{premium} bandwidth product to Tor
users. Under this scheme, financially willing users send direct payments to each
relay along their circuits in exchange for higher internet bandwidth relative to
unpaid users. While we cannot strictly enforce that all relays will correctly
following the protocol in a decentralized network, the client can monitor her
own bandwidth and only make payments when it appears to be beneficial. This
setup can be viewed from a game theoretic tit-for-tat relationship in which the
relay has no apparent incentive to deviate.

In our setup, a key economic question to address is the issue of price
determination. While it would be tempting to enlist any number of market-based
mechanisms to set premium bandwidth prices, any price differentiation between
clients or relays inevitably leaks information. We therefore impose the
constraint that all users should pay a single uniform price for premium
bandwidth at any time $t$. This price may be centralized calculation by the
authorities or a more dynamical consensus vote reached by the network.

\textbf{Death and Taxes} A more convoluted problem is the issue of wealth
distribution. Given that Tor is a fundamentally nonprofit ecosystem, it is not
quite clear that an optimal profit-seeking network will closely aligned with the
core mission of the Tor Project. We address this problem with the introduction
of a taxation element. From every payment, a percentage of funds is anonymously
diverted to an account controlled by network authories and transparently
redistributed to relays or even other users with desirable properties. In
essence, taxation provides a tunable mechanism for the Tor Project to shape the
topology of the network towards better diversity and performance.


\section{moneTor Design}
\label{sec:design}
\subsection{Payment System Design}

\subsection{Implementation Design}


\section{Experimental Analysis}
\label{sec:analysis}
\subsection{Methodology}
\subsection{Data Collection}
\label{subsec:datacollection}

We deployed a data collection system to look for more realistic information about lifetime and bandwidth consumption through time of Tor circuits. Our objective is to have a deeper understanding of typical Tor usage, and if such usage can benefit from our channel-based payment system. For example, those measurements could capture some notion about the type and magnitude of potential premium traffic. We define the type of traffic based on the port used to connect to the request service. Besides the classical ports 80 and 443 for web traffic, we aggregate data based on some other families, such as the WHOIS protocol~\cite{rfc3912} and RWHOIS~\cite{rfc2167} with port 43 and 4321. The complete list of families is constructed from the reduced exit policies~\cite{reducedexitpolicies} we run on our relays. It allows us to reason based on application specific traffic.
%We interested to know about the distribution lifetime of Tor circuits for each port we allow. We are also interested to picture how many cells those circuits handled through their lifetime with some level of granularity.

\subsubsection{Efforts to preserve users privacy}

\subsubsection{Observations}

\begin{figure*}
	\centering
	\begin{subfigure}[t]{0.32\textwidth}
		\centering
		\includegraphics[scale=0.28]{images/exitmeasurement.png}
		\label{fig:stats_a}
	\end{subfigure}
	\begin{subfigure}[t]{0.32\textwidth}
		\centering
		\includegraphics[scale=0.28]{images/totcellcountscdf.png}
		\label{fig:stats_b}
	\end{subfigure}
	\begin{subfigure}[t]{0.32\textwidth}
		\centering
		\includegraphics[scale=0.28]{images/stddevs.png}
		\label{fig:stats_c}
	\end{subfigure}
	\label{fig:measurements}
	\caption{Tor measurements}
\end{figure*}
\subsection{Ethical Considerations}


\section{Limitations}
\label{sec:limitations}
\section{Future work}
\label{sec:future_work}

\section{Code and Data Reproducibility}
\label{sec:code}

\section{Conclusion}
\label{sec:conclusion}
\bibliographystyle{ACM-Reference-Format}
\bibliography{sections/references}

\appendix

\section{Algorithms}
\label{sec:algorithms}
This appendix describes the algorithms we design to operate moneTor nanopayment
channels.

\subsection{Conventions}

We adopt the following conventions in our algorithms.

\begin{itemize}
\item All variable names in this section, except for possibly helper
  functions, are globally unique.
\item Variable subscripts denote a party or role ((I)ntermediary,
  (C)lient, (R)elay, (E)nd user).
\item New nanopayment variables are prefixed with the character
  (n). All other variables reference a value from the original Bolt
  scheme, although the name might be altered somewhat.
\item Payment values ($\epsilon, \delta$) are signed integers with
  respect to the end user. For example, $\delta_C$ is negative and
  $\delta_R$ is positive in the typical case where a client is paying
  a relay.
\end{itemize}

\subsection{Variable Index}

The follow itemizes variables used in the algorithm design. The first level of
variables are used for actual cryptographic and accounting operations. These are
bundled into groups of higher level variable names meant to represent
abstraction concepts such as payment channels and party states. Only these high
level variables are saved outside the context of the algorithms.

$nT = (\delta_C, \delta_R, n, hc^0)$ --- ``Nanopayment Channel Token'' ---
Stores static, public information that defines a nanopayment channel including
the payment values on both legs, the max number of payments, and the hashchain
head. This can be passed around freely by all parties.

$ncsk_C = (nwpk_C, nwsk_C, HC)$ --- ``Client Nanopayment Secrets'' --- Includes
a Public/private key pair which allows the client to setup and close a
nanopayment channel and a precomputed hash chain to make incremental
nanopayments

$nS_C = (k, hc^k)$ --- ``Client Nanopayment State'' --- Mutable state of the
nanopayment; includes the count of payments made so far and the latest sent hash
pre-image

$nrt_C$ --- ``Client Nanopayment Refund'' --- Allows the client to make a claim
to the ledger on escrowed money at any time. This refund is signed by the
intermediary and conditioned on revealing the latest hash pre-image that the
client claims to have sent.

$nrc_C$ --- ``Client Channel Closure Message'' --- Final message that is
posted to the ledger by the client to claim all funds of the
micropayment channel including any completed nanopayments.

$nS_I = \{nT: channel\_state\}$ --- ``Intermediary Nanopayment State'' Map of
all past and present nanopayment channels and the corresponding channel
state. Possible states are:

\begin{itemize}
\item $\bot$ --- failed attempt at setting up a nanopayment channel
\item $ready$ --- channel has been set up by $C$
\item $established$ --- channel has been established with $R$
\item $closed||hc^k$ --- channel has been closed and no further payments
  are allowed
\end{itemize}

$ncsk_R (nwpk_C, nwsk_C, \bot)$ --- ``Relay Nanopayment Secrets'' --- Includes a
public/private key pair allows the relay to setup and close a nanopayment
channel. Since relays cannot make payments in this setup, the last field is left
blank.

$nS_R = (k, hc^k)$ --- ``Relay Nanopayment State'' --- See $nS_C$

$nrt_R$ --- ``Relay Nanopayment Refund'' --- See $nrt_C$

$nrc_C$ --- ``Relay Channel Closure Message'' --- See $nrc_C$
message
\subsection{Algorithms}

Here we formally specify the algorithms.

\begin{algorithm}
  \begin{algorithmic}[1]
    \caption{Helper function for creating a new wallet}
    \Function{Wal}{$pp, pk_{payee}, w, \epsilon$}
    \State{parse $w$ as $(B, wpk, wsk, r, \sigma^w)$}
    \State{$(wsk', wpk') \gets $KeyGen$(pp)$}
    \State{$r' \gets $Random$()$}
    \State{$wCom' \gets $Commit$(wpk', B + \epsilon, r')$}
    \State{$\pi \gets PK\{(wpk', B, r', \sigma^w)$: \par}
    \State{\hskip\algorithmicindent{} $wCom' = $Commit$(wpk', B + \epsilon, r')\ \wedge$}
    \State{\hskip\algorithmicindent{} Verify$(pk_{payee}, (wpk, B), \sigma^w)\ \wedge$}
    \State{\hskip\algorithmicindent{} $B + \epsilon \geq 0$}
    \State{\Return{$(wsk', wpk', wCom', \pi)\}$}}
    \EndFunction{}

  \end{algorithmic}
\end{algorithm}

\begin{algorithm}
  \caption{Nanopayment Channel Setup --- Protocol between a relay and
    intermediary to create a new nanopayment channel from an existing
    micropayment wallet. Run prior to circuit setup.}
  \begin{algorithmic}[1]
    \Procedure{Client}{$pp, pk_I, w_C, \delta_C, n$}
      \State{parse $w_C$ as $(B_C, wpk_C, wsk_C, r_C, \sigma^w_C)$}
      \If{$B_{C} + (\delta_C * n) < 0$}
        \State{Abort$()$ \Comment{consider new micropayment channel}}
      \EndIf{}
      \State{$\epsilon_C \gets \delta_C * n$}
      \State{$(nwpk_C, nwsk_C, nwCom_C, n\pi_C) \gets$ Wal$(pp, pk_I, w_C, \epsilon_C)$}
      \State{$\delta_R \gets -(\delta_C  + tax)$} \Comment{the tax is a net profit for $I$}
      \State{$HC \gets $MakeHC$($Random$(), n)$}
      \State{$nT \gets (\delta_C, \delta_R, n, HC[0])$}
      \State{Intermediary.Send$(wpk_C, nwpk_C, nwCom_C, n\pi_C, nT)$}
    \EndProcedure{}

    \Procedure{Intermediary}{$pp, S_I, nS_I$}
      \State{$(wpk_C, nwpk_C, nwCom_C, n\pi_C, nT) \gets $Client.Receive$()$}
      \State{parse $nT$ as $(\delta_C, \delta_R, n, hc^0)$}
      \If{$wpk_C \in S_I \vee \neg $Verify$(n\pi_C)$}
        \State{Abort$()$ \Comment{invalid wallet}}
      \EndIf{}
      \If{$-\delta_C \ne price \vee \delta_R + \delta_C + tax \ne 0$}
        \State{Abort$()$ \Comment{incorrect payment prices}}
      \EndIf{}
      \State{$S_I \gets S_I \cup \{wpk_C : \bot, nwpk_C: \bot\}$}
      \State{$nS_I \gets nS_I \cup \{nT : \bot\}$}
      \State{Client.Send$(verified)$}
    \EndProcedure{}

    \Procedure{Client}{}
      \State{$ver \gets $Intermediary.Receive$()$}
      \State{$\epsilon^k_C = B_C + (\delta_C * k)$}
      \State{$nrt_C \gets $Intermediary.Blindsig$(ver, refund || nT || nwpk_C || \epsilon^k_E)$}
      \State{$nS_C \gets (0, HC[0])$}
      \State{$ncsk_C \gets (nwpk_C, nwsk_C, HC)$}
      \State{$\sigma^{rev(w)}_C \gets $Sign$(wsk_C, revoke||wpk_C)$}
      \State{Intermediary.Send$(\sigma^{recv(w)}_C$)}
    \EndProcedure{}

    \Procedure{Intermediary}{}
      \State{$\sigma^{recv(w)}_C \gets $Client.Receive$()$}
      \If{$\neg $Verify$(wpk, revoke||wpk_C, \sigma^{recv(w)}_C) = 1$}
        \State{Abort$()$ \Comment{invalid revocation token}}
      \EndIf{}
      \State{$S_I[wpk_C] \gets \sigma^{recv(w)}_C$}
      \State{$nS_I[nT] \gets ready$}
      \State{Client.Send$(established)$}
    \EndProcedure{}

  \end{algorithmic}
\end{algorithm}

\begin{algorithm}
  \caption{Nanopayment Channel Establish --- Protocol between a client,
    intermediary, and relay to establish the nanopayment channel between the
    client and relay. Run at the start of circuit setup.}
  \begin{algorithmic}[1]

    \Procedure{Client}{$nT$}
      \State{Relay.Send$(nT)$}
    \EndProcedure{}

    \Procedure{Relay}{$pp, pk_I, B_{I:B}, w_R$}
      \State{$nT \gets $Client.Receive$()$}
      \State{parse $w_R$ as $(B_R, wpk_R, wsk_R, r_R, \sigma^w_R)$}
      \State{parse $nT$ as $(\delta_C, \delta_R, n, hc^0)$}
      \If{$B_{I:B} - (\delta_B * n) < 0$}
        \State{Abort$()$ \Comment{consider new micropayment channel}}
      \EndIf{}
      \State{$\epsilon_R \gets \delta_R * n$}
      \State{$(nwpk_R, nwsk_R, nwCom_R, n\pi_R) \gets $Wal$(pp, pk_I, w_R, \epsilon_R)$}
      \State{Intermediary.Send$(wpk_R, nwpk_R, nwCom_R, n\pi_R, nT)$}
    \EndProcedure{}

    \Procedure{Intermediary}{$pp, S_I, nS_I$}
      \State{$(wpk_R, nwpk_R, nwCom_R, n\pi_R, nT) \gets $Relay.Receive$()$}
      \State{parse $nT$ as $(\delta_C, \delta_R, n, hc^0)$}
      \If{$wpk_R \in S_I \vee \neg $Verify$(n\pi_R)$}
        \State{Abort (invalid wallet)}
      \EndIf{}
      \If{$nS_I[nT] \ne ready$}
        \State{Abort (unregistered nanopayment channel)}
      \EndIf{}
      \State{$S_I \gets S_I \cup \{nwpk_R, \bot\}$}
      \State{$nS_I[nT] \gets established$}
      \State{Relay.Send$(verified)$}
    \EndProcedure{}

    \Procedure{Relay}{}
      \State{$ver \gets $Intermediary.Receive$()$}
      \State{$\epsilon^k_R = B_R + (\delta_R * k)$}
      \State{$nrt_R \gets $Intermediary.Blindsig$(ver, refund || nT || nwpk_R || \epsilon^k_R)$}
      \State{$ncsk_R \gets (nwpk_R, nwsk_R, \bot)$} \Comment{match client format}
      \State{$nS_R \gets (0, hc^0)$}
    \EndProcedure{}
  \end{algorithmic}
\end{algorithm}

\begin{algorithm}
  \caption{Nanopayment Channel Pay --- Protocol between the client and relay to
    forward a single nanopayment. Run periodically throughout the lifetime of
    the circuit.}
  \begin{algorithmic}[1]

    \Procedure{Client}{$nT, ncsk_C, nS_C$}
      \State{parse $nT$ as $(\delta_C, \delta_R, n, hc^0)$}
      \State{parse $ncsk_C$ as $(nwpk_C, nwsk_C, HC)$}
      \State{parse $nS_C$ as $(k, hc^k)$}

      \If{$k >= n$}
        \State{Abort$()$ \Comment{out of nanopayments, setup a new channel}}
      \EndIf{}

      \State{$nS_C \gets (k+1, HC[k+1])$}
      \State{Relay.Send$(HC[k+1])$}
    \EndProcedure{}

    \Procedure{Relay}{$nT, nS_R$}
      \State{$hc^{k+1} \gets $Client.Receive$()$}
      \State{parse $nS_R$ as $(k, hs^k)$}
      \If{$k+1 >= n \vee Hash(hc^{k+1}) \ne hc^k$}
        \State{Abort$()$ \Comment{invalid nanopayment}}
      \EndIf{}
      \State{$nS_R \gets (hs^{k+1}, k+1)$}
    \EndProcedure{}
  \end{algorithmic}
\end{algorithm}

\begin{algorithm}
  \caption{Nanopayment Channel Close --- Protocol between an end user (client or
    relay) and an intermediary to close out the nanopayment channel and receive
    a micropayment wallet. Run any time after the circuit closure. Also, the
    relay must close first}
  \begin{algorithmic}[1]
    \State{$\forall E \in \{Client, Relay\}$}
    \Procedure{EndUser}{$pp, pk_I, w_E, nT, ncsk_E, nS_E$}
      \State{parse $w_E$ as $(B_E, wpk_E, wsk_E, r, \sigma^w_E)$}
      \State{parse $nT$ as $(\delta_C, \delta_R, n, hc^0)$}
      \State{parse $ncsk_E$ as $(nwpk_E, nwsk_E, \_)$}
      \State{parse $nS_E$ as $(k, hc^k)$}
      \State{$\epsilon_E \gets \delta_C * k$ if (EndUser = Client) else $ \delta_R * k$}
      \State{$(wpk'_E, wsk'_E, wCom'_E, \pi'_E) \gets $Wal$(pp, pk_I, wpk_B, \sigma^w_E, B_E, \epsilon_E$)}
      \State{Intermediary.Send$(wpk_E, wCom'_E, \pi'_E, nT, \epsilon_E, k, hc^k)$}
    \EndProcedure{}

    \Procedure{Intermediary}{$pp, S_I, nS_I$}
      \State{$(wpk_E, wCom'_E, \pi_E, nT, \epsilon_E, k, hc^k) \gets $EndUser.Receive$()$}
      \State{parse $nT$ as $(\delta_C, \delta_R, n, hc^0)$}
      \If{$\epsilon_E < 0 \wedge closed \not\in nS_I[nT]$}
        \State{Abort$()$ \Comment{client attempting to close before relay}}
      \EndIf{}
      \If{Verify$(\pi_E) \vee nS_I[nT] \ne established$}
        \State{Abort$()$ \Comment{invalid wallet or channel}}
      \EndIf{}
      \If{$k > n \vee \neg $VerifyHC$(hc^0, k, hc^k)$}
        \State{Abort$()$ \Comment{invalid payment hash chain}}
      \EndIf{}
      \State{$nS_I[nT] \gets closed||hc^k$}
      \State{EndUser.Send$(verified)$}
    \EndProcedure{}

    \Procedure{EndUser}{}
      \State{$ver \gets $Intermediary.Receive$()$}
      \State{parse $ncsk_E$ as $(nwpk_E, nwsk_E, \bot)$}
      \State{$\epsilon^k_E = B_E + (\delta_E * k)$}
      \State{$rt'_E \gets $Intermediary.Blindsig$(ver, refund || wpk'_E || \epsilon^k_E)$}
      \State{$\sigma^{rev(nrt)}_E \gets $Sign$(nwsk_E, revoke||nwpk_E)$}
      \State{Intermediary.Send$(nwpk_E, \sigma^{rev(nrt)})$}
    \EndProcedure{}

    \Procedure{Intermediary}{}
      \State{$(nwpk_E, \sigma^{rev(nrt)}_E) \gets $EndUser.Receive$()$}
      \If{$nwpk_E n\in S_I \vee \neg $Verify$(nwpk_E, \sigma^{rev(nrt)})$}
        \State{Abort$()$ \Comment{unregistered channel or revocation token}}
      \EndIf{}
      \State{$S_I[nwpk_E] \gets \sigma^{rev(nrt)}$}
      \State{EndUser.Send$(verified)$}
    \EndProcedure{}

    \Procedure{EndUser}{}
      \State{$ver \gets $Intermediary.Receive$()$}
      \State{$w'_E \gets $Intermediary.Blindsig$(ver, wpk_E'||B_E + \epsilon_E)$}
    \EndProcedure{}

  \end{algorithmic}
\end{algorithm}

\begin{algorithm}
  \caption{Nanopayment Refund --- Algorithm by an end user to close a micropayment
    channel and claim ledger funds. This is a modified version of
    Bolt's Refund algorithm to also allows for granular claims on
    open nanopayment channels}
  \begin{algorithmic}[1]
    \State{$\forall E \in \{Client, Relay\}$}
    \Function{EndUser}{$pp, csk_E, w_E, nT, ncsk_E, nS_E, nrt_E$}
    \State{parse $csk_E$ as $(\_, sk_E, \_, \_, \_, \_)$}
    \State{parse $w_E$ as $(B_E, \_, \_, \_, \_)$}
    \State{parse $nT$ as $(\delta_C, \delta_B, \_, \_)$}
    \State{parse $ncsk_E$ as $(nwpk_E, \_, \_)$}
    \State{parse $nS_E$ as $(k, hc^k)$}
    \State{$\delta_E \gets \delta_C$ if (EndUser = Client) else $ \delta_R$}
    \State{$m_E \gets (refund || nT || nwpk_E || B_E + \delta_E * n, nrt_E, hc^k_E, k_E)$}
    \State{$nrc_E \gets (m_E, Sign(sk_E, m_E))$}
    \State{\Return{$nrc_E$}}
    \EndFunction{}
\end{algorithmic}
\end{algorithm}

\begin{algorithm}
  \caption{Nanopayment Refute --- Algorithm by an intermediary to respond to an end user's refund claim by posting its own channel closure message to the ledger}
  \begin{algorithmic}[1]
    \State{$\forall E \in \{Client, Relay\}$}
    \Function{Intermediary}{$pp, T_E, S_I, nS_I, nrc_E$}
    \State{parse $nrc_E$ as $(m_E, \sigma^m_E)$}
    \State{parse $m_E$ as $(refund || nT || nwpk_E || B^{full}_E, nrt_E, k_E, hc^k_E)$}
    \State{\Comment{$B^{full}_E \gets$ balance if nanopayment channel were saturated}}
    % however we get nrc_E... need to decide
    \State{parse $T_E$ as $(pk_E, \_)$}
    \If{$\neg$Verify$(pk_E, m_E, \sigma^m_E)$}
      \State{Abort$()$ \Comment{bad signature, well be rejeced by ledger}}
    \EndIf{}
    \If{$\neg$Verify$(pk_I, (refund || nT || nwpk_E || B^{full}_E), nrt_E)$}
      \State{Abort$()$ \Comment{unapproved refund token}}
    \EndIf{}
    \If{$S_I[nwpk_E] \ne \bot$}
      \State{\Comment{$E$ is posting an old token, $I$ should refute}}
      \State{$\sigma^{rev(nrt)}_E \gets S_I[nwpk_E]$}
      \State{$nrc_I \gets ((revoked, \sigma^{rev(nrt)}_E), Sign((revoked, \sigma^{rev(nrt)})))$}
    \EndIf{}
    \State{\Comment{Everything checks out; accept the closure}}
    \State{$hc^k \gets nS_I[nT]$}
    \State{$nrc_I \gets ((accepted, k_I, hc^k_I), Sign(accepted, k_I, hc^k_E))$}
    \State{\Return{$nrc_I$}}
    \EndFunction{}
  \end{algorithmic}
\end{algorithm}

\begin{algorithm}
  \caption{Nanopayment Resolve --- Algorithm run by the ledger (and everyone verifying the ledger) to resolve all channel closure messages and allocate the appropriate final balances}
  \begin{algorithmic}[1]
    \State{\Comment{Returns the tuple $(B^{final}_E, B^{final}_I)$}}
    \Function{Ledger}{$pp, T_E, T_I, nrc_E, nrc_I$}
    \State{$B^{total} = B_E^{init} + B_I^{init}$}
    \State{parse $nrc_E$ as $(m_E, \sigma^m_E)$}
    \State{parse $nrc_I$ as $(m_I, \sigma^m_I)$}
    \State{parse $m_E$ as $(refund || nT || nwpk_E || B^{full}_E, nrt_E, k_E, hc^k_E)$}
    \State{\Comment{$B^{full}_E \gets$ balance if nanopayment channel were saturated}}
    \State{parse $nT$ as $(\delta_C, \delta_R, n, hc^0)$}
    \State{$\delta_E \gets \delta_C$ if (EndUser = Client) else $ \delta_R$}

    \If{$nrc_E = \bot$}
      \State{\Comment{$E$ failed to respond closure request in time}}
      \State{\Return{$(0, B_{total})$}}
    \EndIf{}
    \If{$\neg $Verify$(pk_E, m_E, \sigma^m_E) \vee \neg $Verify$(pk_I, m_I, \sigma^m_I)$}
      \State{\Return{$\bot$} \Comment{messages could not be authenticated}}
    \EndIf{}
    \If{$\neg $Verify$(pk_I, refund || nT || nwpk_E || B^{full}_E, nrt_E)$}
      \State{\Return{$(0, B_{total})$} \Comment{$E$ is attempting to use invalid token}}
    \EndIf{}
    \If{$revoked \in m_I$}
      \State{parse $m_I$ as $(revoked, \sigma^{rev(nrt)}_E)$}
      \If{Verify$(nwpk_E, \sigma^{rev(nrt)}_E)$}
        \State{\Return{$(0, B_{total})$} \Comment{$E$ is trying to use old channel}}
        \Else{}
        \State{\Return{$(B_{total}, 0)$} \Comment{invalid revocation from $I$}}
      \EndIf{}
    \EndIf{}
    \State{\Comment{micropayments settled, now resolve nanopayments}}
    \State{parse $m_I$ as $(accepted, k_I, hc^0_I)$}
    \If{$k_I \leq k_E \leq n \wedge $VerifyHC$(hc^0, k_E, hc^k_E) )$}
      \State{\Comment{$E$ has the highest hash preimage}}
      \State{$B^{final}_E = B^{full}_E - \delta_E * (n-k_E)$}
      \State{$B^{final}_I = B_{total} - B^{full}_E + \delta_E * (n-k_E)$}
    \EndIf{}
    \If{$k_E \leq k_I \leq n \wedge $VerifyHC$(hc^0, k_I, hc^k_I) )$}
      \State{\Comment{$I$ has the highest hash preimage}}
      \State{$B^{final}_E = B^{full}_E - \delta_E * (n-k_I)$}
      \State{$B^{final}_I = B_{total} - B^{full}_E + \delta_E * (n-k_I)$}

    \EndIf{}
    \State{\Return{$(B^{final}_E, B^{final}_I)$}}

    \EndFunction{}
  \end{algorithmic}
\end{algorithm}


\end{document}